

%Plantilla basada en "Template for Masters / Doctoral Thesis" (plantilla disponible en writeLaTex) que subió LaTeXTemplates.com

\documentclass[11pt]{book}
\usepackage[paperwidth=17cm, paperheight=22.5cm, bottom=2.5cm, right=2.5cm]{geometry}
\usepackage{amssymb,amsmath,amsthm} %paquete para símbolo matemáticos
\usepackage[spanish]{babel}
\usepackage[utf8]{inputenc} %Paquete para escribir acentos y otros símbolos directamente
\usepackage{enumerate}
\usepackage{graphicx}
\usepackage{mathrsfs}
\usepackage{cite}
\usepackage{subcaption} %Subfiguras
\usepackage{verbatim} %Esto es para poner comentarios largos y demas
\graphicspath{{Img/}} %En qué carpeta están las imágenes
\usepackage[nottoc]{tocbibind}
\usepackage[pdftex,
            pdfauthor={Camilo Leonel Amadio},
            pdftitle={Tesis Amadio Camilo},
            pdfsubject={Física},
            pdfkeywords={QTF},
            pdfproducer={Latex con hyperref},
            pdfcreator={pdflatex}]{hyperref}
\usepackage{amssymb}
\usepackage{dsfont}

\usepackage[dvipsnames]{xcolor}
%\newcommand{\r}[1]{\textcolor{red}{#1}}
\def\red{\color{red}}
\def\magenta{\color{magenta}}

%\includeonly{Capitulos/Conclusiones.tex} %{Apendices/ApB}%Capitulos/%cap3}


\begin{document}

%----------------------------------------------------------------------------------------
%	COMANDOS PERSONALIZADOS
%----------------------------------------------------------------------------------------

%SI TU TESIS TIENE TEOREMAS Y DEMOSTRACIONES, PUEDES DESCOMENTAR Y USAR LOS SIGUIENTES COMANDOS

%\renewcommand{\proofname}{Demostración}
%\providecommand{\norm}[1]{\lVert#1\rVert} %Provee el comando para producir una norma.
%\providecommand{\innp}[1]{\langle#1\rangle} 
%\newcommand{\seno}{\mathrm{sen}}
%\newcommand{\diff}{\mathrm{d}}

%\newtheorem{teo}{Teorema}[section] 
%\newtheorem{cor}[teo]{Corolario}
%\newtheorem{lem}[teo]{Lema}

%\theoremstyle{definition}
%\newtheorem{dfn}[teo]{Definición}

%\theoremstyle{remark}
%\newtheorem{obs}[teo]{Observación}

%\allowdisplaybreaks


%----------------------------------------------------------------------------------------
%	PORTADA
%----------------------------------------------------------------------------------------

\title{Tesis Amadio Camilo Leonel} %Con este nombre se guardará el proyecto en writeLaTex

\begin{titlepage}
\begin{center}

\textsc{\Large Universidad Nacional de La Plata}\\[4em]

%Figura
\begin{figure}[h]
\begin{center}
\includegraphics[width=5 cm]{Escudo.jpg}
\end{center}
\end{figure}

\vspace{1em}

\textsc{\huge \textbf{
Funciones Espectrales \\[2mm]
En Teorías Cuánticas De Campos \\[2mm]
Con Singularidades
}}\\[4em]

\textsc{\large Trabajo de Diploma}\\[1em]

\textsc{Para obtener el Título de }\\[1em]

\textsc{Licenciado en Física}\\[1em]

\textsc{presenta}\\[1em]

\textsc{\Large Camilo Leonel Amadio}\\[1em]

\textsc{\large Director: Pablo Pisani}

\end{center}

\vspace*{\fill}
\textsc{La Plata, Argentina \hspace*{\fill} 2019}

\end{titlepage}

\begin{comment}
%----------------------------------------------------------------------------------------
%	DECLARACIÓN
%----------------------------------------------------------------------------------------

\thispagestyle{empty}
\vspace*{\fill}
\begingroup
``Con fundamento en los artículos 21 y 27 de la Ley Federal del Derecho de Autor y como titular de los derechos moral y patrimonial de la obra titulada ``\textbf{TÍTULO DE LA TESIS}'', otorgo de manera gratuita y permanente al Instituto Tecnológico Autónomo de México y a la Biblioteca Raúl Bailléres Jr., la autorización para que fijen la obra en cualquier medio, incluido el electrónico, y la divulguen entre sus usuarios, profesores, estudiantes o terceras personas, sin que pueda percibir por tal divulgación una contraprestación''.

\centering

\hspace{3em}

\textsc{AUTOR}

\vspace{5em}

\rule[1em]{20em}{0.5pt} % Línea para la fecha

\textsc{Fecha}
 
\vspace{8em}

\rule[1em]{20em}{0.5pt} % Línea para la firma

\textsc{Firma}

\endgroup
\vspace*{\fill}



%----------------------------------------------------------------------------------------
%	DEDICATORIA
%----------------------------------------------------------------------------------------

\pagestyle{empty}
\frontmatter

\chapter*{}
\begin{flushright}
\textit{DEDICATORIA}
\end{flushright}

\end{comment}

%----------------------------------------------------------------------------------------
%	AGRADECIMIENTOS
%----------------------------------------------------------------------------------------

\chapter*{Agradecimientos}
%\markboth{AGRADECIMIENTOS23}{AGRADECIMIENTOS} % encabezado 

Me gustan los agradecimientos cortos, y la birra fría.

\begin{enumerate}
    \item A Mama que me demostró cual es el camino hoy y cada día de mi vida, con el ejemplo y no solo con palabras
    \item papa 
    \item  A mis tíos y tías que siempre me extendieron una mano cuando la necesite
    \item A Mi Abue Mirtha, todo esto te lo dedico a vos, que siempre estuviste 
    \item A  todos mis amigos que me acompañaron a lo largo de estos años, sin ellos me hubiera vuelto a las 2 semanas 
    \item A Wikipedia y Julioprofe
    \item Al Rasta, Lolo y el Diquin, que me acompañaron a salir de joda 
    \item A mis mejores amigos, Lucas, La Sol y la Celia
    \item A la cervecería Quilmes que sin ella probablemente ya me hubiera suicidado hace mucho
    \item A Ana y Analia, que son las mejores bibliotecarias que existen, no me imagino como hubiera sido sin ellas. 
    \item Por último y no menos importante, a todas esas mamas y papas de nuestros compañeros que nos daban de comer como si fueramos sus nietos
\end{enumerate}



%----------------------------------------------------------------------------------------
%	PREFACIO
%----------------------------------------------------------------------------------------

\chapter*{Prefacio}

\pagestyle{plain}
\markboth{PREFACIO23}{PREFACIO} % encabezado 



	En el año 2003, el profesor D.V. Vassilevich recopiló toda la información existente hasta la fecha sobre la utilización del {\it Heat Kernel} en el cálculo de la regularización de la acción efectiva  \cite{Vassilevich:2003xt}. 	
	Este report contiene 117 páginas de las cuales hay solamente 6 dedicadas a la presentación de problemas singulares, los cuales son:
\begin{enumerate}
\item Potenciales no integrables, como el oscilador armónico.
\item Singularidades cónicas como las que aparecen en las soluciones clásicas de las ecuaciones de Einstein.
\item Teorías de Mundo Brana, en las cuales la métrica es singular sobre la superficie.
\item Contornos no suaves, como por ejemplo, las esquinas en un dominio rectangular.
\item La propagación de ondas electromagnéticas en medios dielétricos, las cuales poseen velocidad de la luz variable.
\end{enumerate}	
El estudio de cada uno de estos sistemas conlleva a resultados contrarios a la teoría general presentada en \cite{Vassilevich:2003xt}.\\

En este trabajo se estudió la {\it función-$\zeta$} de un operador singular en un intervalo compacto, encontrando que los polos se ubican como habría de esperarse en el caso regular, pero con la aparición de un polo doble; lo cual implica que el desarrollo del {\it heat-kernel} posee términos logarítmicos tal como se conjeturó en 1980, por Constantine Callias y Clifford H. Taubes \cite{callias1980}. Luego en el último capítulo se desarrolló un formalismo numérico para poder calcular las energías de vació evadiendo la calculación explícita de la función $\zeta$.\\

\begin{comment}
 En el año 2002 en un trabajo realizado por H. Falomir, P.A.G. Pisani y M.A. Muschietti  \cite{doi:10.1063/1.1809257} se estudío la resolvente y extenciones autoadjuntas de operadores singulares de la forma $A = - \partial ^2 _x + g(g-1) x ^{-2}$ en un intervalo unidimensaional compacto, encontrando que la {\it función-$\zeta$} posee polos simples en $s _n = - \left( g - \frac{1}{2} \right) n$ donde $n \in \mathbb{N} _+$. Lo cual se aparta del comportamiento esperado para el caso regular en el cual $\zeta (s)$ poseería polos simples en $s _n = \frac{1 - n}{2}$ con  $n \in \mathbb{N} _+$.
\end{comment}



%----------------------------------------------------------------------------------------
%	TABLA DE CONTENIDOS
%---------------------------------------------------------------------------------------

\tableofcontents


%----------------------------------------------------------------------------------------
%	TESIS
%----------------------------------------------------------------------------------------
\mainmatter %empieza la numeración de las páginas
\pagestyle{headings}

%  Incluye los capítulos en el folder de capítulos

\chapter{Introducción}

    En este capitulo se van a explicar los conceptos matemáticos utilizados con un ejemplo, para luego aplicarlos al estudio del caso que nos interesa. Se va a calcular la función $\zeta _A (s)$ mediante distintas técnicas, calculando asintóticamente los autovalores, y luego ponerlos en la definición o en el Heat Kernel, y luego se procedió a utilizar el calculo completo.

\section{El Operador Diferencial}

En estudiaran los autovalores del operador diferencial $A$ determinado por:
\begin{equation}
\begin{array}{c}
    A \phi (x) = - \partial ^2 _x \ \phi (x)  \\
    \phi (0) = 0 \\ 
    \partial _x \phi (L) + \gamma \phi (L) = 0
\end{array}
\end{equation}

Donde las condiciones de contorno me fijan un espectro autovalores $\lambda > 0 $, que está dado cualquiera de las dos ecuaciones equivalentes: 

\begin{equation}
\begin{array}{cc}
    \frac{\lambda}{\gamma}  \ Cos( L \lambda ) +   Sin( L \lambda ) = 0 \\
    \frac{\lambda}{\gamma}  + Tg(\lambda L )  = 0 
\label{autovalores}
\end{array}
\end{equation}

Una vez obtenidos los autovalores el siguiente paso es calcular la funcion $\zeta$ definida por:

\begin{equation}
    \zeta _ {A } (s) = \sum_{n = 1} ^{ \infty } \lambda _n ^ {-s}
\end{equation}

Al no poder encontrar explícitamente los autovalores se proceden a distintas técnicas para hallar distintas aproximaciones de la función $\zeta _A (s)$

\section{Calculo Asintótico de los autovalores}


Haciendo el cambio de variables $\mu = \lambda L$ y $\theta = \gamma L $ las ecuaciones (\ref{autovalores}) se pueden expresar de la forma:

\begin{equation}
\begin{array}{c}
    Tg[\mu] + \frac{\mu}{\theta} = 0 \\
    \frac{\mu}{\theta} Cos[\mu] + Sin[\mu] = 0
\end{array}
\label{eq.asintota}
\end{equation}

Tal como se puede ver en la figura (\ref{fig:Dibujo}), los autovalores $\mu _n$ tienden a pegarse a la asíntota vertical de $ Tg [ \mu ] $ a medida que $\mu _n$ se hace cada vez mas grande.

\begin{figure}
    \centering
    \includegraphics[scale=0.3]{Dibujo.jpg}
    \caption{Aquí se busca encontrar los terminos que no se anulan a n = $\infty$ del desarrollo asintotico de autovalores de la ecuacion (\ref{eq.asintota}) }
    \label{fig:Dibujo}
\end{figure}

Se puede ver entonces de los autovalores $\mu _n$ se pueden descomponer en una parte correspondiente a las asíntotas verticales de $Tg(x)$ mas una corrección que tiende a cero a medida que n tiende a $\infty$

\begin{equation}
\begin{array}{c}
    \mu _n = n \pi + \frac{\pi}{2} + \epsilon _n \\
    Donde \ \epsilon _n \rightarrow{0} \ si \ cuando \ n \rightarrow{0}
\end{array}
\label{eq.mu}
\end{equation}


conocer $\epsilon _n $ es equivalente a resolver la ecuación (\ref{eq.asintota}), la cual no puede resolverse analíticamente, en vez de eso voy a obtener un desarrollo de $\epsilon _n $ para $n \rightarrow \infty$.

voy a intesertar (\ref{eq.mu}) en la segunda ecuacion de (\ref{eq.asintota}) y desarrollar alrededor de $\epsilon \rightarrow{0}$ obteniendo:

\begin{equation}
\begin{array}{c}
    Sin[ n \pi + \frac{\pi}{2} + \epsilon _n ] = 
    - \frac{\mu _n}{\theta}  \ Cos[ n \pi + \frac{\pi}{2} + \epsilon _n ]  \\
 \\

         \sum _{p=0} ^{\infty} \frac{(-1) ^n  \epsilon ^{2 p }}{(2p)!} 
    =  \frac{-1}{\theta}  \ (n \pi + \frac{\pi}{2} + \epsilon ) \
     \sum _{p=0} ^{\infty} \frac{(-1) ^n \epsilon ^{2 p + 1}}{(2p+1)!} 
\end{array}
\end{equation}


Donde acomodando la igualdad, obtengo la ecuacion :

\begin{equation}
    1 = \sum _{p=1} ^{\infty} \epsilon ^{2p+2} \ (-1) ^p
    \left( 
    \frac{1}{(2p+2)!} + \frac{1}{\theta} \frac{(1 )}{(2p+1)!} 
    \right ) +
    \frac{1}{\theta} \left(n \pi + \frac{\pi}{2} \right)
    \sum _{p=0} ^{\infty} \frac{(-1) ^p \epsilon ^{2p+1}}{(2p+1) !}
\label{igualdad epsilon}
\end{equation}

Suponiendo que $\epsilon _n $ tiene un desarrollo en serie de la forma (\ref{eq.epsilon}).

\begin{equation}
    \epsilon = 
    \frac{\epsilon ^{(1)}}{n}  + 
    \frac{\epsilon ^{(2)}}{n ^2}  + 
    \frac{\epsilon ^{(3)}}{n ^3}  + ...
\label{eq.epsilon}
\end{equation}


Insertando este desarrollo de $\epsilon$ en la ecuación (\ref{igualdad epsilon}), e igualando orden a orden se obtiene, para los primeros ordenes de $\epsilon$:

\begin{equation}
    \epsilon  = \frac{\theta}{n \pi} 
     - \frac{ \theta}{2 \pi n ^2 } 
    + \frac{\theta}{n^3 \pi} 
        \left( \frac{1}{4} - 
        \frac{\theta ^2}{2 \pi ^2} -
        \frac{\theta}{\pi ^2}
        \right) 
\label{epsilons}
\end{equation}

Una vez obtenido el desarrollo de $\mu _n $ puedo calcular $\lambda _n = \frac{\mu _n }{L}  $ 



\begin{equation}
    \lambda _n = 
    \frac{1}{L}
    \left(
    \alpha n + 
    \beta + 
    \frac{\gamma}{n} +
    \frac{\delta}{n ^2} +
    \frac{\eta}{n^3} +
    O(\frac{1}{n^4} ) 
    \right)
\end{equation}
    
Donde $\alpha, \beta , ... $ quedan escritos en funcion de los $\epsilon ^{(1)},\epsilon^{(2)}, ... $ ; 
Luego se calcula la funcion $\zeta _A (s) $ utilizando el desarrollo asintotico de los autovalores.
    
\begin{equation}
\begin{array}{cc}
    \zeta _{A} (s) =  \sum _{n=1} ^{\infty} \lambda _n ^ {-2 s} =
    L ^{2s}
    \sum _{n=1} ^{\infty} 
    \left(
    \alpha n + 
    \beta + 
    \frac{\gamma}{n} +
    \frac{\delta}{n ^2} +
    \frac{\eta}{n^3} +
    O( \frac{1}{n ^{4} }  )
    \right) ^{-2 s} = \\
    ( \frac{L}{\alpha} ) ^{2s}    
    \sum _{n=1} ^{\infty} 
    n ^{- 2 s} 
    \left(
    1 +     
    \underbrace{
        \frac{\beta}{\alpha n} + 
        \frac{\gamma}{\alpha n^2} +
        \frac{\delta}{\alpha n ^3} +
        \frac{\eta}{\alpha n^4} +
        O(\frac{1}{n ^{5}} ) } _{ \chi _n}
    \right ) ^{-2 s}
\end{array}
\end{equation}

Para calcular esta serie voy a hacer un desarrollo binomial alrededor de $\chi _n \rightarrow{0} $ 

\begin{equation}
\begin{array}{c}
\zeta _{A} (s) = 
( \frac{L}{\alpha} ) ^{2s}
\sum _{n=1} ^{\infty}
  n  ^{-2 S} \\
(
1 - 2 s \chi _n + 2 s(s+1) \frac{\chi _n ^2}{2} - 2s(2s+1)(2s+2) \frac{ \chi _n ^3}{3!}  + \\ 
2s(2s+1)(2s+2)(2s+3) \frac{\chi ^4 _n}{4!}
+ O( \frac{1}{n ^5}) )

\end{array}
\end{equation}

Donde hay que tener en cuenta que cada termino $\chi _{n} ^{m} $ contribuye al orden mas bajo en la sumatoria en una potencia $\frac{1}{n ^m}$.


Calculando explicitamente cada termino, obtengo:

\begin{equation}
\sum _{n=1} \infty n ^{-2 s} = \zeta (2s)
\end{equation}


\begin{equation}
\begin{array}{c}
\sum _{n=1} ^{\infty}
 n  ^{-2 S}  \chi _n =  \\
\frac{\beta}{\alpha} \zeta (2s+2) + \frac{\gamma}{\alpha} \zeta(2s+2) + \frac{\delta}{\alpha} +
\frac{\eta}{\alpha} \zeta (2s+4) + \sum _{n=1} ^{\infty} n ^{-2s} O(\frac{1}{n ^5})
\end{array}
\end{equation}

\begin{equation}
\begin{array}{c}
    \sum _{n=1} ^{\infty}
    n   ^{-2 S} \chi _n ^2 = \\ 
    \frac{\beta ^2}{\alpha ^2} \zeta(2s+2) +
    \frac{2 \gamma \beta }{\alpha} \zeta (2s+3) + 
    \left( \frac{2 \beta \delta}{\alpha ^2} + \frac{\gamma ^2}{\alpha ^2} \right) \zeta (2s+4) + 
    \sum _{n=1} ^{\infty} n ^{-2s} O( \frac{1}{n^s} )
\end{array}
\end{equation}

\begin{equation}
\begin{array}{cc}
    \sum _{n=1} ^{\infty} 
    n ^{-2 S} \chi _n ^3 =
    \frac{\beta ^3}{\alpha ^3} \zeta (2s+3) + 
    \frac{3 \gamma \beta ^2}{\alpha ^3} \zeta (2s+4)
    + \sum _{n=1} ^{\infty} n ^{-2s} O( \frac{1}{n ^5} )
    
\end{array}
\end{equation}

\begin{equation}
\sum _{n=1} \infty n ^{-2s} \chi _n ^4 \\
\frac{\beta ^4}{\alpha ^4} \zeta (2s+4) + 
\sum _{n=1} ^{\infty} n ^{-2s} O( \frac{1}{n ^5} )
\end{equation}



Una vez calculados todos los terminos los sumo y saco factor común cada $\zeta( 2s+n )$


Obteniendo como resultado 

\begin{equation}
    \zeta _A (s) = 
    C _s \ \zeta (s) +
    C _{s+1} \ \zeta (s+1) + 
    C _{s+2} \ \zeta (s+2) +
    C _{s+3} \ \zeta (s+3) + 
    C _{s+4} \ \zeta (s+4) + ....
\end{equation}

Se puede ver que el resultado esta expresado como suma de funciones $\zeta (s+n)$ acompañados de unos argumentos $C_{s+n}$ que dependen de losvalores $\alpha,\beta,\gamma, ...$ que estan calculados en la ecuacion (\ref{epsilons}.

\section{Calculo de la funcion zeta mediante calculo complejo:}

Conocida la función de la cual hay que despejar los autovalores, se puede proceder a calcular la funcion $\zeta$ utilizando variable de calculo complejo, utilizando el camino representado en la figura (\ref{fig:contorno}) el cual al no poseer mas ceros, se puede deformar hasta el camino (\ref{fig:contorno}) 

\begin{equation}
\begin{array}{c}
   \zeta _A (s) =  \frac{1}{2 \pi i} \int _{C} \frac{f'(x)}{f(x)} z ^{-2s} = \\ 
   \\ 
    \frac{1}{2 \pi i} \int _{C}
    \frac{ cos(\gamma z) \left(\gamma + \frac{1}{L} \right) - sin(\gamma z) \frac{z \gamma}{L}
    }
    {cos(\gamma z) \frac{z}{L} + sin(\gamma z)
    }
    z ^{-2 s} dz
\end{array}
\end{equation}




\begin{figure}
\centering
\includegraphics[scale=0.7]{contorno.jpg}
\caption{Camino tenido en cuenta para realizar la integral de contorno en el plano complejo}
\label{fig:contorno}
\end{figure}


La integral de linea puede descomponerse en 3 partes, un contorno circular y dos lineas rectas, la contribución circular es regular para todo s, entonces no aporta a la estructura de polos, en cuanto a las contribuciones del lado recto puedo expresar ambos tramos como la integral (\ref{contorno}) 

\begin{equation}
    \frac{1}{ \pi } 
    Sin(\pi s)
    \int _1 ^{\infty} 
    t ^{-2s}
    \underbrace{
    \left( 
    \frac{(1+L \gamma) Cosh( \gamma t)+ t \gamma Sin(\gamma t)}
    {t \ Cosh(t \gamma)+ L Sinh(t \gamma)}
    \right)} _{\chi}
    dt 
\label{contorno}
\end{equation}

Luego voy a desarrollar asintoticamente  $\chi$  (\ref{eq:chi}), para despues insertarlo en (\ref{contorno} da como resultado y realizar la integral termino a termino para luego obtener (\ref{eq.zeta.com})

\begin{equation}
    \gamma + \frac{1}{t}-\frac{L}{t^2}+\frac{L^2}{t^3}+O(\frac{1}{t^4})
\label{eq:chi}
\end{equation}

\begin{equation}
    \zeta _A (s) = 
    \frac{Sin(\pi s)}{\pi}
    \left(
    \frac{\gamma}{2s-1} + 
    \frac{1}{2s} -
    \frac{L}{2s+1} +
    \frac{L^2}{2s+2} + ...
    \right)
\label{eq.zeta.com}
\end{equation}

De donde se pueden comparar los polos y residuos con el caso anterior 

\section{Calculo del Heat-Kernel}

\chapter{Ejemplos Sencillos}
    
En este capitulo se van a aplicar las herramientas vistas en el capítulo anterior. Se va a calcular de distintas formas la función $ \zeta _A (s) $, para luego obtener la energía de vacio, sobre un operador diferencial $A = - \partial ^2 _x$ sometido a distintas condiciones de contorno.
    
A partir de aquí se utilizaran las unidades naturales $\hbar=c=1$.

\section{Dirichlet,Neumann y Periódicas}

En esta sección estudiará el espectro y las energías de vacío del operador $A = - \partial ^2 _x$, al aplicarle condiciones de contorno Dirichlet,Neumann y Periódicas: \\

\textbf{Dirichlet:}

El operador $A$ estará dado por:

\begin{equation}
\begin{array}{c}
	A \phi (x) = - \partial _x ^2 \phi (x) \\
    \phi (0) = \phi(L) = 0 
\end{array}
\end{equation}


Cuyos autovalores y autofunciones están dados por  : 

\begin{equation}
\begin{array}{c}
	\phi _n (x) = \sqrt{\frac{2}{L}} Sin( \frac{n \pi x}{L} ) \\
	\lambda _n ^2 = \left( \frac{n \pi }{L} \right) ^2 \\
	n = 1,2,3, ...
\end{array}
\end{equation}

La función $\zeta _A (s)$ queda determinada por:

\begin{equation}
\begin{array}{c}
\zeta _A (s) = 
\sum _{n=1} ^{\infty} \left( \frac{\lambda _n}{\mu} \right) ^{-2s} =  \\
\left(  \frac{\pi}{L \mu} \right) ^{-2s}   \sum _{n=1} ^{\infty} n ^{-2s} = 
\left( \frac{\pi}{L \mu} \right) ^{-2s}  \zeta (2s)
\end{array}
\end{equation}

Que es regular en $s=-1/2$, siendo entonces la energía de vacío:

\begin{equation}
E _0 = - \frac{\pi}{12 L}
\end{equation}

Lo cual conduce a una Energía de Vacío atractiva, notar que aquí la constante $\mu$ no aparece en elx resultado final, debido a que la energía de vacio fue finita.\\

\textbf{Neumann:}

El operador $A$ estará dado por:

\begin{equation}
\begin{array}{c}
	A \phi (x) = - \partial _x ^2 \phi (x) \\
    \phi ' (0) = \phi ' (L) = 0 
\end{array}
\end{equation}



Cuyos autovalores y autofunciones están dados por  : 

\begin{equation}
\begin{array}{c}
	\phi _0 (x) = \sqrt{ \frac{1}{L} } \\
	\phi _n (x)  = \sqrt{\frac{2}{L}} Cos( \frac{n \pi x}{L} ) \\
	\lambda _n ^2 = \left( \frac{n \pi }{L} \right) ^2 \\
	n = 1,2,3, ...
\end{array}
\end{equation}

La función $\zeta _A (s)$ será la misma que la calculada anteriormente (debído a que excluyo los modos cero), así como también la energía de vacío. \\

\textbf{Periódicas:}

El operador $A$ estará dado por:

\begin{equation}
\begin{array}{c}
    \phi (0) = \phi (L)  \\ 
    \phi ' (0) = \phi ' (L)
\end{array}
\end{equation}

Cuyos autovalores y autofunciones están dados por  : 

\begin{equation}
\begin{array}{c}
	\phi _{0} = \sqrt{\frac{1}{L}} \\ 
	\phi _{n} (x) = \sqrt{\frac{2}{L}} Cos( \frac{2 n \pi x}{L} ) \\
	\lambda _n ^2 = \left( \frac{2 n \pi }{L} \right) ^2 \\
	n = 1,2,3, ...
\end{array}
\end{equation}

La función $\zeta _A (s)$ queda determinada por:

\begin{equation}
\begin{array}{c}
\zeta _A (s) = 
\sum _{n=0} ^{\infty} \left( \frac{\lambda}{\mu} \right)^{-2s} =  
\left( \frac{2 \pi}{L} \right) ^{-2s} \mu ^{2s} \sum _{n=1} ^{\infty} n ^{-2s} =  \\
\mu ^{2s} \left( \frac{2 \pi}{L} \right) ^{-2s} \zeta (2s)
\end{array}
\end{equation}

Que al igual que en los casos anteriores es regular en $s=-1/2$, obteniendo entonces para la Energía de Vacío:

\begin{equation}
E _0 = - \frac{\pi}{6 L}
\end{equation}

Lo cual nuevamente conduce a una Energía de Vacío atractiva.

\section{Condiciones de Contorno Mixtas}

En este caso el operador va a depender de un parametro arbitrario $\gamma$, el cual va a estar dado por:

\begin{equation}
\begin{array}{c}
    A \phi (x) = - \partial ^2 _x \ \phi (x)  \\
    \phi (0) = 0 \\ 
    \partial _x \phi (L) + \gamma \phi (L) = 0
\end{array}
\end{equation}

El cual posee autofunciones de la forma:

\begin{equation}
\phi _n (x) = 
B _n Sin( \lambda _n x )
\end{equation}

El espectro autovalores $\lambda _n > 0 $ está dado por  cualquiera de las dos ecuaciones equivalentes: 

\begin{equation}
\begin{array}{cc}
    \frac{\lambda}{\gamma}  \ Cos( L \lambda ) +   Sin( L \lambda ) = 0 \\
    \frac{\lambda}{\gamma}  + Tg(\lambda L )  = 0 
\label{autovalores}
\end{array}
\end{equation}



Una vez obtenidos los autovalores el siguiente paso es calcular la función $\zeta _A (s) $ definida por:

\begin{equation}
    \zeta _ {A } (s) = \mu ^{2s} \sum_{n = 1} ^{ \infty } \lambda _n ^ {-2 s}
\end{equation}

Al no poder encontrar explícitamente los autovalores voy a utilizar 3 técnicas distintas para hallar aproximaciones de la función $\zeta _A (s)$.

\subsection{Calculo Asintótico de los autovalores}


Haciendo el cambio a variables adimensionales $\tau = \lambda L $ y $\theta = \gamma L $, las ecuaciones (\ref{autovalores}) se pueden expresar de la forma:

\begin{equation}
\begin{array}{c}
    Tg(\tau) + \frac{\tau}{\theta} = 0 \\
    \frac{\tau}{\theta} Cos( \tau ) + Sin( \tau ) = 0
\end{array}
\label{eq.asintota}
\end{equation}

Tal como se puede ver en la figura [\ref{fig:Dibujo1}], los autovalores $\tau _n$ tienden a pegarse a la asíntota vertical de $ Tg ( \tau ) $ a medida que $\tau _n$ se hace cada vez mas grande, por lo cual los autovalores $\tau _n$ se pueden descomponer en una parte correspondiente a las asíntotas verticales de $Tg( \tau )$ mas una corrección que tiende a cero a medida que n  $ \rightarrow \infty$ :

\begin{figure}
    \centering
    \includegraphics[scale=0.6]{Dibujo.jpg}
    \caption{En este ejemplo, se puede ver que la intersección entre $Tan(x)$ y $-x$ tiende a las asíntotas verticales de la tangente, el mismo comportamiento se aprecia para cualquier recta de la forma $- a x$.}
    \label{fig:Dibujo1}
\end{figure}

\begin{equation}
\begin{array}{c}
    \tau _n = n \pi + \frac{\pi}{2} + \epsilon _n \\
    Donde \ \epsilon _n \rightarrow{0}  \ cuando \ n \rightarrow{0}
\end{array}
\label{eq.mu}
\end{equation}


Conocer $\epsilon _n $ es equivalente a resolver la ecuación (\ref{eq.asintota}), la cual no posee solución analítica, en vez de eso se va a obtener un desarrollo de $\epsilon _n $ para $n \rightarrow \infty$.

Utilizando (\ref{eq.mu}) en la segunda ecuación de (\ref{eq.asintota}) y desarrollar alrededor de $\epsilon \rightarrow{0}$ se obtiene:

\begin{equation}
\begin{array}{c}
    Sin( n \pi + \frac{\pi}{2} + \epsilon _n ) = 
    - \frac{n \pi + \frac{\pi}{2} + \epsilon _n}{\theta}  \ Cos( n \pi + \frac{\pi}{2} + \epsilon _n )  \\
 \\

         (-1) ^n \sum _{p=0} ^{\infty} \frac{(-1) ^p  \epsilon _n ^{2 p }}{(2p)!} 
    =  \frac{-(n \pi + \frac{\pi}{2} + \epsilon _n) }{\theta}  \  \	
    (-1) ^n
     \sum _{p=0} ^{\infty} \frac{(-1) ^ {p+1} \epsilon _n ^{2 p + 1}}{(2p+1)!} 
\end{array}
\end{equation}


Donde reacomodando se llega a la igualdad:

\begin{equation}
    1 = 
    \sum _{p=0} ^{\infty} (-1) ^p     \left[
   	\epsilon _n ^{2p+2 }\left( \frac{1}{(2p+1)! \theta } + \frac{1}{(2p+2)} \right) +
  	\frac{n \pi + \frac{\pi}{2}}{\theta} \frac{  \epsilon _n ^{2p+1}}{(2p+1)!} 			\right]
\label{igualdad epsilon}
\end{equation}

Aquí se puede ver que  $\epsilon _n $ posee un desarrollo en serie de la forma:

\begin{equation}
    \epsilon _n = 
    \frac{\epsilon ^{(1)}}{n}  + 
    \frac{\epsilon ^{(2)}}{n ^2}  + 
    \frac{\epsilon ^{(3)}}{n ^3}  + ...
\label{eq.epsilon}
\end{equation}


Reemplazadno este desarrollo de $\epsilon _n$ en (\ref{igualdad epsilon}), e igualando orden a orden se obtiene para los dos primeros oredenes:

\begin{equation}
    \epsilon _n = \frac{\theta}{n \pi} 
     - \frac{ \theta}{2 \pi n ^2 } + O \left( \frac{1}{n ^3}\right) 
\label{epsilons}
\end{equation}

\newpage


La función $ \zeta _A (s)$ queda expresada a este orden como:
    
\begin{equation}
\begin{array}{cc}
    \zeta _{A} (s) =  
    \sum _{n=1} ^{\infty} 
    \left( \frac{\lambda _n }{\mu} 
    	\right) ^ {-2 s}  =
    \sum _{n=1} ^{\infty} 
    \left(
	\frac{n \pi}{L \mu} + 
    \frac{\pi}{2 L \mu} +
    \frac{\gamma}{n \pi \mu } -
    \frac{\gamma}{2 n ^2 \pi \mu } +
    O \left(  \frac{1}{n^3} \right) 
    \right) ^{-2 s} = \\
    ( \frac{L \mu }{\pi} ) ^{2s}    
    \sum _{n=1} ^{\infty} 
    n ^{- 2 s} 
    \left(
    1 +     
    \underbrace{
        \frac{1}{2 n} + 
        \frac{L \gamma}{n^2 \pi ^2} -
        \frac{L \gamma}{2 n ^3 \pi ^2} +
        O(\frac{1}{n ^{4}} ) } _{ \chi _n}
    \right ) ^{-2 s}
\end{array}
\end{equation}

Para calcular esta serie se desarrolla el binomio alrededor de $\chi _n \rightarrow{0} $ hasta el orden cubico, dado que  cada termino $\chi _{n} ^{m} $ contribuye al orden mas bajo en una potencia $\frac{1}{n ^m}$ .

\begin{equation}
\begin{array}{c}
\zeta _{A} (s) = 
( \frac{L \mu }{\pi} ) ^{2s}
\sum _{n=1} ^{\infty}
  n  ^{-2 S} \\
(
	1 - 
	2 s \chi _n +  s(2s+1) \frac{\chi _n ^2}{2} - 
	\frac{2}{3} s(2s+1)(s+1) \chi _n ^3  + O( \frac{1}{n ^4}) )

\end{array}
\end{equation}

Calculadas todas las sumatorias, el resultado final es:





\begin{equation}
\begin{array}{c}
    \zeta _A (s) = \left( \frac{L \mu }{\pi} \right) ^{2s} \\
	\Bigg(
		\zeta ( 2 s ) -
		s \zeta ( 2s+1 ) +
		 \zeta (2s +2 ) s \left( \frac{1}{4} + \frac{s}{2} - \frac{2 L  \gamma}{\pi ^2} \right)  \\
		 - \zeta (2s+3) \left(  
							\frac{s(s+1) ( \pi ^2 + 2 \pi ^2 s - 24 L \gamma)}{12 \pi ^2 }
		 					\right) 
		+ ...
		\Bigg)
\end{array}
\end{equation}


Los polos de mi función $\zeta _A (s)$ estarán dados por los polos de las funciones $\zeta (s+n)$, desarrollando en los primeros polos obtengo:

\begin{equation}
\begin{array}{c}
\zeta _A (s \rightarrow 1/2) = 
\frac{L \mu }{2 \pi } \frac{1}{s-1/2} + \ finito \\
\zeta _A (s \rightarrow 0) = \ finito \\
\zeta _A (s \rightarrow -1/2) = \frac{\gamma}{2 \pi \mu } \frac{1}{s+1/2} \\
\zeta _A (s \rightarrow -1) = finito \\
\end{array}
\end{equation}

Lo cual está de acuerdo con lo expuesto en el capítulo 1.

\subsection{Calculo de la función zeta mediante calculo complejo:}

Conocida la ecuación de autovalores, se puede proceder a calcular la función $\zeta _A (s) $ sin calcular explícitamente los autovalores.

Si los  autovalores están definidos por una función $f(\lambda ) = 0$ de la cual son ceros simples, entonces la función $f'(z) / f(z) $ va a tener polos simples en los autovalores de $\hat{A}$, así la  función $\zeta _A (s)$ va a poder representarse como una integral en el plano complejo, donde el camino de intregración es cualquiera de los dos caminos representados en la figura [\ref{fig:contorno}]:

\begin{equation}
\begin{array}{c}
   \zeta _A (s) =  \sum _{n=1} ^{\infty} \left( \frac{\lambda _n}{\mu} \right) ^{-2s} =  \\
   \frac{1}{2 \pi i} \int _{C} \frac{f'(z)}{f(z)} \left( \frac{z}{\mu} \right) ^{-2s} dz =  
	\frac{1}{2 \pi i} \int _{C} \partial _z Ln f(z) \ 
	\left( \frac{z}{\mu} \right) ^{-2s}

\end{array}
\label{asd}
\end{equation}

Reemplazando $f(z)$ por  \ref{autovalores} se obtiene:

\begin{equation}
    \frac{1}{2 \pi i} \int _{C}
    \frac{ cos(L z) \left(L + \frac{1}{\gamma} \right) - sin(L z) \frac{z L}{\gamma}
    }
    {cos(L z) \frac{z}{\gamma} + sin(L z)
    }
    \left( \frac{z}{\mu} \right) ^{-2 s} dz
\end{equation}


\begin{figure}
\centering
\includegraphics[scale=0.3]{contorno.jpg}
\caption{Camino tenido en cuenta para realizar la integral de contorno en el plano complejo}
\label{fig:contorno}
\end{figure}


Voy a utilizar el camino de la derecha de [\ref{fig:contorno}], el cual puedo descomponer en 3 integrales, una angular y dos rectas, la contribución angular es regular para todo s por lo tanto no aporta a la estructura de polos, en cuanto a las contribuciones sobre los ejes se va a parametrizar de la forma $z = \pm i  t$. \\ 


	Luego de tirar los términos exponencialmente decrecientes, que no contriuyen a los polos se obtiene:

\begin{equation}
	\zeta _A (s) = 
    \frac{Sin(\pi s)}{ \pi } 
    \int _1 ^{\infty} 
    \left( \frac{t}{\mu}  \right)^{-2s}
    \left(
    	L + 
	    \underbrace
    	{
		\frac{1}{\gamma + t}   
		} _{\chi} 
	\right)
    dt 
\label{contorno}
\end{equation}

Donde en el demonimador se puede sacar factor comun $t$ y utilizar la serie geometrica para obtener:

\begin{equation}
    \chi =   \sum _{m=0} ^{\infty} \frac{(-1) ^{m} \gamma ^{m} }{t ^{m+1}}
\label{eq:chi}
\end{equation}

Luego de integrar termino a termino, el resultado final es:

\begin{equation}
    \zeta _A (s) = 
    \frac{Sin(\pi s) \mu ^{2s }}{\pi } 
    \left(
    \frac{L}{2s-1} + 
    \sum _{m=0} ^{\infty}
    \frac{(-1) ^{m} \gamma ^{m} }{2s+m}
    \right)
\label{eq.zeta.com}
\end{equation}

De aquí se puede ver que la función $\zeta _A (s)$ tiene polos simples en $s=1/2$ y los semienteros negativos, la estructura de polos queda determinada por:

\begin{equation}
\begin{array}{c}

\zeta(s \rightarrow 1/2) = \frac{L \mu }{2 \pi} \frac{1}{s-1/2} + finito\\
\zeta (s \rightarrow -n - 1/2)  = \frac{ (-1) ^n \gamma ^{2n+1}  }{2 \pi \mu ^{2n + 1}} \frac{1}{s + n + 1/2} + finito

\end{array}
\end{equation}


El cual coincide con lo calculado anteriormente. Utilizando esta tecnica se puede sacar la estructura entera de polos de la función $\zeta _A (s) $, pero sin calcular la parte finita de estas expresiones, para calcular la parte finita hay que tener en cuenta la parte angular de (\ref{asd}), y los terminos exponenciales tirados para llegar a (\ref{contorno}).


\subsection{Uso del Heat-Kernel:}

En el capitulo anterior, se dio una expresión para el desarrollo del Heat-Kernel en el limite $t \rightarrow 0$, lo cual a través de la transformada de Mellin permite, conocer los polos de la función $\zeta _A (s)$, dada las condiciones de nuestro operador (variedad unidimensional, sin curvatura, sin potencial), los coeficientes del Heat-Kernel se van a simplificar considerablemente, quedando

\begin{equation}
\begin{array}{c}
C _0 (A) = \frac{1}{\sqrt{4 \pi}} \int _{0} ^{L} dx = \frac{L}{\sqrt{4 \pi}} \\
C _1 (A) = \frac{1}{4} \left( (-1) + (+1) \right) = 0 \\
C _2 (A) = \frac{12}{6} \frac{1}{\sqrt{4 \pi }} \left(  - \gamma \right) = - \frac{\gamma}{\sqrt{\pi}} \\
C _3 (A) = \frac{192}{384}  (- \gamma ) ^2 = \frac{\gamma ^2}{2} \\
C _4 (A) = \frac{480}{360} \frac{1}{\sqrt{4 \pi}} (- \gamma) ^3 = \frac{-2 \gamma}{3 \sqrt{\pi}}

\end{array}
\end{equation}

Insertando estos coeficientes en (REF) obtengo:

\begin{equation}
\begin{array}{c}
Res[ \zeta _A (s)] | _{s=1/2} = \frac{L}{2 \pi} \\
Res[ \zeta _A (s)] | _{s=0} = 0 \\
Res[ \zeta _A (s)] | _{s=-1/2} = \frac{\gamma}{2 \pi} \\
Res[ \zeta _A (s)] | _{s=-1} = \frac{\gamma}{2 \pi} = 0 \\
Res[ \zeta _A (s)] | _{s=-3/2} = - \frac{\gamma ^2}{2 \pi} = 0 \\


\end{array}
\end{equation}

Lo cual coincide con lo calculado por los dos métodos anteriores.

\chapter{Estudio del Problema Singular}
{\label{cap.singular}}

El presente capítulo contiene el resultado original de este Trabajo de Diploma, que consiste en el análisis de un operador diferencial con un coeficiente singular. Como hemos mencionado, el desarrollo asintótico (\ref{eq.heat.expansion})
es válido para operadores diferenciales con coeficientes suaves. En efecto,
mostraremos a continuación que la presencia de una singularidad puede
conducir a un desarrollo asintótico de la traza del heat-kernel que no con-
tiene exclusivamente potencias semi-enteras de $T$, contradiciendo el resultado \ref{eq.heat.expansion}. Veremos que
en el caso del operador singular que hemos analizado el desarrollo del heat-kernel contiene términos $\log T$.


\section{El Operador Singular}


En está sección se estudiará el siguiente operador diferencial definido sobre funciones $\phi (x)\in \mathbb{L} ^2 [0,L]$,
\begin{equation}
\begin{aligned}
    A \phi (x) &= - \partial ^2 _x  \phi(x) + \frac{\alpha}{x} \phi(x) \\[5pt]
    \phi(0) &= \phi(L) = 0 \, .
\end{aligned}
\label{operador}
\end{equation}
Por simplicidad, se analizaran condiciones Dirichlet en ambos extremos.  El parámetro $\alpha \in \mathbb{R _{+}}$ caracteriza la singularidad en el origen.
Los autovalores están dados por la ecuación 
\begin{equation}
\begin{aligned}
    A  \phi (x)  &=   \lambda ^2 \phi (x) \\[5pt]
    \lambda ^2 \ &\in \ \mathbb{R}  
    \, ,
\end{aligned}
\label{eq.aut.sin}
\end{equation}
las autofunciones en términos de dos soluciones linealmente independientes $\phi _1 (x), \phi _2 (x)$ pueden escribirse
\begin{align}
\label{eq.phi}
&
    \phi (x) =
\\
&
	    C _1
    	\underbrace{
				     \ e ^{-i \lambda x} \ x \ 
				     F _{1} ^{1} 
				     		\left(  
				     			1 - \frac{i \alpha}
				     			{2\lambda}
				     		,2,2 i \lambda x \right) 
				     } _ {\phi_1} + 
      C _2 
      \underbrace{ 
      			   \ e^{-i \lambda x } \ x \ 
      			   U 
      			   	\left( 
      			   		1- \frac{i \alpha}{2 \lambda}
      			   		,2,2 i \lambda x \right) } _{\phi_2} 
    \, ,
\nonumber
\end{align}
donde $F _1 ^1(a,b,z)$ y $ U(a,b,z)$ son las soluciones LI de la ecuación {\mbox{hypergeométrica} }
\begin{equation}
    z \ \partial ^2 _z \ \psi (a,b,z) + (b-z) \
    \partial _z \psi (a,b,z)
    -a \ \psi (a,b,z) = 0 \, .
\end{equation}
las cuales están dadas por \cite{Abramowitz:hyper}
\begin{align}
	U(a,b,z) = &\frac{1}{\Gamma (a)} 
	\int _0 ^{\infty} e ^{-zt}
	t ^{a-1}
	(1+t) ^{b-a-1}
	dt \\
	& {\rm si \ } Re (b) > Re(a) 
	\nonumber
	\\[5pt]
	F _1 ^1 (a,b,z) =& \sum _ {k=0} ^{\infty} 
	\frac{(a) _k}{(b) _k} 
	\frac{z ^k}{k!} 
	\, ,
\end{align}
donde $(.) _n$ es el símbolo de Pochhammer.
Si se cumple que  $a=-m,b \neq -n$ donde $ m,n \in \mathbb{N}$, $F _1 ^1 (a,b,z)$ es un polinomio en $z$ y $U(a,b,z)$ es linealmente dependiente con $F _1 ^1 (a,b,z)$, en tal caso se usa $z^{1-b} M(a+1,-b,2-b,z)$ como segunda solución LI.


Las constantes $C_1,C_2$ están sujetas al vínculo impuesto por las condiciones de contorno.
Para imponer las condiciones de contorno, primeramente desarrollamos $\phi (x)$ cerca de $x \simeq 0$ 
\begin{align}
\phi  ( x ) &=
C _1  x  + 
C _2 \ x 
\left( 
\frac{1}{  \alpha x  \Gamma ( - \frac{i \alpha}{2  \lambda}  )   }  +
\frac{\log (x) }{\Gamma ( - \frac{ i \alpha}{2 \lambda} ) } + \mathscr{C} \right) + O(x ^2)
	\nonumber
\\[10pt]
\mathscr{C} &= 
\frac{
-1 + 2 \gamma + \log ( 2  i \lambda ) + \psi (1 - \frac{i \alpha}{2 \lambda})
}
{\Gamma (\frac{i \alpha}{2 \lambda})}
\, ,
\label{eq.scat}
\end{align}
Un detalle a tener en cuenta es que si sucede que $\Gamma ( \frac{i \alpha}{2 \lambda}  ) \rightarrow \infty$ entonces $\phi (0) = 0$ indepenientemente de $C _1$ y $C _2$, el problema radica en que no puede simplemente hacerse el reemplazo $\lambda \rightarrow  \frac{i \alpha }{ n } $ con $n=1,2,3 \dots$ en \eqref{eq.phi}. 

Para tratar este caso se escribe el autovalor $\lambda ^2 = - \frac{\alpha ^2}{4 n ^2}$ con $n \in \mathbb{R _+}$ en la ecuación \eqref{eq.aut.sin} cuyas soluciones están dadas por
\begin{align}
& \nonumber 
	\phi (x)     =
\\ &        
	    C _1
    	\underbrace{
				     \ e ^{- \frac{x \alpha}{2n}} \ x \ 
				     F _{1} ^{1}
				     \left( 1+n,2, \frac{x \alpha}{n} \right) 
				     } _ {\phi_1} 				     
				     + 
      C _2 
      \underbrace{ 
      			   \ e^{- \frac{x \alpha}{2n}} \ x \ 
      			   U 
      			   \left( 1+n,2, \frac{x \alpha}{n} \right)
      			   } _{\phi_2} 
    \, ,
\label{eq.phi.2}
\end{align}
utilizando esta solución se obtiene para $\phi (x) $ cerca de $x \simeq 0$ 
\begin{align}
\nonumber
	\phi (x) 
&=
	C _1 x 
	\left(1 + \frac{(1+n) \alpha x}{2 n } \right)
	+ C _2 x
	\left(
	\frac{1}{\alpha \Gamma (n) x } + \frac{\log (x)}{\Gamma (n)} +\mathscr{C}
	\right)
	+ O(x)
\\
\mathscr{C} 
&=
\frac{\psi (1+n) + \log \left( \frac{\alpha}{n} \right) -1 + 2 \gamma}{\Gamma (n)}
\, ,
\end{align}
lo cual implica $C _2 = 0$. Con lo cual el problema de energías negativas queda determinado por los ceros de.
\begin{equation}
	F _1 ^1 
	\left(
		1+n,2, \frac{L \alpha}{n}
		\right) = 0
\, ,		
\end{equation}
con lo cual el autovalor queda determinado por $\lambda ^2 = - \frac{\alpha ^2}{4 n ^2}$ donde \mbox{$n \in \mathbb{N}$}.


Para los cálculos que nos interesan solo se utilizarán los autovalores $\lambda ^2 >0$. 

Del desarrollo \ref{eq.scat} se ve que la condición de contorno en $x=0$ fija $C _2 =0$. 
Imponiendo entonces la condición de contorno $\phi (L)=0$ se obtiene que los autovalores positivos $\lambda ^2$ están dados por las soluciones de
\begin{equation}
F _1 ^1 \left(1-\frac{i \alpha}{2 \lambda},2,2 i \lambda L \right)  = 0
	\, .
\label{eq.1}
\end{equation}
En la figura (\ref{fig:funcion}) se encuentra graficado
\mbox{$ | F _1 ^1 (1-\frac{i \alpha}{2 \lambda},2,2 i \lambda L) | ^2 $} para $\alpha=1, \ L=1$. Las intersecciones con el eje horizontal indican los autovalores positivos del espectro del operador singular.


\begin{figure}[h!]
\centering
\includegraphics[scale=0.7]{Funcion.pdf}
\caption{En esta imagen se pueden ver los primeros ceros de la función $| F _1 ^1 (1+\frac{ \alpha}{2 i \lambda},2,2 i \lambda L) | ^2$ para $\alpha=1$ y $L=1$, los cuales representan a los primeros autovalores del operador $A$.}
\label{fig:funcion}
\end{figure}


Para calcular $\zeta(s)$ se utilizará el desarrollo asintótico de la funcion $F _1 ^1 (a,b,z)$ a $z \rightarrow \infty$ \cite{Abramowitz:hyper},
\begin{equation}
\begin{aligned}
    F _1 ^1 (a,b,z) &= \Gamma (b) 
    \left(
    \frac{e^z z ^{a-b} }{\Gamma(a)}  S_1 (z) + \frac{(-z) ^{ -a}}{ \Gamma(b-a)} 
    S_2 (z)
    \right) \\[5pt]
    S _1 (z) &= \sum _{n=0} ^{\infty} \frac{(b-a) _n (1-a) _n}{n!} z ^{-n} \\[5pt]
    S _2 (z) &= \sum _{n=0} ^{\infty} \frac{(a) _n (1+a-b) _n}{n!} (-z) ^{-n}     
		\, ,
\end{aligned}
\label{eq.aprox}
\end{equation}
aquí $S_1$ y $S _2$ representan el desarrollo a todo orden.
Utilizando este desarrollo $F _1 ^1 \left(1+  \frac{  \alpha}{2 i \lambda} ,2 ,2 i \lambda L  \right)$ queda determinada por
\begin{align}
\label{eq.completa}
	F _1 ^1 \left(1+  \frac{  \alpha}{2 i \lambda} ,2 ,2 i \lambda L  \right) 
&	
	\sim
    \frac{i e ^{ \frac{\pi \alpha }{4 \lambda}  } }{2 \lambda L}
    \left(
    \frac{e ^{   \frac{ i \alpha}{2 \lambda}  \log (2 \lambda L) }}               {\Gamma(1+\frac{i \alpha}{2 \lambda})} S _2 ( \lambda )-
    \frac{e ^{-  \frac{i \alpha}{2 \lambda}  \log (2 \lambda L) } e ^{2 i \lambda L} }{\Gamma(1-\frac{i \alpha}{2 \lambda})} 
    S _1 ( \lambda )
    \right) 
\nonumber
\\[5pt]
&
    =  i  \frac{e ^{ \frac{\pi \alpha }{4 \lambda}  } }{2 \lambda L}     M (\lambda) 
    \, .
\end{align}
Donde $S _{1,2}$ están expresados en la sección \ref{sec.sig.polos}.
Dado que $M( \lambda)$ posee los mismos ceros que $F _1 ^1 \left(1+  \frac{  \alpha}{2 i \lambda} ,2 ,2 i \lambda L  \right)$ se puede utilizar tanto $M$ como $F$ para estudiar $\zeta (s)$.


En las siguientes dos secciones \ref{seq.2.asin} y \ref{seq.2.com} se utilizará $M ( \lambda )$ al orden mas bajo (lo que corresponde a tomar $S _1 (\lambda) = S _2 ( \lambda )= 1$) para estudiar el polo de $\zeta \left( - \frac{1}{2} \right)$. En la sección \ref{sec.sig.polos} se tendrán en cuenta $S _1 (\lambda)$ y $S _2 ( \lambda )$ en $M ( \lambda)$ para estudiar la estructura de polos en la region ${\rm Re }(s) \geq - \frac{1}{2}$.
Luego finalmente en la sección \ref{sec.regular} se utilizará el desarrollo completo \eqref{eq.completa} para estudiar la parte finita de $\zeta \left( - \frac{1}{2} \right)$.

\section{Cálculo Asintótico de los autovalores}\label{seq.2.asin}

En esta sección seguiremos el procedimiento descrito en \ref{seq.asin} para estudiar la estructura de polos de $\zeta (s)$: calcularemos el desarrollo asintótico de grandes autovalores, que luego será utilizado para determinar los residuos en los primeros polos $s= \frac{1}{2}$ y $s= - \frac{1}{2}$.



Es conveniente definir las variables adimensionales: $\rho _n = \lambda _nL $ y $\beta = \alpha L$. Luego en vez de utilizar $M (\rho _n)$ definida en (\ref{eq.completa}) se trabajará con la función $N (\rho _n)$ definida de la forma
\begin{align}
\label{eq.otro.mu}
\nonumber
N (\rho _n) &=
e ^{\frac{i \beta }{\rho _n} \log(2 \rho _n) }
\Gamma \left( 1 + \frac{ \beta}{2 i \rho _n} \right)
M (\rho _n) \\ 
&=  
e ^{\frac{i \beta }{\rho _n} \log(2 \rho _n) }
\frac{\Gamma \left(1 + \frac{ \beta}{2 i \rho _n} \right)}
	{\Gamma \left(1 - \frac{ \beta}{2 i \rho _n} \right)}
- e ^{2 i \rho _n}
\, .
\end{align}
En el limite de $\rho _n \rightarrow \infty$ se obtiene
\begin{equation}
    N(\rho _n  \rightarrow \infty) = 
	1 - e ^{2 i \rho _n}
		\, ,
\end{equation}
de aquí se puede ver que los ceros de $N ( \rho ) $ cumplen la condición
\begin{align}
\label{eq.mu2}
    &\rho _n = n \pi + \epsilon _n \\[5pt]
\label{eq.mu2.limite}
	&\lim \limits _{n \rightarrow{0}} \epsilon _n  = 0
		\, .
\end{align}
Utilizando las ecuaciones (\ref{eq.mu2}) y \eqref{eq.mu2.limite} en el desarrollo (\ref{eq.otro.mu}) se obtiene una ecuación para $\epsilon _n$
\begin{equation}
	e ^{ i \frac{\beta}{ \rho _n} \log (2 \rho _n)}     
    \frac{\Gamma(1 + \frac{ \beta}{2  i \rho _n} ) }
    {\Gamma(1 -  \frac{ \beta}{2  i \rho _n} )} =    
    e ^{2 i \epsilon _n }
    	\, .
\label{eq.a.desarrollar}
\end{equation}
Utilizando $ \lim \limits_{\rho _n \rightarrow \infty} \frac{\log (2 \rho _n)}{2 \rho _n } \rightarrow 0$ se puede representar (\ref{eq.a.desarrollar}) por su desarrollo en serie en los límites $ \rho _n \rightarrow \infty $ y $\epsilon _n \rightarrow 0$,
\begin{equation}
    \left(
    \sum _{p = 0} ^{\infty} \frac{ \left( i \frac{\beta}{ \rho _n } \log(2 \rho _n ) \right) ^p }{p!}
    \right)
    \left(
	\sum _{q = 0} ^{\infty} \frac{a _q}{\rho _n ^q}
	\right)
    =
    \left(
    \sum _{l = 0} ^{\infty} \frac{( 2 i \epsilon _n)^l}{l !}
    \right)
    	\, .
\end{equation}
A partir de esta ecuación puede verse que el orden dominante de $\epsilon _n$ está determinado por
\begin{equation}
\left( 1 + \frac{i \beta}{ \rho _n} \log ( 2 \rho _n) \right) 
\left(1 + \frac{i  \gamma \beta}{ \rho _n} \right)  =
(1 + 2 i \epsilon _n) \, ,
\end{equation}
resolviendo asintóticamente la ecuación anterior se obtiene $\epsilon _n$
\begin{equation}\label{anterior}
    \epsilon _n =  \frac{\beta }{2 n \pi} \log (2 n \pi) +
                \frac{\gamma \beta}{2 n \pi} +
                O\left(  n^{-2} \right)
                	\, .
\end{equation}
Obteniendo finalmente un desarrollo de  $\lambda _n$ a grandes autovalores
\begin{equation}
	\lambda _n L= 
	\rho _n = 
	n \pi
	+ \frac{\beta }{2 n \pi} \log (2 n \pi)
	+ \frac{\gamma \beta}{2 n \pi}
	+ O ( n^{-2} )
\, .
\end{equation}
Para calcular la $\zeta (s)$ se utiliza este desarrollo de $\rho _n$
\begin{equation*}
\begin{aligned}
    \zeta (s) &= \sum _{n=1} ^{\infty} \left( \frac{\lambda _n}{\mu} \right) ^{-2 s}  
    = ( \mu L) ^{2 s} \sum _{n=1} ^{\infty}  \rho _n  ^{-2 s}  \\
    & =    ( \mu L) ^{2 s} \sum _{n=1} ^{\infty} 
    \left( 
    n \pi + \frac{\alpha L }{2 n \pi} \log (2 n \pi) + \frac{\gamma \alpha L}{2 n \pi} +
    O \left( n^{-2} \right)
    \right) ^{-2s}
    	\, ,
\end{aligned}
\end{equation*}
resultado que puede reescribirse en términos de un parámetro $\chi _n$ pequeño tal como se hizo en la ecuación \eqref{eq.abajo.chi}
\begin{equation}
\begin{aligned}
    \zeta  (s) &= \left( \frac{\mu L }{\pi} \right)  ^{2 s} 
    \sum _{n=1} ^{\infty} n ^{- 2  s} 
    \left(
    	1 + \chi _n  + O( n^{-2} )
    	\right) ^{-2 s} \\[5pt]
		 \chi _n &= 
    	\frac{\alpha L  }{2 n^2 \pi ^2} \log (2 n \pi) + 
    	\frac{\gamma \alpha L}{2 n^2 \pi ^2 }  
    			\, .
\end{aligned}
\end{equation}
Desarrollando consistentemente hasta el orden retenido se obtiene
\begin{align}\label{eq.zeta.c}
    \zeta  (s) &= \left( \frac{\mu L}{\pi} \right) ^{2 s}
    \sum _{n=1} ^{\infty} 
    n ^{-2s}
    \left(
    1 - 2 s \chi _n + O \left( n ^{-3} \right) \
    \right)   \nonumber \\[5pt]
     &= \left( \frac{\mu L }{\pi} \right) ^{2 s}
    \sum _{n=1} ^{\infty} n ^{-2 s} 
    \left(
    1 - 2s \left(
    \frac{\alpha L }{2 n ^2 \pi ^2} \log ( 2  n \pi) + 
    \frac{\gamma \alpha L }{2 n ^2 \pi ^2} 
	\right) +
    O \left( n ^{-3}   \right)
    \right) \nonumber \\[5pt]
    &=   \left( \frac{\mu L }{ \pi } \right) ^{2 s}  
    \left( \zeta _R (2 s) -
	\frac{ s \alpha L}{ \pi ^2} \zeta _R (2s+2)
	\left(
	    \log (2  \pi ) + \gamma
	\right) + 
    \frac{s \alpha L}{\pi ^2}
	\zeta ' _R(2s+2) \right) \nonumber \\[5pt]
	& + \sum _{n=1} ^{\infty} O \left( n ^{-2s-3} \right) \, .
\end{align}    
Dado que $\sum _{n=1} ^{\infty} O \left( n ^{-2s-3} \right)$ es regular en la región  ${\rm Re }(s) > -1$, de la expresión anterior se ve que el polo en $s=  \frac{1}{2}$ está determinado por
\begin{equation}\label{eq.res.2}
    ( s-1/2 ) \zeta  (s) | _{s \rightarrow \frac{1}{2}} = 
    \frac{\mu L }{2 \pi}
    	\, ,
\end{equation}
lo cual coincide con el resultado general (\ref{eq.vol}). Luego desarrollando (\ref{eq.zeta.c}) alrededor de $s=-\frac{1}{2}$ se obtiene
\begin{align}\label{eq.res.1}
    &\zeta  (s) =  \frac{\alpha}{8  \pi \mu \left(s+\frac{1}{2} \right)^2} +
    \frac{ \alpha ( \gamma  +  \log (2\mu L ) -1 ) }
    	{4  \pi \mu \left(s+\frac{1}{2} \right) }  + 
	{\rm PF }\zeta \left(- \frac{1}{2} \right)
    	\, .
\end{align}
Donde PF $\zeta \left(- \frac{1}{2} \right)$ significa la parte finita la cual será calculada en la sección \ref{sec.parte.finita.vacio}.

Aquí se encuentra el primer resultado de esta tesis, la existencia de un polo doble en $\zeta \left( -\frac{1}{2} \right)$ lo cual está en contradicción con el resultado (\ref{eq.ceros.zeta}) el cual determina que $\zeta (s)$ posee solamente polos simples.

\section{Cálculo Utilizando Variable Compleja}\label{seq.2.com}


En la sección anterior se estudiaron los polos de la función $\zeta$ en $s=\frac{1}{2}$ y $s=-\frac{1}{2}$ desarrollando los autovalores.
En esta sección se utilizará el mismo procedimiento que en el capítulo \ref{sec.complejo}: se va a expresar la función-$\zeta $ como una integral en el plano complejo.

Dado que se van a estudiar los polos en $s= \frac{1}{2}$ y $s=- \frac{1}{2}$, al igual que en la sección anterior, se utilizará el orden dominante de la función $M ( \lambda )$ definida en \eqref{eq.completa}. Así
$ \zeta (s)$ queda determinada por
\begin{equation}
\zeta (s) = 
\frac{1}{2 \pi i} 
\int _{\mathcal{C}}
\partial _z \ \log 	M(z)  \left( \frac{z}{\mu} \right) ^{-2s} \ dz
	\, ,
	\tag{\ref{asd}}
\end{equation}
Para realizar esta integral se va a utilizar el contorno dado en la figura \ref{fig.medio}. Se utilizará la parametrización $ z (t) = \pm i t$ sobre los ejes verticales, con lo cual el termino $e ^{2 i \lambda L}$ va a generar un termino exponencialmente creciente/decreciente dependiendo de la rama. Reteniendo solamente los términos que aportan a los polos se obtiene
\begin{comment}
\begin{equation}
\begin{array}{c}
    \zeta  (s) = \\
     \frac{1}{2 \pi i} \int _{\infty} ^{1}
     \frac{ i \alpha }{2 t^2} 
     \left(
      1 + \frac{i \pi}{2} + Ln[2 t] + \psi (1 + \frac{\beta}{2 t})
     \right)
     t ^{-2s}
     e ^{- i \pi s} (i dt) + \\
     \frac{1}{2 \pi i} \int _{\infty} ^{1} 
     \left(
     2 + \frac{\beta}{2 t^2}
     \left(
     1 + \frac{i \pi}{2} - Ln[2 t] - \psi (1+ \frac{\beta}{2 t})
     \right)
     t ^{-2s}
     e ^{ i \pi s} (-i dt)
     \right)     
\end{array}
\end{equation}
\end{comment}
\begin{align}\label{eq.logatirmos}
\log ( M ( \lambda = i t ) ) &=   
\frac{i \alpha }{2 \lambda}  \log (2 \lambda L) - 
 \log \left( \Gamma \left( 1 - \frac{ \alpha}{2 i \lambda} \right) \right) \\ 
\log ( M ( \lambda=-i t ) ) &=   -  
\frac{i \alpha }{2 \lambda}  \log ( 2 \lambda L ) - 
 \log \left( \Gamma \left( 1 + \frac{ \alpha}{2 i \lambda} \right) \right) +
2 i \lambda L  \nonumber
	\,	,
\end{align}
con lo cual $\zeta (s)$ queda representada de la forma
\begin{align}\label{eq.zeta.logs}
     & \zeta  (s) = \\
     & \frac{e^{-i \pi s} \mu ^{2s}}{2 \pi i} \int _{\infty} ^{1}
     \frac{ i \alpha}{2 t^2}
     \left(
     - 1 +  \log (2 L t) + \frac{i \pi}{2}  - \psi \left( 1+\frac{\alpha}{2 t} \right)
     \right)
     t^{-2 s}
      \nonumber
     (i dt) + \\
     & \frac{e^{i \pi s} \mu ^{2s}}{2 \pi i} \int _1 ^{\infty}
	\left(      
     \frac{ i \alpha}{2  t^2}
     \left(
     1 -  \log (2 L t) + \frac{i \pi}{2} + \psi \left( 1 + \frac{\alpha}{2 t} \right)       
     \right)
     + 2 i L
     \right)
     t^{-2 s}
     (-i dt) \nonumber
     	\, .
\end{align}
Solo hay un término que aportará al polo en $s= \frac{1}{2}$, el término con la potencia mas alta de $t$
\begin{equation}
    \frac{e^{i \pi s} \mu ^{2s} }{2 \pi i }
    \int _1 ^{\infty}
    2 i L    
    t ^{-2 s}
    (-i dt) =  
    \frac{L e^{i \pi s} \mu ^{2s}}{2 \pi i} \frac{1}{s-1/2   }
    	\, ,
\end{equation}
desarrollando el numerador al orden lineal se obtiene para el polo en $s= \frac{1}{2}$ 
\begin{equation}
	\left(s- \frac{1}{2} \right)
    \zeta (s) \Big| _{s=1/2} = \frac{\mu L }{2 \pi} 
    	\, ,
\end{equation}
lo cual coincide con el resultado del capítulo anterior \eqref{eq.res.2}.
De esta forma una vez calculado este término, el resto de la integral \eqref{eq.zeta.logs} puede reescribirse de la forma
\begin{align}
    & \frac{\alpha \mu ^{2s} }{2 \pi} \sin(\pi s)
    \int _1 ^{\infty}
    t ^{-2 s-2} 
    \left(
    1 -  \log (2Lt) + \psi \left( 1 + \frac{\alpha}{2t} \right)
    \right) dt \ + 
    	\nonumber \\[5pt]
    &
    \frac{\alpha \mu ^{2s} }{4} 
    \cos (\pi s)
    \int _1 ^{\infty} t^{-2s-2} dt
    	\, ,
\end{align}
donde todos los términos son calculables analíticamente, excepto el que contiene $\psi \left( 1 + \frac{\alpha}{2t} \right)$, para lo cual se puede utilizar el desarrollo
\begin{equation}
    \psi \left(1 + \frac{\alpha}{2 t} \right) =
    - \gamma + O \left( t^{-1} \right)
    \, .
\end{equation}
Donde $\gamma \approx 0.57$ es la Constante de Euler-Mascheroni.  
Una vez realizadas todas las integrales $\zeta (s)$ queda expresada como  

\begin{align}
\nonumber
    \zeta  (s)  & = 
     \frac{\mu ^{2s} L ^{2 s} e ^{i \pi s}}{2 \pi i (s-\frac{1}{2})}  
    -\frac{\alpha \mu ^{2s} \sin(\pi s)}{8 \pi \left( s+\frac{1}{2} \right) ^2}  + 
    \frac{\alpha \mu ^{2s} \sin  (\pi s) (1 - \gamma -   \log (2 L))}{4 \pi (s+ \frac{1}{2} )} 
\\
\label{eq.laurent}
    & + \frac{\alpha \mu ^{2s} \cos(\pi s)}{8 (s+\frac{1}{2} )} +
    \int _1 ^{\infty} O \left( t ^{-2s-3} \right) dt
    	\, .
\end{align}
Para calcular el polo en $s=- \frac{1}{2}$ se desarrolla en Serie de Laurent la expresión anterior
\begin{align}
\label{eq.result.zeta.c}
    &\zeta  (s) =  \frac{\alpha}{8  \pi \mu (s+1/2)^2} +
    \frac{ \alpha ( \gamma  +  \log (2\mu L ) -1 ) }{4  \pi \mu (s+1/2) }  + 
	{\rm PF } \zeta \left( - \frac{1}{2} \right)
    	\, ,
\end{align}
obteniendo un polo doble, lo cual coincide con el resultado \eqref{eq.res.1} de la sección anterior.
La parte finita de $ \zeta \left( - \frac{1}{2} \right)$ será calculada en la sección \ref{sec.parte.finita.vacio}

\section{Siguientes Polos}\label{sec.sig.polos}


En las dos secciones anteriores \ref{seq.2.asin} y \ref{seq.2.com}  se utilizó el primer orden en el desarrollo de la función $ M (\lambda )$ definida en (\ref{eq.completa}) para obtener los polos en $s= \frac{1}{2}$ y $s=- \frac{1}{2}$. En esta sección tambien se va a utilizar la función $M ( \lambda )$, pero teniendo en cuenta $S _1 ( \lambda ) $ y $S _2 ( \lambda )$ para poder calcular la estructura completa de polos, demostrando que todos son polos simples ubicados en los semienteros negativos. 


Se va a calcular la función-$\zeta$ utilizando variable compleja, como en la sección anterior, pero tomando todos los términos en el desarrollo de $M ( \lambda )$,
\begin{align}
\label{larga}
M( \lambda ) &= 
-
 \frac{e ^{2 i \lambda L } e ^{ - \frac{i \alpha  }{2 \lambda } Ln \left( 2 \lambda L \right) }  }
      { \Gamma \left( 1 - \frac{i \alpha}{2  \lambda}  \right) } S _1 ( \lambda ) +
 \frac{ e ^{   \frac{i \alpha  } {2 \lambda } Ln \left(2 \lambda L \right) } }
      { \Gamma \left( 1 + \frac{i \alpha}{2  \lambda}  \right)   } S _2 ( \lambda )        
\\[10pt]      
	S _1 ( \lambda ) &= \sum _{n=0} ^{ \infty }
\left(1 - \frac{ \alpha}{2 i \lambda}  \right) _n
\left(- \frac{ \alpha}{2 i \lambda}  \right) _n
\frac{1}{( 2 i \lambda L ) ^n \ n!} = 
	1 + \sum _{n=1} ^{\infty} \frac{S ^{(1)} _n (\lambda)}{\lambda ^n} 
\nonumber
\\[10pt]
	S _2 (\lambda ) &= \sum _{n=0 } ^{\infty}
\left( 1 + \frac{ \alpha}{2 i \lambda }  \right) _n
\left( \frac{ \alpha }{2 i \lambda} \right) _n
\frac{1}{( - 2 i \lambda L ) ^n \ n!} = 
1 + \sum _{n=1} ^{\infty} \frac{S ^{(2)} _n (\lambda)}{\lambda ^n}
\nonumber
\, .
\end{align}
Donde $S _n ^{(1,2)}$ son un polinomios en $\frac{1}{ \lambda}$ donde la potencia mas alta es $3 n$ y la mas baja $n$, lo cual permite controlar el orden del desarrollo.


Al igual que en la sección \ref{seq.2.com} al integrar sobre los ejes verticales va a existir un término exponencialmente creciente/decreciente, teniendo en cuenta solo los términos que contribuyen a los polos se obtiene
\begin{align}
	\log ( M ( \lambda = i t ) ) 
&
	=   \log (S _2) + 
	\frac{i \alpha }{2 \lambda}  \log (2 \lambda L) - 
 	\log \left( \Gamma \left( 1 - \frac{ \alpha}{2 i \lambda} \right) \right) 
\\ 
	\log ( M ( \lambda=-i t ) ) 
&
	=  \log (S _1) -  
	\frac{i \alpha }{2 \lambda}  \log ( 2 \lambda L ) - 
	\log \left( \Gamma \left( 1 + \frac{ \alpha}{2 i \lambda} \right) \right) +
	2 i \lambda L  \nonumber
	\,	,
\end{align}
donde la única diferencia con los logaritmos calculados anteriormente  en la ecuación (\ref{eq.logatirmos}) es la aparición de los términos $\log ( S _1 )$ y $ \log ( S _2) $.

Teniendo en cuenta solo los términos que contribuyen a los polos en  $R{\rm } e(s) < - \frac{1}{2}$ se obtiene para $\zeta (s)$
\begin{align}
\nonumber
	\zeta  (s) =& 	
\\[10pt]
& 
	\frac{e ^{- i \pi s} \mu ^{2s } }{2 \pi}
	\int _{\infty} ^{1} t ^{-2s } 
		\frac{S _2' (it)}{S _2 (it)}
		d t - 
	\frac{e ^{i \pi s} \mu ^{2s}}{2 \pi}
	\int _{1} ^{\infty} t ^{-2s } 
	\frac{S _1 ' (-it)}{S _1 (-it)}
	d t
\nonumber 
	 \\[10pt]
	&  + \frac{\alpha \mu ^{2s} }{2 \pi }	\sin ( \pi s)  \int _1 ^{\infty}
	t ^{-2s-2}  \psi \left( 1 + \frac{\alpha}{2 t}\right) dt 
		\, .
\end{align}
Utilizando la propiedad  $\frac{S _1 ' (-it)}{S _1 (-i t)} = - \frac{S _2 ' (i t)}{S _2 (it)} = \frac{S'(t)}{S(t)}  $ se puede reescribir la ecuación anterior de la forma
\begin{align}
\zeta  (s) &= 
\frac{\alpha \mu ^{2s} sin( \pi s )}{2 \pi } \int _{1} ^{\infty} 
	  \psi \left( 1 + \frac{\alpha}{2 t} \right) 
	   t ^{-2s-2} dt
\\[5pt]
\nonumber
	& -  \frac{i \mu ^{2s}  sin (\pi s)}{\pi} \int _1 ^{\infty} t 			^{-2s} \frac{S'(t)}{S(t)} dt 
	\, ,
\end{align}
de aquí se puede observar que solo van a existir polos simples en los semienteros negativos dado que $\sin (\pi s)$ posee ceros simples en los enteros negativos, lo cual coincide con el caso regular y cuyas implicancias se detallarán las concluciones.


Para controlar el orden del desarrollo de $ \frac{S' ( t)}{ S ( t)} $ el primer paso es desarrollar $\log S _1 (\lambda)$ alrededor de $\lambda \rightarrow \infty$, luego tomarle su derivada respecto de $\lambda$ y finalmente evaluar en $\lambda = -i t$
\begin{align}
&
\nonumber
	\frac{S _1 '( \lambda)}{S _1 ( \lambda )} 
	\simeq 
	\partial _{\lambda} Log \left(
						1 + \sum _{n=1} ^{N-1}  \frac{S ^{(1)} _n}{\lambda ^n}
						\right) =
\partial _{\lambda} 
\sum _{m = 1} ^{N/2} 
	\left(
	\frac{(-1) ^{m+1} }{m}
	\left(
		\sum _{n=1} ^{N-1} \frac{S ^{(1)} _n}{\lambda ^n}
		\right) ^m 
	\right)  \\[10pt]
	&=
\left(								
	\sum _{l = 1} ^{N-1} 
	\frac{S  ^{(1) '} _l}{\lambda ^l} - l \frac{S ^{(1)}  _l}{\lambda ^{l+1}}
	\right)							
\left(
	\sum _{m = 1} ^{N/2} (-1) ^{m+1} 
	\left(
			\sum _{n=1} ^{N-1} \frac{S ^{(1)} _n}{\lambda ^n}
			\right) ^{m-1}		
	\right)
\, .
\end{align}	
Este desarrollo  es correcto hasta el orden N, en caso de que N sea semi-entero hay que tomarle la parte entera. Los primeros términos de $\frac{S'(t)}{S(t)}$ son
\begin{equation}
\frac{S'(t)}{S(t)} = 
\frac{i \alpha}{2 L t^3} -
\frac{3 i (L \alpha ^2 - 2 \alpha)}{8 L^2 t ^4}
+ O (t ^{-5}) 
\, .
\end{equation}
Con este desarrollo el polo en $s=-\frac{3}{2}$ está dado por 
\begin{equation}
\left( s+ \frac{3}{2} \right)
\zeta  (s) \Big| _{s = - \frac{3}{2}}= 
\frac{L ^2 \alpha  ^3 \psi ^{(2)} (1) + 12   \alpha  - 6 L \alpha ^2}{32 L^2 \pi \mu ^3}
\, .
\end{equation}
Para seguir calculando la estructura de polos, basta con seguir desarrollando las funciones $S(t)$ y $\psi (1 + \frac{\alpha}{2 t})$.


%Estos comentarios tienen las contribuciones de la parte finita
\begin{comment}
Las primeras 3 integrales se pueden realizar analíticamente de manera sencilla, dado que son todas series de potencias, la integral angular va a tener que ser evaluada numéricamente dado que es de la forma (parametrizando $\lambda = e ^{i \theta}$ y llamando $M_c$ a todo el termino adentro del Logaritmo) :

\begin{equation}
\begin{array}{c}

\frac{1}{2 \pi i} \int _{\pi /2 } ^{- \pi /2} 
\frac{e ^{-2 s i \theta} d \theta}{M [e ^{i \theta}]} \\

\Bigg[

\frac{
e ^{- \frac{i \alpha (Ln[2 L] + i \theta)}{2 e ^{i \theta} }} e ^{2 i L e ^{i \theta}}
}{\Gamma \left( 1 - \frac{i \alpha}{2 e ^{i \theta}} \right)}
	\left(
		\left(
			2 i L -
			\frac{i \alpha}{2 e ^{2 i \theta} } + 
			\frac{i \alpha( Ln[2 L ] + e ^{i \theta} ) }{2 e^{2 i \theta}}
			- \frac{i \alpha \psi \left( 1 - \frac{i \alpha}{2 e ^{i \theta}}\right)}
				   {2 e ^{2 i \theta}}
			\right) S1 [e ^{i \theta}] +
		S1 ' [e ^{i \theta }]
		\right)
  \\

- \frac{
e ^{ \frac{i \alpha (Ln[2 L] + i \theta)}{2 e ^{i \theta} }}
}{\Gamma \left( 1 + \frac{i \alpha}{2 e ^{i \theta}} \right)}
	\left(
		\left(
			\frac{i \alpha}{2 e ^{2 i \theta} } - 
			\frac{i \alpha( Ln[2 L ] + e ^{i \theta} ) }{2 e^{2 i \theta}}
			+ \frac{i \alpha \psi \left( 1 + \frac{i \alpha}{2 e ^{i \theta}}\right)}
				   {2 e ^{2 i \theta}}
			\right) S2 [e ^{i \theta}] +
		S'2 [e ^{i \theta }]
		\right)


\Bigg]

\end{array}
\end{equation}

La cual va a dar una constante independiente de $\lambda$ y se puede calcular numéricamente.


A la hora de calcular los términos de la forma $S'/S$ hay que tener en cuenta hasta que orden hay que llevar el numerado y el denominador para poder ser consistente con la expansión en serie.

\begin{equation}
\begin{array}{c}
\frac{S'(x)}{S(x)} =
\frac{
		- \sum _{n=1} ^{\infty} \frac{n a_n}{x ^{n+1}}
      }
      {
		1 + \sum _{m=1} ^{\infty} \frac{a _n}{x ^{n}}
            } =
            

\left(
	    - \sum _{n=1} ^{\infty} \frac{n a_n}{x ^{n+1}}
		\right)
\sum _{p =0} ^{\infty}
		\left(
			    \sum _{m=1} ^{\infty} \frac{a _n}{x ^{n}}
	    		\right) ^{p}
\end{array}
\end{equation}

Donde se puede hacer el producto de Cauchy para tener la solución exacta de hasta que términos hay que desarrollar $S,S'$ y la Serie Geométrica.

\begin{equation}
\frac{1 }{2 \pi i}
\int _{circulo} \lambda ^{-2s } \partial \lambda \ Ln \left[
					\frac{e ^{\frac{i \alpha  \log ( 2 \lambda L )}{2 \lambda}} e ^{2 i \lambda L} S1}
					{\Gamma \left( 1 - \frac{i \alpha}{2 \lambda} \right)} - 
					\frac{e ^{\frac{-i \alpha  \log (2 \lambda L )}{2 \lambda}} S2}
					{\Gamma \left( 1 + \frac{i \alpha}{2 \lambda} \right)}					
					\right] d \lambda
\end{equation}
\end{comment}

\section{La energía de vacío}\label{sec.parte.finita.vacio}

La energía de vacío al igual que en el capítulo anterior queda definida por la ecuación (\ref{eq.casimir.mu})
\begin{equation}
\nonumber
    E _0 = \frac{\mu }{2}  
    \zeta  \left( - \frac{1}{2} \right) 
	\, ,
\end{equation}
Utilizando la expresión de $\zeta  (s )$ obtenida en las secciones \ref{seq.2.com} y \ref{seq.2.asin} se obtiene para la energía de vacío
\begin{equation}\label{eq.casimir.resultado}
E _0 ( \epsilon ) = \frac{1}{2} \left(
				\frac{\alpha}{8 \pi  \epsilon ^ 2}  + 
				\frac{\alpha ( \gamma  +  \log (2\mu L ) -1 )}{4 \pi  \epsilon}
				\right) + 
				\frac{\mu}{2} {\rm PF } \zeta \left( - \frac{1}{2} \right)
\, .
\end{equation}
Donde se puede apreciar que la energía de vacío depende del cutoff $\epsilon$. El término ${\rm PF } \zeta \left( - \frac{1}{2} \right)$ se calculará en la sección siguiente, conduciendo a la forma completa de la energía de vacío dada por las expresiones  (\ref{energia.vacio.final}) y (\ref{eq.vacio.completa.finita})


\section{Parte finita de la energía de vacío}
\label{sec.regular}

En los capítulos \ref{seq.2.asin} y \ref{seq.2.com} se calcularon los polos de \mbox{$\zeta (s)$} en $s= \frac{1}{2}$ y $s=-\frac{1}{2}$ encontrando que en $s=- \frac{1}{2}$ existe en polo doble. En \ref{sec.sig.polos} se estudió la estructura completa de polos encontrando que $\zeta (s)$ posee polos son simples ubicados en los semienteros negativos. En \ref{sec.regular} se expresó la energía de vacío siendo la parte finita de esta $\frac{\mu}{2} {\rm PF } \zeta \left( - \frac{1}{2} \right)$.

En esta sección se estudiará ${\rm PF } \zeta \left( - \frac{1}{2} \right)$ de dos maneras diferentes: primero se utilizará el desarrollo asintótico correspondiente a la sección \ref{seq.2.asin} y luego se hará uso de la representación integral tal como se hizo en la seccion \ref{seq.2.com}.

\subsection{Desarrollo asintótico}\label{seq.desarrollo.asintotico}

Utilizando las variables adimensionales $\rho _n = \lambda _n L$ y $\beta = \alpha L$, $\zeta (s)$ puede expresarse como
\begin{equation*}
\zeta (s) = \left( \mu L \right) ^{2s} \sum _{n=1} ^{\infty} \rho _n ^{-2s} 
\, ,
\end{equation*}
como se demostró que $\zeta (s)$ posee un único polo doble en $s = -\frac{1}{2}$, $\rho _n$ posee un desarrollo de la forma
\begin{equation}
\begin{aligned}
\rho _n  &= 
			n \pi + \epsilon _n \\
			\epsilon _n &= 
			\frac{ \beta }{2 n \pi } \log (2 n \pi) +
			\sum _{p=1} ^{\infty} \frac{a _p}{n ^p }
			\, .
\end{aligned}
\end{equation}
Procediendo igual que en el capítulo \ref{seq.2.asin}, $\zeta (s)$ puede expresarse como un desarrollo en serie
\begin{align}
\zeta (s) =& 
( \mu L ) ^{2s}
\sum _{n=1} ^{\infty}
( n \pi) ^{-2s} \left( 1 + \frac{ \epsilon _n }{n \pi } \right) ^{-2s } \\
\nonumber
 =& 
(\mu L) ^{2s} \sum _{n=1} ^{\infty}
( n \pi) ^{-2s} \left(
						1 -2s  \frac{\epsilon _n}{n \pi} + 
						\left( s + \frac{1}{2} \right) 
						O \left( \frac{ \log ^2 ( 2 n \pi ) }{ n ^4} \right)  
						\right)
\, .
\end{align}
Donde el último término no presenta polos en \mbox{${\rm Re}(s) >- \frac{3}{2}$} dado que
\begin{equation} 
	\sum _{n=1} ^{\infty}
	 n  ^{-2s} O \left( \frac{ \log ^m (n)}{ n ^p} \right)
	\propto 
		(-1) ^m \zeta _R ^{(m)} (2s+p) 
\, ,
\end{equation}
lo cual es finito para $s \neq \frac{1-p}{2}$.
Por lo tanto todos los términos siguientes en el desarrollo también se anularán debido a la existencia del factor $s + \frac{1}{2}$, lo que implica que las términos que contribuyen a $\zeta \left( - \frac{1}{2} \right)$ están dados por
\begin{align}
\zeta (s) &= \left( \frac{\mu L}{\pi} \right) ^{2s} \times \\
			\nonumber
			&\times
			\left(
					\zeta _R (2s) - \frac{s \beta}{\pi ^2} 
						\left(
							\log (2 \pi ) \zeta _R (2s+2) - \zeta _R '(2s+2)
							\right)-
					\frac{2 s}{\pi} \sum _{p=1} ^{\infty}
						a _p \zeta _R (2s+p+1)
					\right)
\, .					
\end{align}
Esta expresión completa la ecuación (\ref{eq.zeta.c}) en  la cual no se tiene en cuenta el último término.
Para calcular $\zeta \left( - \frac{1}{2} \right)$ se utilizan los desarrollos
\begin{align}\label{cortar}
	\log (2 \pi) \zeta _R (2s+2) -
	\zeta _R ' (2s+2) = & 
	\frac{1}{4 \left( s + \frac{1}{2} \right) ^2} + 
	\frac{ \log (2 \pi ) }{2 \left( s + \frac{1}{2} \right) } 
\\ \nonumber
	+ &  \gamma \log (2 \pi ) + \gamma _1 + O \left( s + \frac{1}{2} \right) \\
	\zeta _R (2s+2) = &\frac{1}{2 \left( s + \frac{1}{2} \right)} + \gamma + O \left( s + \frac{1}{2} \right)
	 ,
\end{align}
donde $\gamma _1$ es la constante de Stieltjes. Obteniendo para $\zeta \left( - \frac{1}{2} \right) $
\begin{align}\label{eq.zeta.final}
\zeta \left( - \frac{1}{2} + \epsilon \right) &=
		\frac{\beta}{8  \pi \mu L \epsilon ^2}	 +
	    \frac{
	    	\beta ( \gamma  +  \log (2 \mu L ) -1 ) }
	    	{4  \pi \mu L \epsilon } 
\\[5pt]
\nonumber
&
+
		\frac{\beta \log ^2 \left( \frac{\mu L}{\pi} \right)}{4 \pi \mu L}  +
		\frac{
			\beta \log \left( \frac{\mu L}{\pi}\right)
				( \log (2 \pi ) + \gamma -1)}
			{2 \pi \mu L}  
- \frac{\beta (\gamma + \log(2 \pi) )}{2 \pi \mu L}
\\[5pt]
\nonumber
&
+
\frac{\pi}{\mu L}  
					\left(
							- \frac{1}{12} +
							\frac{\beta}{2 \pi ^2} \left(
														\gamma \log (2 \pi)
														+ \gamma _1
														\right) +
								a _1 \frac{\gamma}{\pi} +
								\frac{1}{\pi} \sum _{p=2} ^{\infty}
								a_p \zeta (p) 
							\right) 
\, .
\end{align}
De aquí se puede ver que los polos coinciden con los calculados en las secciones \ref{seq.2.asin} y \ref{seq.2.com}, así ${\rm PF} \zeta (s)$ definida en (\ref{eq.casimir.resultado}) y \eqref{eq.res.1} queda determinado por el segundo y tercer renglon de la expresión anterior. 
Utilizando esta definición de $\zeta \left( - \frac{1}{2} \right) $ se obtiene para la energía de vacío
\begin{align}\label{energia.vacio.final}
	E_ 0 (\epsilon )&=
		\frac{\alpha}{16  \pi  \epsilon ^2}	 
		+   \frac{
	    	\alpha ( \gamma  +  \log (2\mu L ) -1 ) }
	    	{8  \pi \epsilon } 
\\[5pt]
\nonumber
&+
	\frac{\alpha \log ^2 \left( \frac{\mu L}{\pi} \right)}{8 \pi}  +
		\frac{ 
			\alpha \log \left( \frac{\mu L}{\pi}\right)
				( \log (2 \pi ) + \gamma -1)}
			{4 \pi }  
	- \frac{\alpha (\gamma + \log (2 \pi ) )}{4 \pi}
\\[5pt]
\nonumber
&
+
	\frac{\pi}{2 L}  
			\left(
				- \frac{1}{12} +
				\frac{\alpha L}{2 \pi ^2} 
				\left(
					\gamma \log (2 \pi)
					+ \gamma _1
					\right) +
								\frac{\gamma ^2 \alpha L}{2 \pi ^2} +
								\frac{1}{\pi} \sum _{p=2} ^{\infty}
								a_p \zeta (p) 
							\right) 
\, .
\end{align}
Dado que $a _p$ es un polinomio en $\alpha L$, en el límite $\alpha \rightarrow 0$ se obtiene
\begin{equation}
\lim \limits_{\alpha \rightarrow 0} E _0 = 
		- \frac{\pi}{24 L}
\, ,
\end{equation}
lo cual corresponde a la energía de vacío sin potencial con condiciones de contorno Dirichlet en ambos extremos, tal como se calculó en la ecuación (\ref{eq.energia.dirichlet}) del capítulo \ref{sec.Dirichlet}.

En el apéndice \ref{Apendice.1} se encuentra un script del software Mathematica que permite obtener los términos $a _p$, En este caso se calcularon hasta $p=10$. En la figura \ref{fig.finitas} está graficada la energía de vacío adimensionalizada obtenida por este método junto con el método que se describirá a continuación.



\begin{figure*}[t!]
    \centering
    \begin{subfigure}[t]{0.5\textwidth}
        \centering
        \includegraphics[height=1.3in]{exportar1.pdf}
        \caption{}
        \label{fig.derecha}
    \end{subfigure}%
    ~ 
    \begin{subfigure}[t]{0.5\textwidth}
        \centering
        \includegraphics[height=1.3in]{Finita.pdf}
        \caption{}
        \label{fig.izquierda}
    \end{subfigure}
    ~
    \caption{En esta imagen se muestran dos posibles adimensionalizaciones de la energía de vacío, una vez obtenida la curva \ref{fig.izquierda} se pueden generar todas las curvas de la figura \ref{fig:vacios} haciendo los cambios de variables $\beta \rightarrow \alpha L$, $E _0 \rightarrow \frac{E _0}{\alpha}$.}
\label{fig.finitas}
\end{figure*}

\begin{figure*}[t!]
    \centering
    \begin{subfigure}[t]{0.5\textwidth}
        \centering
        \includegraphics[height=1.45in]{Energias.pdf}
        \caption{}
        \label{fig.izquierda123}
    \end{subfigure}%
    ~ 
    \begin{subfigure}[t]{0.5\textwidth}
        \centering
        \includegraphics[height=1.45in]{Ls.pdf}
        \caption{}
%        \label{fig.derecha}
    \end{subfigure}
    \caption{En esta imagen se puede observar la dependencia de la energía de Casimir $E _0$ para distintos valores de $\alpha$ y $L$, puede verse en \ref{fig.izquierda123} que a medida que incrementa $\alpha$ la energía de vacío posee un máximo local mas abrupto.}
\label{fig:vacios}
\end{figure*}

\subsection{Integral compleja}\label{sec.finita.compleja}

En la sección \ref{sec.complejo} se mostró que la función $\zeta (s)$ puede representarse mediante una integral en el plano complejo, junto con posibles caminos de integración en la figura \ref{fig:contorno}. En los capítulos anteriores \ref{cap.sencillos}  y \ref{cap.singular} se utilizó el contorno \ref{fig.medio} para obtener la estructura de polos, en este capítulo se utilizará el contorno \ref{fig.derecha.derecha} para obtener tanto los polos como la parte finita  de  $\zeta \left( - \frac{1}{2} \right)$.

A diferencia de la seccion \ref{seq.2.com} donde se utilizó la función $M (\lambda )$ en vez de $F _1 ^{1} \left( 1+\frac{ \alpha}{2 \lambda i },2,2 i \lambda x \right)$ para calcular $\zeta (s)$ dado que ambas funciones poseen los mismos ceros, aquí se utilizará 
$F _1 ^{1} \left( 1+\frac{ \alpha}{2 \lambda i },2,2 i \lambda x \right)$, quedando la función $\zeta$ determinada por
\begin{equation}
	\zeta (s) = 
	\frac{1}{2 \pi i} \int _{\mathcal{C}} 
						\left( \frac{\lambda}{\mu} \right) ^{-2s}
						\partial _ \lambda 
						\log F _1 ^{1} 
						\left( 1+\frac{ \alpha}{2 \lambda i },
							2,2 i \lambda x 
							\right)												
						d \lambda
	\, .
\end{equation}
Utilizando las variables adimensionales $\beta = \alpha L$ y  $\rho = \lambda L$ la ecuación anterior puede reescribirse de la forma
\begin{align}
\label{eq.ultima.int}
	\zeta (s) =& 
	\frac{\left(L \mu \right)^{2s}}{2 \pi i} \int _{\mathcal{C}} 
	f (\rho , \beta) \rho ^{-2s} d \rho 
\, ,
\end{align}
donde $f( \rho, \beta)$ está dada por
\begin{align}
f(\rho, \beta) =& 	
i
\frac{
		\left(1 + \frac{ \beta}{2 i \rho} \right) 
		F _1 ^1 
			\left( 2 + \frac{ \beta}{2 i \rho} ,3 ,2 i \rho \right)
		+ \left( \frac{\beta				
				}
				{2 \rho ^2 } 
				\right)
				( F _{1} ^1 ) ^{(1,0,0)}
				\left( 1 + \frac{\beta}{2 i \rho} ,2 ,2 i \rho
						\right)
		}
		{F _1 ^1 \left( 1 + \frac{\beta}{2 i \rho},2,2 i \rho \right)} 
\, ,	
\nonumber
\end{align}
donde $( F _{1} ^1 ) ^{(1,0,0)} (a,b,z)$ significa $ \partial _a F _{1} ^1  (a,b,z)$.


Utilizando el camino de integración \ref{fig.derecha.derecha}, la integral \ref{eq.ultima.int} puede reescribirse como suma de cuatro integrales
\begin{align}
\label{eq.zeta.completa.2}
	\zeta (s) 
&	
	= 
\\
\nonumber
&
- \frac{L ^{2s}}{2 \pi } 
\Bigg(	  e ^{- i \pi s} \int _0 ^{C _0}
			f (i t,\beta )
			t ^{-2s}  dt 
		+ e ^{- i \pi s} \int _{C _0} ^{\infty}
			f (i t,\beta )
			t ^{-2s}  dt 
\\
\nonumber
&
		+ e ^{i \pi s} \int _{0} ^{C _0} 
			f (-i t,\beta )
			t ^{-2s}  dt 
		+ e ^{i \pi s} \int _{C _0} ^{\infty}
			f (-i t,\beta )
			t ^{-2s}  dt 
	\Bigg)
\, .
\end{align}
Donde la primer integral del segundo y tercer renglón pueden calcularse de manera numérica en $s= -\frac{1}{2}$ dado son convergentes. Las otras dos integrales contienen los polos de $\zeta \left(- \frac{1}{2} \right)$ calculados en la ecuación (\ref{eq.result.zeta.c}), para calcular la parte finita de estas ultimas se utiliza el desarrollo \eqref{eq.completa} obteniendo
\[ 
f   ( i t ,\beta )=
\begin{cases} 
	  f _{+} ( t, \beta) = 
	  i  \left(
			\frac{1}{t} - \frac{\beta}{2 t ^2 } + \frac{\beta}{2 t^2}
			\log (2 t) + \frac{\beta \gamma}{2 t^2} 
			\right) + O (t ^{-3})
\\
	  f _{-} ( t, \beta) =
      i  \left(
			- \frac{1}{t} + \frac{\beta}{2 t ^2 } - \frac{\beta}{2 t^2}
			\log (2 t) - \frac{\beta \gamma}{2 t^2} +2
			\right) + O (t ^{-3})
   \end{cases}   
\]
Donde $f (\pm i t ) = f _{\pm} (t) $.
Utilizando este desarrollo las ultimas dos integrales pueden expresarse
\begin{align}
\nonumber
	\int _{C _0} ^{\infty}
			f (i t,\beta )
			t ^{-2s}  dt =& 
	\int _{C _0} ^{\infty}
		\left(
			f (it, \beta) - f _{+} (t, \beta )			
				\right) t ^{-2s} dt 
\label{eq.arriba1}
\\ &+ 
	\int _{C _0} ^{\infty}
			f _{+} ( t, \beta)
			 t ^{-2s} dt  \\
\nonumber
	\int _{C _0} ^{\infty}
			f (-i t,\beta )
			t ^{-2s}  dt 
=& 
	\int _{C _0} ^{\infty}
		\left(
			f (-it, \beta) - f _{-} (t, \beta )			
				\right) t ^{-2s} dt 
	\\ &+ 
\label{eq.arriba2}
	\int _{C _0} ^{\infty}
			f _{-} ( t, \beta)
			 t ^{-2s} dt
\, .
\end{align}
Donde las integrales en las que se sustrajo $f _{+},f_ {-}$ pueden integrarse numéricamente en $s=- \frac{1}{2}$ dado que son convergentes. 
Las contribuciones divergentes están dadas por
\begin{align}
\label{arriba}
&
	\int _{C _0} ^{\infty}
			f _{+} (t, \beta )			
			 t ^{-2s} dt =  
	O \left( s + \frac{1}{2} \right)
\\[5pt]
\nonumber			
&+
	i \left(- C _0 
		    - \frac{\beta \log C_0 (\gamma + \log 2 - 1 ) 
		    		}{2} 
		    - \frac{\beta \log ^2 C _0}{4}
		    + \frac{\beta ( \gamma + \log 2 -1 )}{4 (s + 1/2)} 
		    + \frac{\beta}{8 (s + \frac{1}{2}) ^2}
					\right)
\\[5pt]
\label{abajo}
&
	\int _{C _0} ^{\infty}
			f _{-} ( t, \beta)
			 t ^{-2s} dt =
	O \left(s + \frac{1}{2} \right)
\\[5pt]
\nonumber
&+
	i \left(C _0 
			- C _0 ^2
		    + \frac{\beta \log C_0 (\gamma + \log 2 - 1 ) 
		    		}{2} 
		    + \frac{\beta \log ^2 C _0}{4}
		    - \frac{\beta ( \gamma + \log 2 -1 )}{4 (s + 1/2)} 
		    - \frac{\beta}{8 (s + \frac{1}{2}) ^2}
					\right)
\end{align}
Utilizando las ecuaciones (\ref{abajo},\ref{arriba},\ref{eq.arriba2},\ref{eq.arriba1}) en (\ref{eq.zeta.completa.2}) se obtiene
\begin{align}
\zeta \left( - \frac{1}{2}  + \epsilon \right) &=
- \frac{i}{2 \pi L \mu} 
\Bigg(	  
		 \int _{C _0} ^{\infty}
			\left(
					f (i t,\beta )
					- f (-i t,\beta )
					- f _{+} (t) 
					+ f _{-} (t)
					\right)
			t   dt   \nonumber
\\ \nonumber
&+
		 \int _{- C _0} ^{C _0}
			f (i t,\beta )
			t   dt 	
	\Bigg)
\\ \nonumber
&
	- \frac{\beta \log ^2 C _0}{4 \pi L \mu}
	- \frac{\beta \log C _0 (\gamma + \log 2 -1 )}{2 \pi L \mu} 
	- \frac{16 C_0 - 8 C _0 ^2 + \pi ^2 \beta}{16 \pi L \mu}
\\ \nonumber
&
	+\frac{\beta \log ^2 (L \mu )}{4 \pi L \mu} 
	+ \frac{\beta \log  (L \mu) (\gamma + \log 2 -1)}{2 \pi L \mu}
\\ 
&	+ \frac{\beta}
		 {8 \pi L \mu  \epsilon ^2} +
	\frac{\beta (\gamma + \log (2 L \mu) -1 ) }
		 {4 \pi L \mu  \epsilon } 
\, .
\end{align}
Donde en la última linea están los polos que coinciden con lo calculado anteriormente en (\ref{eq.zeta.final},\ref{eq.result.zeta.c} y\ref{eq.res.1}), en la penultima linea se encuentra la dependencia no trivial con la escala $\log (L \mu)$ que coincide con la calculada con el método anterior en la ecuación (\ref{eq.zeta.final}).

Utilizando este resultado se obtiene para la energía de vacío
\begin{align}
\label{eq.vacio.completa.finita}
\nonumber
	E _0 (\epsilon )&=  
		\frac{1}{4 \pi i L} 
		\left(
			\int _{-C _0} ^{C _0} f (i t) t dt
			+ \int _{C _0} ^{\infty}  \left( f(i t) - f(-i t) - f _{+} (t) + f _{-} (t) \right) t dt
			\right) 
\\ \nonumber &
	- \frac{\beta \log ^2 (C _0) }{8 \pi L}
	-\frac{\beta \log (C _0) (\gamma -1 + \log 2  )}{4 \pi L}
	+ \frac{C _0 ^2}{4 \pi L}
	- \frac{C _0}{2 \pi L}
	- \frac{\pi \beta}{32 L}
\\ \nonumber &
	+ \frac{\beta \log ^2 L \mu}{8 \pi L}
	+ \frac{\beta (\gamma -1 + \log 2  )  \log L \mu}{4 \pi L}
\\ &
	+ \frac{\beta}{16 \pi L \epsilon ^2}
	+ \frac{\beta (\gamma -1 + \log 2 L \mu )}{8  \pi L \epsilon }
	\, .
\end{align}


\subsection{Comparaciones}

Al momento de graficar las energías de vacío se definio la Energía Regularizada $E _{Reg}$ en las ecuaciones \eqref{integral.ante} y \eqref{integral.ultima}, lo que corresponde tomar $\mu L = 1$ en     \eqref{energia.vacio.final} y \eqref{eq.vacio.completa.finita} respectivamente, de manera que la energía de vacío no dependa de la escala $\mu$.
\begin{align}
\nonumber
&
	\frac{E_ {Reg} ( \beta )}{\alpha}  =
	\frac{ \log ^2 \left( \frac{ 1 }{\pi} \right)}{8 \pi}  +
		\frac{ 
			 \log \left( \frac{ 1 }{\pi}\right)
				( \log (2 \pi ) + \gamma -1 )}  
			{4 \pi }  
	- \frac{ (\gamma + \log (2 \pi ) )}{4 \pi}
\\[5pt]
&
+
	\frac{\pi}{2 \beta}  
			\left(
				- \frac{1}{12} +
				\frac{\beta}{2 \pi ^2} 
				\left(
					\gamma \log (2 \pi)
					+ \gamma _1
					\right) +
								\frac{\gamma ^2 \beta }{2 \pi ^2} +
								\frac{1}{\pi} \sum _{p=2} ^{\infty}
								a_p \zeta (p) 
							\right) 
\, .
\label{integral.ante}
\end{align}
\begin{align}
\nonumber
&
	\frac{E _{Reg} ( \beta ) }{\alpha} =  
		\frac{1}{4 \pi i L} 
		\left(
			\int _{-C _0} ^{C _0} f (i t) t dt
			+ \int _{C _0} ^{\infty}  \left( f(i t) - f(-i t) - f _{+} (t) + f _{-} (t) \right) t dt
			\right) 
\\  &
	- \frac{\beta \log ^2 (C _0) }{8 \pi L}
	-\frac{\beta \log (C _0) (\gamma -1 + \log 2  )}{4 \pi L}
	+ \frac{C _0 ^2}{4 \pi L}
	- \frac{C _0}{2 \pi L}
	- \frac{\pi \beta}{32 L}
	\, .
\label{integral.ultima}
\end{align}

En la figura \ref{fig.finitas} se muestran las energías de vacío adimensionalizacionalizadas de la forma $\frac{E _{Reg}}{\alpha}$ y $E _{Reg} L$. En el rango $0 < \beta < 2$ las curvas se solapan perfectamente, en el intervalo $2 < \beta < 8$ puede observarse una ligera diferenicia entre ambas curvas (lo cual se espera que al aumentar la cantidad de términos $a _p$ disminuya), luego a partir de $10 < \beta$ el método analítico  deja de converger y empieza a exibir un crecimiento polinómico.

En la fígura \ref{fig:vacios} se encuentra graficada la energía de vacio para distintos valores de $\alpha$ y $\beta$, generadas a partir de la interpolacion de los puntos obtenidos con el método integral que se presenta en la figura \ref{fig.finitas}. En esta figura puede observarse que la energía de vacío posee un máximo local alrededor de $\alpha L \sim 0.2$ el cual se hace mas pronunciado al aumentar $\alpha$, este punto determina una región donde la energía de vacío pasa de ser atractiva a repulsiva, lo cual ocurre independientemente de cuan pequeño sea $\alpha$.


\chapter{Aproximación númerica utilizando Monte Carlo}
{\label{cap.singular}}

En el presente capitulo se utilizarán simulaciones numéricas para obtener los mismos resultados  (poner figuras), 



\section{Algoritmo de las lineas-v}


Dados dos puntos inicial y final $ y_0$, $ y_N$ respectivamente, el algoritmo se basa en crear un ensamble de lineas que unan estos dos puntos siguiendo una distribución de velocidad gaussiana.
 


\begin{equation}
\mathcal{N} \int _{y_0} ^{y_N} \mathcal{D} y 
e^{-\frac{N}{4} S _{W} [y]} :=
\mathcal{N} \int \prod _{j=1} ^{N-1} d ^{D} y _{j} e^{- \frac{N}{4} \sum _{i=1} ^{N} ( y _i - y _{i-1} )^2 } ,
\end{equation}

el objetivo es realizar un sistema de cambios lineales de manera que la distribución de probabilidad se vuelve puramente guassiana, con este fin se completan cuadrados para la variable $ y_1$


\begin{equation}
S _{W} = 	2 \left( y _1 - \frac{y_0 + y_2}{2} \right) ^2 + 
			\frac{1}{2} \left( y ^2 _2 + y _0 ^2 \right)   -
			y _0 y_2 +
			\sum _{i = 3} ^{N} (y _i - y _{i-1}) ^2 
\end{equation}
Donde existe un solo término que contiene $y_0$, se define entonces la variable
\begin{equation}
z _1 := y_1 - \frac{y _0  + y_2 }{2} ,
\end{equation}
realizando el mismo procedimiento para la variable $y_1$ se obtiene


\begin{equation}
S _W = 2 z_1 ^2 +
		\frac{3}{2} \left( y _2 - \frac{y _0 + 2 y_3}{3} \right) ^2 +
		\frac{1}{3} \left( y _3 ^2 + y _0 ^2 \right) -
		\frac{2}{3} y_0 y_3 +
		\sum _{i = 4} ^{N} (y _i - y _{i-1}) ^2 
		,
\end{equation}
se obtiene entonces una forma cuadratica utilizando el cambio de variables 
\begin{equation}
z _2 := y_2 - \frac{y _0  + 2 y_3 }{3} ,
\end{equation}
la expresión general luego de completar cuadrados en la variable $y_i$ queda determinada por 
\begin{equation}
a _i y _i ^2 - 2 y_i ( y _{i+1} + b _i y_0 ) =
a _i \left( y _i - \frac{y _{i+1} + b_i y_0}{a_i} \right) ^2 -
\frac{\left( y _{i+1} + b_i y_0 \right) ^2}{a _i}
\end{equation}


Donde los coeficientes $a_i$ y $b_i$ están dados por el sistema de ecuaciones recurrentes

\[ 
f   ( i t ,\beta )=
\begin{cases} 
	  a_{i+1} = 2 - \frac{1}{a_i},  \, a _1 = 2
\\
	  	  b_{i+1} = \frac{b _i}{a _i},  \, b _1 = 1
   \end{cases}   
\] 

, donde la solución está dada por
\begin{equation}
a _1 = \frac{i+1}{i} , \, b_i = \frac{1}{i} ,
\end{equation}

por lo tanto la forma general de las variables $z _i$ queda definida por
\begin{equation}
z _i = y _i - \frac{y_0}{i+1} - \frac{i}{i+1} y _{i+1} ,
\end{equation}

reescribiendo la acción con estas nuevas variables, el exponente queda diagonalizado de la forma

\begin{equation}
S _{W}  = \sum _{i = 1} ^{N-1} \frac{i +1}{i} z _i ^2 + c y _0 ^2 + d y _N ^2
, 
\end{equation}
los valores $c$ y $d$ no son importantes ya que son constantes que se cancelarán con la correspondiente normalización.
En resumen el algoritmo de las lineas v dados $N-1$ puntos intermedios queda determinado por 
\begin{enumerate}
\item Generar $N-1$ numeros $w_i$ que estén distribuidos de acuerdo a la distribución de velocidades $e^{- w _{i} ^2}$
\item normalizar las variables $w _i$ para obtener las variables auxiliares $z _i$
	\begin{equation}
	z _i = \sqrt{\frac{4}{N}} \sqrt{\frac{i}{i+1}} w _i
	\end{equation}

\item Una vez obtenidos los puntos $ z_i$ se calculan los puntos $ y_i$ por medio de la formula recursiva
	\begin{equation}
	y _i = z _i + \frac{1}{i+1} y_0 + \frac{i}{i+1} y _{i+1} ,
	\end{equation}
\end{enumerate}
en el caso donde $y_0 = y _N$ se obtienen loops cerrados de $N$ puntos.

\section{Cálculo Asintótico de los autovalores}\label{seq.2.asin}



\thispagestyle{empty}
\chapter{Conclusiones}

Se estudío la estructura de polos del operador singular $A = - \partial ^2 + \frac{\alpha}{x} $ en el compacto $[0,L]$ y en el continuo $[0, \infty)$, en ambos casos no solo que se presentó un polo doble en $s= -\frac{1}{2}$, sino que ademas coinciden.
\begin{align*}
&
	\zeta \left( - \frac{1}{2} + \epsilon \right) = 
	\frac{\alpha}{8 \pi \mu  \epsilon  ^2} +
	\frac{\alpha \left( \log (2 L \mu ) + \gamma -1  \right)}{4 \pi \mu  \epsilon } +
	f \left( - \frac{1}{2} \right)
\\
&
	\left| f \left( - \frac{1}{2} \right) \right| < \infty
\end{align*}
Luego se estudio a través de dos métodos diferentes la parte finita de $\zeta \left( - \frac{1}{2} \right)$ obteniendo una representación fundamental de la energía de vacío en función de un parámetro adimensional $\beta$ a partir del cual se pudieron generar energías de vacío para distintos valores de $\alpha,L$.

Se estudio la estructura completa de polos de $\zeta \left( - \frac{1}{2} \right)$ obteniendo que al igual que en el caso regular, los polos son todos simples (excepto en $s= - \frac{1}{2}$) y están ubicados en los semienteros negativos.




\thispagestyle{empty}


%----------------------------------------------------------------------------------------
%	APÉNDICES
%----------------------------------------------------------------------------------------

%\addtocontents{toc}{\vspace{1cm}} 
% Agrega espacios en la toc pero de manera rara

\appendix % Los siguientes capítulos son apéndices

%Incluye los apéndices en el folder de apéndices

\chapter{Calculo de las correciones asintoticas la Funcion Hypergeometrica}

n el capitulo 3, cuando se calcularon asintoticamente los autovalores, se utilizo la aproximacion:  

\begin{equation}
    F _1 ^1 (a,b,z) = \Gamma (b) 
    \left(
    \frac{e^z z ^{a-b} }{\Gamma(a)} * A_1 + \frac{(-z) ^{ -a}}{ \Gamma(b-a)} 
    * A_2
    \right)
\end{equation}

Donde $A_1$ y $A_2$ quedan determinadas por:

\begin{equation}
\begin{array}{c}
    A _1 = 
    \sum _{n=0} ^{R-1} 
    \left(
    \frac{(a) _n (1+a-b) _n}{n!} (-z) ^{-n} + O(|z| ^{-R})
    \right) \\
    A _2 = 
    \sum _{n=0} ^{S-1} 
    \left(
    \frac{(b-a) _n (1-a) _n }{n!} z ^{-n} + O(|z| ^{-s})
    \right)
\end{array}
\end{equation}

Donde $(a)_ n$ es el símbolo de Pochhammer, las primeras correcciones en $A_1$ y $A_2$ quedan,Luego de reemplazar $a,b,z$ por sus valores:

\begin{equation}
\begin{array}{c}
    A _1 = 1 + \frac{a(1-a-b)}{-z} + O(|z| ^ {-2})
    =  - \frac{\beta}{4 \mu ^2} + \frac{\beta ^2}{8 i \mu ^3} + O(|\mu| ^{-3}) \\ 
    A _2 = 1 + \frac{(b-a)(1-a)}{z} + O(|z| ^{-2}) = 
    - \frac{\beta ^2}{8 i \mu ^3} - \frac{\beta}{4 \mu ^2} + 
    O(|\mu| ^{-3})
\end{array}
\end{equation}

De aquí puede verse que la primer corrección asintótica entra al orden mas bajo en la potencia $\mu ^2$, que no contribuye al polo en $s= -1/2$ ya que la ecuación asintótica, el del orden $\frac{1}{\mu ^2}$ cuando quiero calcular el polo en $s=-3/2$.

\chapter{Función-\texorpdfstring{$\zeta$}{}  de Riemman} \label{Apendice.2}

En este apéndice se estudiará la extensión analítica y la estructura de polos de la función-$\zeta$ de Riemman $\zeta _R (s)$.

La función $\zeta _R (s)$ se define como la prolongación analítica de la serie
\begin{align}
	\zeta _R (s) = 
	\sum _{n=1} ^{\infty} \frac{1}{n ^{s}}
	\, ,
\end{align}
donde utilizando la identidad
\begin{align}
	\frac{1}{n ^{s}} =
	\frac{1}{\Gamma (s)} 
	\int _0 ^{\infty} t^{s-1} e ^{-n t}  dt
	\, ,
\end{align}
se obtiene la representación integral
\begin{align}
	\zeta _R (s) = 
	\frac{1}{\Gamma (s)}
	\int _0 ^\infty
	\frac{t ^{s-1}}{e ^t -1} dt
	\, ,
\end{align}
con el fin de obtener una extensión analítica se puede utilizar el desarrolo de Maclaurin
\begin{equation}
	\frac{1}{e ^t -1} = 
	\sum _{n=-1} ^{\infty}
	\frac{ \mathcal{B} _{n+1}}{(n+1)!} t ^n
	\, ,
\label{eq.ap.des}
\end{equation}
donde $\mathcal{B} _{n}$ son los llamados {\it Números de Bernoulli}, sumando y restando este desarrollo hasta el orden $N$ se obtiene una extensión analítica dada por
\begin{align}
\nonumber
	\zeta _R (s) &= 
	\frac{1}{\Gamma (s)}
	\int _0 ^1 
	t ^{s-1} 
	\left(	
		\frac{1}{e ^t -1} -
		\sum _{n=-1} ^{N}
		\frac{ \mathcal{B} _{n+1}}{(n+1)!} t ^n	
		\right)		
		+
	\frac{1}{\Gamma (s)}
	\int _1 ^\infty
	\frac{t ^{s-1}}{e ^t -1} dt
\\
	&+
	\frac{1}{\Gamma (s)}
	\sum _{n=-1} ^{N}
	\frac{ \mathcal{B} _{n+1}}{ (n+1)! (n + s)}
	\, ,
\end{align}
así $\zeta _R (s)$ puede ser calculada $\forall s \in \mathds{C}$ tal que $Re (s) > -N-1$, utilizando esta representación se obtiene dos conocidos valores de $\zeta _R (s)$ en $s = \pm 1$
\begin{align}
	\zeta _R (-1) = 
&
	\lim _{s \rightarrow -1}
			\frac{1}{\Gamma (s)} \frac{\mathcal{B} _2}{2! (s+1)}
			 =
	- \frac{1}{12}
\\
	\zeta _R (\epsilon \rightarrow 1) =
& 
	\frac{1}{\Gamma (1)} \frac{\mathcal{B} _0}{ \epsilon} + O (\epsilon -1 ) = 
	\frac{1}{ \epsilon} + O (\epsilon -1 ) 
	\, .
\end{align}
Otra propiedad importante es que $ \zeta _{R}$ es una función meromorfa con un único polo en $s = 1$, para esto se desarrolla el término $\frac{1}{\Gamma (s)} \frac{ \mathcal{B} _{n+1}}{ (n+1)! (n + s)}$ alrededor de $s = -n$ obteniendo
\begin{align}
	\lim _{s \rightarrow -n }	
	\zeta ( s  ) = 
	\frac{1}{\Gamma (s)}
	\frac{ \mathcal{B} _{n+1}}{ (n+1)! (n + s)}
	+ O (s + n) = 
	\frac{(-1) ^n \mathcal{B} _{n+1}}
		{n+1}
	\, ,
\end{align}
con lo que se concluye que existe un único polo en $s = 1$.














%\chapter{Otras condiciones de contorno}\label{Apendice.3}



En este apéndice se demostrará que el operador \ref{operador} solamente admite condiciones de contorno Dirichlet en ambos extremos
\begin{equation}
A = - \partial ^2 _x + \frac{\alpha}{x}
\end{equation}





\thispagestyle{empty}


%\addtocontents{toc}{\vspace{2cm}} % Agrega espacio en la toc


%----------------------------------------------------------------------------------------
%	BIBLIOGRAFÍA
%----------------------------------------------------------------------------------------
\backmatter
\nocite{*}
\bibliographystyle{plain}
\bibliography{bib.bib} %Aquí ponen el nombre del archivo .bib


\end{document}
