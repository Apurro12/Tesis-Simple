\chapter{Función-\texorpdfstring{$\zeta$}{}  de Riemman} \label{Apendice.2}

En este apéndice se estudiará la extensión analítica y la estructura de polos de la función-$\zeta$ de Riemman $\zeta _R (s)$.

La función $\zeta _R (s)$ se define como la prolongación analítica de la serie
\begin{align}
	\zeta _R (s) = 
	\sum _{n=1} ^{\infty} \frac{1}{n ^{s}}
	\, ,
\end{align}
donde utilizando la identidad
\begin{align}
	\frac{1}{n ^{s}} =
	\frac{1}{\Gamma (s)} 
	\int _0 ^{\infty} t^{s-1} e ^{-n t}  dt
	\, ,
\end{align}
se obtiene la representación integral
\begin{align}
	\zeta _R (s) = 
	\frac{1}{\Gamma (s)}
	\int _0 ^\infty
	\frac{t ^{s-1}}{e ^t -1} dt
	\, ,
\end{align}
con el fin de obtener una extensión analítica se puede utilizar el desarrolo de Maclaurin
\begin{equation}
	\frac{1}{e ^t -1} = 
	\sum _{n=-1} ^{\infty}
	\frac{ \mathcal{B} _{n+1}}{(n+1)!} t ^n
	\, ,
\label{eq.ap.des}
\end{equation}
donde $\mathcal{B} _{n}$ son los llamados {\it Números de Bernoulli}, sumando y restando este desarrollo hasta el orden $N$ se obtiene una extensión analítica dada por
\begin{align}
\nonumber
	\zeta _R (s) &= 
	\frac{1}{\Gamma (s)}
	\int _0 ^1 
	t ^{s-1} 
	\left(	
		\frac{1}{e ^t -1} -
		\sum _{n=-1} ^{N}
		\frac{ \mathcal{B} _{n+1}}{(n+1)!} t ^n	
		\right)		
		+
	\frac{1}{\Gamma (s)}
	\int _1 ^\infty
	\frac{t ^{s-1}}{e ^t -1} dt
\\
	&+
	\frac{1}{\Gamma (s)}
	\sum _{n=-1} ^{N}
	\frac{ \mathcal{B} _{n+1}}{ (n+1)! (n + s)}
	\, ,
\end{align}
así $\zeta _R (s)$ puede ser calculada $\forall s \in \mathds{C}$ tal que $Re (s) > -N-1$, utilizando esta representación se obtiene dos conocidos valores de $\zeta _R (s)$ en $s = \pm 1$
\begin{align}
	\zeta _R (-1) = 
&
	\lim _{s \rightarrow -1}
			\frac{1}{\Gamma (s)} \frac{\mathcal{B} _2}{2! (s+1)}
			 =
	- \frac{1}{12}
\\
	\zeta _R (\epsilon \rightarrow 1) =
& 
	\frac{1}{\Gamma (1)} \frac{\mathcal{B} _0}{ \epsilon} + O (\epsilon -1 ) = 
	\frac{1}{ \epsilon} + O (\epsilon -1 ) 
	\, .
\end{align}
Otra propiedad importante es que $ \zeta _{R}$ es una función meromorfa con un único polo en $s = 1$, para esto se desarrolla el término $\frac{1}{\Gamma (s)} \frac{ \mathcal{B} _{n+1}}{ (n+1)! (n + s)}$ alrededor de $s = -n$ obteniendo
\begin{align}
	\lim _{s \rightarrow -n }	
	\zeta ( s  ) = 
	\frac{1}{\Gamma (s)}
	\frac{ \mathcal{B} _{n+1}}{ (n+1)! (n + s)}
	+ O (s + n) = 
	\frac{(-1) ^n \mathcal{B} _{n+1}}
		{n+1}
	\, ,
\end{align}
con lo que se concluye que existe un único polo en $s = 1$.













