\chapter{Función-\texorpdfstring{$\zeta$}{}  de Riemman} \label{Apendice.2}

En este apéndice se demostraran las propiedades de la función $\zeta$ de Riemman $\zeta _R (s)$

La función $\zeta _R (s)$ se define como la prolongación analítica de la siguiente serie
\begin{align}
	\zeta _R (s) = 
	\sum _{n=1} ^{\infty} \frac{1}{n ^{s}}
	\, ,
\end{align}
para lo cual se puede utilizar la identidad
\begin{align}
	\frac{1}{n ^{s}} =
	\frac{1}{\Gamma (s)} 
	\int _0 ^{\infty} t^{s-1} e ^{-n t}  dt
	\, ,
\end{align}
obteniendo una representación integral de $\zeta _R (s)$
\begin{align}
	\zeta _R (s) = 
	\frac{1}{\Gamma (s)}
	\int _0 ^\infty
	\frac{t ^{s-1}}{e ^t -1} dt
	\, ,
\end{align}
para lograr dar una expresión de $\zeta _R (s) $ se utilizá el desarrollo de taylor 
\begin{equation}
	\frac{1}{e ^t -1} = 
	\sum _{n=-1} ^{\infty}
	\frac{ \mathcal{B} _{n+1}}{(n+1)!} t ^n
	\, ,
\label{eq.ap.des}
\end{equation}
donde $\mathcal{B} _{n}$ son los {\it Números de Bernoulli}, utilizando esta representación $\zeta _R (s)$ queda expresada como
\begin{align}
	\zeta _R (s) = 
&
	\frac{1}{\Gamma (s)}
	\int _0 ^1 
	t ^{s-1} 
	\left(	
		\frac{1}{e ^t -1} -
		\sum _{n=-1} ^{N}
		\frac{ \mathcal{B} _{n+1}}{(n+1)!} t ^n	
		\right)		
		+
	\frac{1}{\Gamma (s)}
	\int _1 ^\infty
	\frac{t ^{s-1}}{e ^t -1} dt
\\
	+
&
	\frac{1}{\Gamma (s)}
	\sum _{n=-1} ^{N}
	\frac{ \mathcal{B} _{n+1}}{ (n+1)! (n + s)}
	\, ,
\end{align}
así $\zeta _R (s)$ podrá ser calculada $\forall s $ aumentando el orden del desarrollo \eqref{eq.ap.des}, dos valores característicos de $\zeta _R (s)$ están dados por
\begin{align}
	\zeta (-1) = 
&
	\lim _{s \rightarrow -1}
			\frac{1}{\Gamma (s)} \frac{\mathcal{B} _2}{2! (s+1)}
			 =
	- \frac{1}{12}
\\
	\zeta (\epsilon \rightarrow 1) =
& 
	\frac{1}{\Gamma (1)} \frac{\mathcal{B} _0}{ \epsilon} + O (\epsilon -1 ) = 
	\frac{1}{ \epsilon} + O (\epsilon -1 ) 
	\, .
\end{align}
otra propiedad importante es la no existencia de polos exceptuando $s=1$, para esto se ve que la única posible contribución a un polo son los términos de la forma  $	\frac{1}{\Gamma (s)} \frac{ \mathcal{B} _{n+1}}{ (n+1)! (n + s)}$, para ver esto puede desarrollarse $\Gamma (s)$ alrededor de $s = -n$ obteniendo
\begin{align}
	\lim _{s \rightarrow -n }	
	\zeta ( s  ) = 
	\frac{1}{\Gamma (s)}
	\frac{ \mathcal{B} _{n+1}}{ (n+1)! (n + s)}
	+ O (s + n) = 
	\frac{(-1) ^n \mathcal{B} _{n+1}}
		{n+1}
	\, ,
\end{align}
de lo cual se concluye que
\begin{align}
	\zeta ( -n  ) = 
	\frac{(-1) ^n \mathcal{B} _{n+1}}{n+1}
\end{align}














