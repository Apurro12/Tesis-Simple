\chapter{Estudio del Problema Singular}

En este capitulo se van a aplicar las herramientas desarrolladas en el capitulo anterior al estudio del espectro de un operador singular

\section{El Operador Singular}

Como trabajo de tesis se propondra aplicar los metodos desarrollados en el capitulo anterior al operador diferencial (\ref{operador}).

\begin{equation}
\begin{array}{c}
    A \phi (x) = - \partial ^2 _x  \phi(x) + \frac{\alpha}{x} \phi(x) \\
    \phi(0) = \phi(L) = 0 
\end{array}
\label{operador}
\end{equation}

Para eso voy a resolver la ecuación de autovalores (\ref{eq.aut.sin}).

\begin{equation}
\begin{array}{c}
    A  \phi (x)  =   \omega ^2 \phi (x) \\ 
    \omega \ \in \ \mathfrak{R}
\end{array}
\label{eq.aut.sin}
\end{equation}

La cual posee soluciones LI $ y_1 $ y $ y_2 $ dadas por :

\begin{equation}
    \phi (x) = 
    \underbrace{
    C[1] \ e ^{-i \omega x} \ x \ F _{1} ^{1} (1+\frac{ \alpha}{2 \omega i },2,2 i \omega x) } _ {y_1}
    + \underbrace{C[2] \ e^{-i \omega x } \ x \ U (1+\frac{ \alpha}{2 \omega i },2,2 i \omega x) } _{y_2} 
\end{equation}


%Para autovalor $- \omega ^2 $ obtengo las siguientes soluciones:

%\begin{equation}
%    \phi ^{-} (x) = 
%    \underbrace{
%    C[1] \ e ^{-i \omega x} \ x \ F _{1} ^{1} (1 + \frac{ \alpha}{2 \omega},2,2 i \omega x) } _ {y_1}
%    + \underbrace{C[2] \ e^{-i \omega x } \ x \ U (1 + \frac{ \alpha}{2 \omega},2,2 i \omega x) } _{y_2} 
%\end{equation}

Donde $F _1 ^1(a,b,z)$ y $ U(a,b,z)$ son las soluciones LI de la ecuacion hypergeometrica (\ref{eq:hyper})

\begin{equation}
    z \ \partial ^2 _z \ \psi (a,b,z) + (b-z) \
    \partial _z \psi (a,b,z)
    -a \ \psi (a,b,z) = 0
\label{eq:hyper}
\end{equation}

Las cuales tienen las siguientes expresiones analiticas  : 

\begin{equation}
\begin{array}{c}
	U(a,b,z) = \frac{1}{\Gamma (a)} 
	\int _0 ^{\infty} e ^{-zt}
	t ^{a-1}
	(1+t) ^{b-a-1}
	dt \\
	F _1 ^1 (a,b,z) = \sum _ {k=0} ^{\infty} 
	\frac{(a) _k}{(b) _k} 
	\frac{z ^k}{k!} 
\end{array}
\end{equation}

%\textbf{Estados Ligados:} \\

%Primero voy a buscar estados ligados, imponiendo la condicion $\phi ^{-} (0) = 0$, veo que $U(1- \frac{\alpha}{2 \lambda} ,2 ,0) \rightarrow \infty $ obtengo $C[2] = 0$

%Los estados ligados van a surgir entonces de la condicion de contorno en $x=L$

%\begin{equation}
%	F _1 ^{1} (1- \frac{\alpha}{2 \omega} , 2 2 i \omega L ) = L 
%\end{equation}


\textbf{Estados de Scattering:} \\
%    A partir de aquí $\phi ^{+} (x)$ la escribiré $\phi (x)$ \\


Para aplicar la condicion de contorno $\phi (0) = 0$, hay que hacer un desarrollo de $\phi(x \rightarrow 0)$, el cual es:

\begin{equation}
\begin{array}{c}
\phi (x \rightarrow 0) \approx
C[1] ( x + O(x ^2)) + 
C[2] x 
\left( 
\frac{1}{  \alpha x  \Gamma ( \frac{ \alpha}{2 i \omega}  )   }  +
\frac{Log[x] }{\Gamma ( \frac{ \alpha}{2 i \omega} ) } + Cte + O(x)
\right)
\\
Donde,  \ Cte = 
\frac{
-1 + 2 \gamma + Log[2 i \lambda] + \psi (1 + \frac{ \alpha}{2 i \omega})
}
{\Gamma (\frac{i \alpha}{2 \lambda})}
\end{array}
\label{eq.scat}
\end{equation}

Donde se ve que $y _1 (x \rightarrow 0 ) \rightarrow 0$ y $y _2 (x \rightarrow 0)  \rightarrow
\frac{1}{  \alpha   \Gamma ( \frac{i \alpha}{2 \lambda}  )   } $ , entonces la forma de satisfacer la condicion de contorno es poniendo C[2] = 0

Los autovalores estaran dados entonces por los ceroes de $y_1 (x= L)$, como el único termino que se anula en el produco es la funcion Hypergeometrica, los autovalores vienen dados por los ceros de la ecuacion (\ref{eq.1}), donde se puede ver su comportamiento en la figura (\ref{fig:funcion}) en funcion de $\omega$ para  $\alpha=1, \ L=1$.:


\begin{equation}
y_1 (L, \omega) = F _1 ^1 (1+\frac{ \alpha}{2 i \omega},2,2 i \omega L)  = 0
\label{eq.1}
\end{equation}

\begin{figure}
\centering
\includegraphics[scale=0.7]{Funcion.jpg}
\caption{En esta imagen se puede ver el comportamiento de los ceros de función $F _1 ^1$, para $\alpha=1$ y $L=1$}
\label{fig:funcion}
\end{figure}

A partir de ahora en vez de trabajar con la función $F _1 ^1$, voy a trabajar con el primer termino de su desarrollo en serie a $ \omega \rightarrow \infty  $ dado por (\ref{eq.aprox}) 

\begin{equation}
    F _1 ^1 (a,b,z) = \Gamma (b) 
    \left(
    \frac{e^z z ^{a-b} }{\Gamma(a)} * A_1 + \frac{(-z) ^{ -a}}{ \Gamma(b-a)} 
    * A_2
    \right)
\label{eq.aprox}
\end{equation}

Donde $A_1$ y $A_2$ son los demas terminos de la serie, que voy a tomar como 1, en el apéndice 1 voy a demostrar que no contribuyen a la estructura de los polos al orden que estoy calculando, si hay que tenerlos en cuenta para calcular los polos mas halla de donde se calcularon en esta tesis.

Definiendo las variables adimensionales $\mu = \omega L$  y $\beta = \alpha L $ y reemplazando los valores correspondientes en la ecuación (\ref{eq.aprox}) obtengo. 

\begin{equation}
    F _1 ^1 (1+  \frac{  \beta}{2 i \mu} ,2 ,2 i \beta ) = 
   i  \frac{e ^{- \frac{\pi}{4} \frac{\beta}{\mu} } }{2 \mu}
    \left( -
    \frac{e ^{- i \frac{\beta}{2 \mu} Ln(2 \mu) } e ^{2 i \mu} }{\Gamma(1+\frac{ \beta}{2 i \mu})} +
    \frac{e ^{  i \frac{\beta}{2 \mu} Ln(2 \mu) }}               {\Gamma(1-\frac{ \beta}{2 i \mu})}
    \right)
\label{eq.completa}
\end{equation}




La cual queda expresada como un producto de dos funciones, una parte que se anula y una parte que no.

Para obtener los autovalores voy a solo ocuparme de la parte que se anula, la cual esta entre paréntesis.

\begin{equation}
    M (\mu) = -
    \frac{e ^{- i \frac{\beta}{2 \mu} Ln(2 \mu) } e ^{2 i \mu} }{\Gamma(1+\frac{ \beta}{2 i \mu})} +
    \frac{e ^{  i \frac{\beta}{2 \mu} Ln(2 \mu) }}               {\Gamma(1-\frac{ \beta}{2 i \mu})}
\label{eq.aproxx}
\end{equation}

Para el calculo de la funcion $\zeta _A (s) $ se puede utilizar ($\mu$) o cualquier multiplo de ella, ya que posee los mismos autovalores que $F_1 ^1 $

%En todas las cuentas posteriores, para calcular $\zeta _A (s)$ voy a utilizar $M ( \mu )$ ya que posee los mismos autovalores de $ F _1 ^1 $.



\section{Calculo Asintótico de los autovalores}



Necesito conocer los ceros de  de $M(\mu)$ a $\mu \rightarrow{\infty}$, suponiendo que $\mu _n$ la puedo descomponer en una parte divergente mas una infinitesimal cuando $n \rightarrow{\infty}$, tal como en el capitulo anterior.

Para que el desarrollo sea mas simple, voy a voy a escribir a $M (\mu)$ como :

\begin{equation}
M (\mu) = e ^{\frac{i \beta Log[2 \mu]}{\mu}}
\frac{\Gamma (1- \frac{i \beta}{2 \mu})}{\Gamma (1 + \frac{i \beta}{2 \mu})}
- e ^{2 i \mu}
\label{eq.otro.mu}
\end{equation}


En el limite de $\mu _n \rightarrow \infty$ se obtiene:

\begin{equation}
    M(\mu _n \rightarrow \infty) = 
	1 - e ^{2 i \mu}
\end{equation}

De acá se puede ver que el comportamiento de $\mu _n$ esta dado por 


\begin{equation}
\begin{array}{c}
    \mu _n = n \pi + \epsilon _n \\
    Donde \ \epsilon _n \rightarrow{0} ,\ si \ n \rightarrow{0}
\end{array}
\label{eq.mu2}
\end{equation}



Para calcular orden a orden el termino $\epsilon _n$ inserto la ecuación (\ref{eq.mu2}) en (\ref{eq.otro.mu}) e igualo la ecuación a cero, lo cual conduce a la siguiente ecuación para $\mu _n$

\begin{equation}
	e ^{ i \frac{\beta}{ \mu _n} Ln(2 \mu _n)}     
    \frac{\Gamma(1 + \frac{ \beta}{2 i \mu _n} ) }
    {\Gamma(1 - \frac{ \beta}{2 i \mu _n})} =    
    e ^{2 i \epsilon _n }
\end{equation}

teniendo en cuenta que $\frac{ln(2 \mu _n)}{2 \mu _n }$ tiende a cero a $\mu _n$ grade, puedo hacer un desarrollo de la exponencial y las funciones $\Gamma$ alrededor de $ \mu \rightarrow \infty $, y $\epsilon \rightarrow 0$

\begin{equation}
    \left(
    \sum _{p = 0} ^{\infty} \frac{( i \frac{\beta}{ \mu} Log[2 \mu]) ^p }{p!}
    \right)
    \left(
	\sum _{q = 0} ^{\infty} \frac{a _q}{\mu ^q}
	\right)
    =
    \left(
    \sum _{l = 0} ^{\infty} \frac{( 2 i \epsilon)^l}{l !}
    \right)
\end{equation}


Al orden mas bajo se obtiene la ecuación : 

\begin{equation}
(1 + \frac{i \beta}{ \mu} Log[2 \mu)] 
(1 + \frac{i  \gamma \beta}{ \mu})  =
(1 + 2 i \epsilon)
\end{equation}

Todo esto conduce a que el termino subdominante sea $\epsilon _n =  \frac{\beta Ln(2 n \pi)}{2 n \pi}$ , pero con esto no alcanza, ya que existe un termino que decae como 1/n, entonces para calcular el polo que nos interesa se deben calcular todos los términos que decaigan como 1/n, insertando $\epsilon _n =  \frac{\beta Ln(2 n \pi)}{2 n \pi} + \epsilon '$ en la ecuación anterior, obtengo:


\begin{equation}
    \epsilon _n =  \frac{\beta Ln(2 n \pi)}{2 n \pi} 
                \frac{\gamma \beta}{2 n \pi} +
                O(\frac{1}{n^2})
\end{equation}

Luego para calcular la función $\zeta _{A}$ utilizo su definición

\begin{equation}
\begin{array}{c}
    \zeta _A (s) = \sum _n ^{\infty} \omega _n ^{-s}  =
    \sum _{n=1} ^{\infty} \left(\frac{\mu _n}{L} \right) ^{-2 s} =  \\
    L ^{2 s} \sum _{n=1} ^{\infty} 
    \left( 
    n \pi + \frac{\beta Ln(2 n \pi)}{2 n \pi} + \frac{\gamma \beta}{2 n \pi} +
    O(\frac{1}{n^2})
    \right) ^{-2s} = \\
    
\end{array}
\end{equation}

Como en el ejemplo anterior puedo escribir todo como

\begin{equation}
\begin{array}{c}
    \zeta _A (2 s) = \left( \frac{L}{\pi} \right)  ^{2 s} 
    \sum _{n=1} ^{\infty} n ^{- 2  s}
    \left(
    1 - \frac{\beta Ln(2 n \pi)}{2 n^2 \pi ^2} - \frac{\gamma \beta}{2 n^2 \pi ^2 } +
    O(\frac{1}{n^3})  \right) ^{-2 s} = \\
    \sum _{n=1} ^{\infty} n ^{-2 s} 
    \left(
    1 -\chi _n \right) ^{- 2 s}
\end{array}
\end{equation}

Utilizando como en el ejemplo anterior el desarrollo de la ecuacion para $\chi \xrightarrow 0$, puedo expresar mi funcion zeta como 


\begin{equation}
\begin{array}{c}
    \zeta _A (s) = ( \frac{L}{\pi} ) ^{2 s}
    \sum _{n=1} ^{\infty} 
    n ^{-2s}
    \left(
    1 + 2 s \chi + O(\chi ^2)
    \right) =  \\
    ( \frac{L}{\pi} ) ^{2 s}
    \left(
    \sum _{n=1} ^{\infty} n ^{-2 s} 
    \left(
    1 - 2s (
    \frac{\beta Log[2 n \pi]}{2 n ^2 \pi ^2} + 
    \frac{\gamma \beta}{2 n ^2 \pi ^2} +
    O (n ^{-3}  )
    \right)
    \right)
\end{array}
\end{equation}


\begin{equation}
\begin{array}{c}
    \zeta _A (S) = 
    \left( \frac{L }{ \pi } \right) ^{2}  \\
    \left(
    \zeta (2 S) -
	\frac{ s \beta}{ \pi ^2}
	\left(
	\zeta (2s+2)  ( Log[2 n \pi ] + \gamma) +
	\zeta '(2s+2) \
	\right)
    \right)
\end{array}
\end{equation}

Donde sabiendo que $\zeta(s) = \frac{1}{s-1} + Regular$, el primer polo queda determinado como

\begin{equation}
    Res(s=1/2) = \frac{L}{2 \pi}    
\end{equation}

Aquí sabiendo el desarrollo de $\zeta(s)$ alrededor de $s=1$ puedo conocer los desarrollos de $\zeta'(2s+2)$, que lo utilizo para desarrollar alrededor de mi segundo polo en $s = -1/2$ obteniendo:

\begin{equation}
    \zeta _A (s) =  \frac{\beta}{8 L \pi (s+1/2)^2} +
    \beta \frac{(-1 + \gamma + Log[2L ])}{4 L \pi (s+1/2)} + 
    Regular
\end{equation}

\section{Calculo Utilizando Variable Compleja}


Sabiendo que los autovalores de mi Hamiltoniano son todos reales, puedo expresar $\zeta _A (s)$ como una integral en el plano complejo, y deformar la trayectoria hasta el camino dado por la figura [\ref{fig:contorno}], tal como se hizo en el capitulo anterior: \\

Mi funcion $ \zeta _A (s) $ va a quedar definida por:

\begin{equation}
\zeta _A (s) = 
\frac{1}{2 \pi i} 
\int _{\mathcal{C}}
\frac{M ' ( \mu ) }{ M ( \mu ) } d \mu
\label{eq.zeta.compleja}
\end{equation}

Donde $M ( \mu )$ está dada por (\ref{eq.aproxx}).
cuando integre los ejes verticales, voy a obtener un termino exponencialmente decreciente y uno decreciente, que se van a alternar dependiendo de si estoy arriba o abajo del eje real. \\

Reemplazando la parametrizacion $ \mu  (t) = \pm i t$ en (\ref{eq.zeta.compleja}) y tirando los terminos exponencialmente decrecientes, obtengo:


\begin{equation}
\begin{array}{c}
    \zeta _A (s) = \\
     \frac{1}{2 \pi i} \int _{\infty} ^{1}
     \frac{\beta}{2 t^2} 
     \left(
     1 - \frac{i \pi}{2} + Log[2 t] + \psi (1 + \frac{\beta}{2 t})
     \right)
     t ^{-2s}
     e ^{- i \pi s} (i dt) + \\
     \frac{1}{2 \pi i} \int _{\infty} ^{1} 
     \left(
     2 + \frac{\beta}{2 t^2}
     \left(
     1 + \frac{i \pi}{2} - Log[2 t] - \psi (1+ \frac{\beta}{2 t})
     \right)
     t ^{-2s}
     e ^{ i \pi s} (-i dt)
     \right)     
\end{array}
\end{equation}

\begin{equation}
\begin{array}{c}
    \zeta _A (s) = \\
     \frac{1}{2 \pi i} \int _{\infty} ^{1}
     \frac{i \alpha}{2 t^2}
     \left(
     1 - Log(2 L t) - \frac{i \pi}{2} + \psi (1-\frac{\alpha}{2 t})
     \right)
     t^{-2 s}
     e^{-i \pi s} \ 
     (i dt) + \\
     \frac{1}{2 \pi i} \int _1 ^{\infty}
     \frac{ \alpha}{2 i t^2}
     \left(
     \left(
     1 - Log(2 L t) - \frac{i \pi}{2} + \psi (1-\frac{\alpha}{2 t}) 
     \right)
     + 2 i L
     \right)
     t^{-2 s}
     e^{i \pi s}
     (-i dt)
     
\end{array}
\end{equation}

Donde antes de reacomodar los terminos puedo calcular el termino que contiene $2iL$ el cual es la potencia mas alta de $t$ para obtener: 

\begin{equation}
    \frac{1}{2 \pi i }
    \int _1 ^{\infty}
    2 i L
    e^{i \pi s}
    t ^{-2 s}
    (-i dt) =  
    \frac{L e^{i \pi s} }{2 \pi i} \frac{1}{s-1/2   }
\end{equation}

Ningún otro termino aportará a este polo, entonces el residuo en $s= 1/2$ es:

\begin{equation}
    Res (s=1/2) = \frac{L}{2 \pi}
\end{equation}

Que se corresponde con el calculado con el método anterior. \\

Una vez calculado este termino, puedo reorganizar el resto de la integral como:

\begin{equation}
\begin{array}{c}
    - \frac{\alpha}{2 \pi} \ sin[\pi s]
    \int _1 ^{\infty}
    t ^{-2 s-s} 
    \left(
    1 - Log[2Lt] + \psi (1- \frac{\alpha}{2t})
    \right) dt - 
    \frac{\alpha}{4} 
    Cos[\pi s]
    \int _1 ^{\infty} t^{-2s-s} dt
\end{array}
\end{equation}

Donde todos los términos son calculables analíticamente, excepto el que esta multiplicado por $\psi$, para lo cual utilizo el desarrollo de $\psi$ en $t \rightarrow \infty$:

\begin{equation}
    \psi(1-\frac{\alpha}{2 t}) \approx 
    -\gamma + O(\frac{1}{t})
\end{equation}

Realizando todas las integrales la funcion $ \zeta _A (s)$ a este orden queda determinada por:  

\begin{equation}
\begin{array}{c}
    \zeta (s) _{A} = 
    \frac{L e ^{i \pi s}}{2 \pi i} \frac{1}{s-1/2} 
    -\frac{\alpha Sin[\pi s]}{4 \pi} \frac{1}{s+1/2} \\
    \alpha 
    \frac{
    1+S Log(4)+2SLog[L]+Log[2L]
    }
    {8 \pi} \frac{sin(\pi s)}{(s+1/2) ^2}  \\
    \frac{\gamma \alpha Sin[\pi s]}{4 \pi } \frac{1}{s+1/2} 
    \frac{\alpha cos(\pi s) }{8 \pi}  \frac{1}{s+1/2}  \\
\end{array}
\end{equation}

Donde el siguiente termino en el desarrollo de $ \psi $ contribuirá al residuo en $s = -3/2$

Para calcular el residuo en $s=-1/2$ desarrollo todo en Serie de Laurent alrededor de ese punto obteniendo.

\begin{equation}
    - \frac{\alpha}{8 \pi} \frac{1}{(s+1/2)^2} + 
    \alpha \frac{1-\gamma -Log[2 L]}{4 \pi} \frac{1}{s+1/2} + \ Regular
\label{eq.desarrollo}
\end{equation}

Donde se puede ver que el residuo, y la contribución al polo cuadrático en $s=-1/2$ coinciden con las calculadas anteriormente.

%Una aclaracion importante es que el termino $Log[2 L ]$ es correcto dimensionalmente, ya que proviene de evaluar la integral $\int _1 ^\infty  Log[2 L \omega] \ d \omega$

\section{Cálculo de la energía de vacío}

La energía de vacío queda determinada por la ecuación 

\begin{equation}
    E _0 = \frac{\hbar}{2}  
    \zeta (s)  |  _{- \frac{1}{2}}
\end{equation}

$\zeta (s)$ ya está desarrollada alrededor de $s=-1/2$ en (\ref{eq.desarrollo}), queda desarrollar $(E_c) ^{2s+1} $ alrededor de $s=-1/2$ quedando.

\begin{equation}
    E _c \approx 
    1 + 2 Log[E_c] (s + 1/2) +
    2 Log[E_c] ^2 (s+1/2) ^2 + 
    O (s+1/2)^3
\end{equation}

Entonces la energía de vació queda determinada por:

\begin{equation}
    E _0 =
    \left(
    \frac{1}{(s+1/2)^2} 
    \left(
    \frac{- \alpha}{8 \pi}
    \right)+
    \frac{
    \alpha(1 -\gamma-Log[2L]) - 
    \alpha Log[E_c] 
    }{4 \pi (s+1/2)} 
     + Regular
    \right) | _{s=-1/2}
\end{equation}\\



Para calcular la parte finita que no está calculada voy a volver a calcular la funcion $\zeta _A (s) $ utilizando variable compleja pero, utilizando todos los terminos de la expancion en seríe de la funcion Hypergeometrica

\begin{equation}
 \left(
 \frac{e ^{  \frac{i \alpha Log[2 \lambda L ]}{2 \lambda } } e ^{2 i \lambda L } }
 {\Gamma ( 1 - \frac{i \alpha}{2 \lambda} )} S1 - 
 \frac{e ^{ -  \frac{i \alpha Log[2 \lambda L ]}{2 \lambda } } }
 	  {\Gamma (1 + \frac{i \alpha}{2 \lambda})} S2 
 \right)
\end{equation}

Al igual que antes voy a tener una parte exponencialmente creciente/decreciente mas la parte curva , sin realizar explicitamente los calculos como antes, los integrandos quedan:

\begin{equation}
\begin{array}{c}
 \left( \frac{1}{2 \pi i} \right) ^{-1 } \zeta _A (s) = \\
\int _{arriba} \lambda ^{-2s } \partial _{\lambda}
	\left(
		- \frac{i \alpha}{2 \lambda} Log [2 \lambda] - Log[\Gamma [1 + \frac{i \alpha}{2 \lambda}]] +
		Log[ S2 ]
		\right) d \lambda
	+ \\ \\
\int _{abajo} \lambda ^{-2s } \partial \lambda
	\left(
		2 i \lambda + \frac{i \alpha}{2 \lambda} Log [2 \lambda] - 
		Log[\Gamma [1 - \frac{i \alpha}{2 \lambda}]] +
		Log[ S1 ]
		\right)	d \lambda
	+ \\ \\
\int _{circulo} \lambda ^{-2s } \partial \lambda \ Log \left[
					\frac{e ^{\frac{i \alpha Log[2 \lambda L ]}{2 \lambda}} e ^{2 i \lambda L} S1}
					{\Gamma \left[ 1 - \frac{i \alpha}{2 \lambda} \right]} - 
					\frac{e ^{\frac{-i \alpha Log[2 \lambda L ]}{2 \lambda}} S2}
					{\Gamma \left[ 1 + \frac{i \alpha}{2 \lambda} \right]}					
					\right] d \lambda
\end{array}
\end{equation}

Donde se puede ver que los únicos cambios respecto a la funcion $\zeta _A (s)$ calculada antes, es la suma de dos factores que van con el logatirmo de las series que acompañan a las exponenciales, mas la parte circular.

La integral angular queda (Llamando como simepre $M[\lambda]$ al argumento):

\begin{equation}
\begin{array}{c}

\frac{1}{2 \pi i} \int _{\pi /2 } ^{0} 
\frac{e ^{-2 s i \theta} d \theta}{M [e ^{i \theta}]} \\

\Bigg[

\frac{
e ^{\frac{i \alpha (Log[2 L] + i \theta)}{2 e ^{i \theta} }} e ^{2 i L e ^{i \theta}}

	\left(
		\left(
			\frac{i \alpha}{2 e ^{2 i \theta} } - 
			\frac{i \alpha( Log[2 L ] + e ^{i \theta} ) }{2 e^{2 i \theta}}
			- \frac{i \alpha \psi \left( 1 - \frac{i \alpha}{2 e ^{i \theta}}\right)}
				   {2 e ^{2 i \theta}}
			\right) S1 [e ^{i \theta}] +
		S'1 [e ^{i \theta }]
		\right)
}{\Gamma \left( 1 - \frac{i \alpha}{2 e ^{i \theta}} \right)} + \\

\frac{
e ^{- \frac{i \alpha (Log[2 L] + i \theta)}{2 e ^{i \theta} }}

	\left(
		\left(
			\frac{-i \alpha}{2 e ^{2 i \theta} } + 
			\frac{i \alpha( Log[2 L ] + e ^{i \theta} ) }{2 e^{2 i \theta}}
			+ \frac{i \alpha \psi \left( 1 - \frac{i \alpha}{2 e ^{i \theta}}\right)}
				   {2 e ^{2 i \theta}}
			\right) S2 [e ^{i \theta}] +
		S'2 [e ^{i \theta }]
		\right)
}{\Gamma \left( 1 + \frac{i \alpha}{2 e ^{i \theta}} \right)}

\Bigg]

\end{array}
\end{equation}



Las otras dos integrales, solo me van a contribuir a la parte finita los terminos

\begin{equation}
\begin{array}{c}
\frac{e ^{- i \pi s} }{2 \pi}
\int _ {\infty} ^{1}
\frac{S2 ' [i t]}{S2 [i t]}
t ^{-2 s } dt - 
\frac{e ^{ i \pi s} }{2 \pi}
\int _ {1} ^{\infty}
\frac{S1 ' [-i t]}{S1 [-i t]}
t ^{-2 s } dt  - \\
\frac{\alpha}{2 \pi}
\int _{1} ^{\infty }
t ^{-2s-2}
\left(
\psi \left( 1 - \frac{\alpha}{2 t} \right )+ \gamma
\right) dt
\end{array}
\end{equation}



Donde las primeras dos integrales se pueden juntar en una sola dado que $S1(x) = S2 (x) ^{*}$