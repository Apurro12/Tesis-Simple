\chapter{Introduccion}


\section{Aplicaciones Físicas de la Regularizacion}

En Teoría Cuántica de Campos el calculo de ciertas magnitudes físicas (Acción efectiva, energía de Casimir, Espectacion del Vacio, etc) conducen a valores formalmente divergentes, por lo cual se requiere un metodo para regularizar estas magnitudes, en lo siguiente se hará un resumen de la energía de Casimir, y la Accion Efectiva, junto con los metodos regularizacion Heat-Kernel y Funcion-$ \zeta _A (s) $.\\





\textbf{Acción Efectiva}\\


En Teorías Cuantícas de Campos, la Funcional Generatriz es el elemento basico mediante el cual se calculan todas las magnitudes físicas, utilizando el tiempo euclideo está dada por:

\begin{equation}
Z[J] = 
\int D \phi
e^{-\frac{1}{\hbar} S[\phi] + \frac{1}{\hbar} (J,\phi) }  =
e ^{-\frac{1}{\hbar} \Gamma[J] + \frac{1}{\hbar} (J,\phi )}
\end{equation}

Voy a hacer una traslacion del campo $\phi (x) = \phi _0 (x) + \psi (x) $, donde $\phi _0 (x)$ es la solucion que extremiza la accion clasica y $\psi (x)$ es la fluctuacion cuantica respecto a esa solucion, obteniendo:

\begin{equation}
Z[j] = 
\int D \phi
e^{-\frac{1}{\hbar} S[\phi _0 + \psi] + \frac{1}{\hbar} (J,\phi _0 + \psi ) }  
\end{equation}

Desarrollando la accion $S[\phi _0 + \psi]$ alrededor de $\phi _0$, y utilizando de $\phi _0$ es la solucion clasica obtengo la Accion Efectiva:

\begin{equation}
\Gamma [\psi] = 
- \hbar 
Log
\left[
	e ^{-\frac{1}{\hbar} S[\phi _0] }
	\int D \phi \
	e ^{-\frac{1}{2 \hbar} (\phi,A \psi) }
	\right]
\end{equation}



Obteniendo en la aproximacion a 1-loop para la accion efectiva:

\begin{equation}
\Gamma [\psi] = S [\psi] + \frac{\hbar}{2} Log \ Det (A) +
O ( \hbar ^2 )
\end{equation}


Si el operador $A$ tiene una base completa de autofunciones $ \{ \lambda _n \} _{n \in N}$ su determinante se puede escribir como:

\begin{equation}
Det A = \underset{ n \in N }{ \Pi } \ \lambda _n
\end{equation}

Donde por lo general conduce a una cantidad divergente que debe ser regularizada.\\


\textbf{Energía de Casimir:} \\ 

En el año 1948 Hendrik Casimir tomando la idea de que dos moleculas neutras se atraen debido a las fuerzas de Van der Waals, llego a la conclucion que dos placas melaticas paralelas neutras en el vacio sufren una fuerza atractiva dada por (\ref{casimir.1}) aunque , la primer medicion del efecto fue en 1958 por Sparnaay que no pudo ser conclusivo debido al 100 \% de insertidumbre en la medicion, se han echo varias mediciones para distintas configuraciones (ya que la fuerza de Casimir depende la geometria), el experimento mas preciso fue echo por U. Mohideen y Anushree Roy, con una diferencia entre teoria y experimento del 1 \%, en la cual midieron la fuerza entre una esfera metalica y una placa plana, donde se utilizo un microscopio de fuerza atómica \cite{BORDAG20011} .


\begin{equation}
\begin{array}{c}
F(d) = - \frac{\pi ^2 \hbar c}{240} \frac{A}{d^4} \\
\end{array} 
\label{casimir.1}
\end{equation}




En el caso de dos placas paralelas el efecto casimir se puede ver, calculando el valor de espetacion del vacio del campo entre las placas, el campo va a satisfacer la ecuacion de Klein-Gordon.

\begin{equation}
( \partial _0 ^2 - \nabla  ^2  ) \phi (\vec{x} ,t) = 0 
\end{equation}

Descomponiendo al campo en modos normales de oscilacion, llego a la ecuacion de autovalores:

\begin{equation}
\begin{array}{c}
\phi ( \vec{x},t) = e ^{-i \omega t} \phi ( \vec{x}) \\
\nabla ^2 \phi ( \vec{x}) = - \frac{\omega ^2}{c ^2} \phi ( \vec{x})
\end{array}
\end{equation}

Imponiendo que la funcion de onda se anule sobre los bordes conductores, y condiciones de contorno periodicas sobre la direccion transveral (para luego tomar el limite yendo a infinito), los modos normales de oscilacion estan dados por:

\begin{equation}
\omega _n = c \sqrt{ k _x ^2 + k _y ^2 + \left( \frac{n \pi}{L} \right) ^2 }
\end{equation}

Teniendo en cuenta las dos polarizaciones posibles del campo electromagnetico obtengo para la energía de vacio:

\begin{equation}
E _0 = \frac{A \hbar }{(2 \pi) ^2} \int dk _x dk _y 
\sum _{n=1} ^{\infty} 
c
\sqrt{
		\left( \frac{n \pi}{a } \right) ^2 + k _x ^2 + k _y ^2
		}
\end{equation}


Lo cual conduce a una energía que es divergente, la fuerza estaría dada por la derivada respecto de $L$, para obtener un resultado finito se requiere un proceso de regularizacion.

\section{Heat Kernel y Funcion Zeta}


En los trabajos \cite{ Seeley:1967ea,10.2307/2373309,10.2307/2373312} se han estudiado trazas de operadores diferenciales $A$ con coeficientes derivables, actuando sobre variedades compactas $M$ con borde suave $\partial M$, una de las funciones espectrales que se pueden definir es lo que se llama $\zeta _A (s)$ (Funcion zeta del operador A), si el espectro del operador $A$ está dado por $ \{ \lambda _n \} _{n \in N}$ la funcion $\zeta _A (s)$ queda expresada como:


\begin{equation}
\zeta _A (s) = Tr A ^{-s} = \sum _{n \in N}  \lambda _n ^{-s}
\label{funcion.zeta}
\end{equation}

La cual en principio converge para valores grandes de $s$, pero una vez calculada se puede hacer la prolongacion analítica al plano complejo, en particular la funcion $\zeta _A (s)$ va a tener polos simples en $x _n$ dados por:   :

\begin{equation}
x _n = \frac{m-n}{d} 
\label{eq.ceros.zeta}
\end{equation}

Donde $m$ es la dimension de la variedad,$d$ el orden del operador A y $n= 0,1,2,3 ...$ \\

%En esta tesis se estudiaran los polos de la funcion $\zeta _A (s)$ de un operador singular unidimensional de segundo orden.
Tambien es posible definir otra funcion dependiente el espectro de A, La traza del Heat-Kernel \cite{VASSILEVICH2003279}, que está dada por:

\begin{equation}
K (t) =  Tr \ e ^{-t A} = 
\sum _{n  \in N} e ^{-t \lambda _{n} }
\end{equation}

En el caso que el operador diferencial $A$ sea del tipo Laplace acuando con dondiciones de contorno local, sobre campos escalares $\phi $ :

\begin{equation}
\begin{array}{c}

A = - \left(
			g ^{\mu \nu} \nabla _{\mu} \nabla _{\nu} + V
			\right) \\
\left (\partial _m + S \right) \phi | _{\partial M} = 0

			

\end{array}
\end{equation}

Donde $\nabla$ es la derivada covariante, $V$ es el potencial y $\partial _m$ es la derivada normal con respecto al borde, $K(t)$ admite un desarrollo asintótico para valores pequeños de t  de la forma:

\begin{equation}
K(t) \approx 
\sum _{n=0} ^{\infty}
C _n (A) \ 
t ^{\frac{(-m+n)}{2}}
\label{eq.heat.expansion}
\end{equation}



Los primeros 5 terminos $C _n (A) $ están calculados en \cite{VASSILEVICH2003279}, la forma explicita de los primeros 3 es: 

\begin{equation}
\begin{array}{c}
C _0 (A) = (4 \pi ) ^{-m/2}  \int _M d ^m x \sqrt{g}  \\
C _1 (A) = \frac{(4 \pi) ^{-(m-1)/2} }{4} \int _{\partial M } d ^{m-1} \sqrt{h} \\
C _2 (A) = \frac{(4 \pi) ^{-m/2} }{6} \left(
									\int _M d ^m x\sqrt{g} (6 V + R) +
									\int _{\partial M } d ^{n-1} x 
									\sqrt{h} (2 L _{aa}  + 12 S)
									\right)
\end{array}
\end{equation} 

Donde $C _0$ y $C _1$ representan el volumen de la variedad y del borde respectivamente, $C _2$ así como el resto de los coeficientes son funciones del potencial $V$, el Campo de Gauge $\omega $, la condicion de contorno $S$, en tensor de curvatura de la variedad $R _{\mu \nu \rho \sigma }$ y el tensor de curvatura extrinseca $K _{\mu \nu }$ sobre el borde de la variedad. \\

A su vez funcion $\zeta _A (s) $ y $K(t)$ estan relacionadas a travez de la Transformada de Mellin.



\begin{equation}
\zeta (s) = \frac{1}{\Gamma (s) } 
\int _0 ^{\infty} dt \
t ^{s-1} K(t) 
\end{equation}

Se puede ver entonces que residuos de la funcion $\zeta _A (s)$ estan dados por:

\begin{equation}
Res[\zeta _A (s)] | _{s= m/2 - n/2} = \frac{C _n (A)}{\Gamma (n/2 + m/2)}
\end{equation}

Lo cual coincide con (\ref{eq.ceros.zeta}) para un operador del tipo Laplace. \\

\textbf{Regularizacion de la Accion Efectiva:} \\

Inspirados en las propiedades de las matrices se puede dar una definiciones al determinante funcional, de tal forma que sea convergene, para matrices de $ NxN$ se tiene la propiedad:

\begin{equation}
Log  Det \left( \ e ^B \right)  = Tr (B)
\end{equation}

Escribiendo entonces $A= e ^{B} \rightarrow A ^{-s} = e ^{-s B}$, se ve que:

\begin{equation}
\partial _s A ^{-s} | _{s=0} = (- B e ^{-s B}) | _{s=0} = - B
\end{equation}

Juntando todo obtengo:

\begin{equation}
Log Det ( A ) = Tr B = - Tr  \left( \partial _s A ^{-S} | _{s=0} \right) =
 - \partial _s \left( Tr A ^{-s} \right) | _{s=0}
\end{equation}

Puedo definir entonces el determinante funcional de un operador como:

\begin{equation}
Det A = e ^{- \zeta _A ' (0)}
\end{equation}

Lo cual conduce a una contribucion finita a 1-loop de la accion efectiva (siempre que $\zeta _A (0)$ sea regular). \\

\textbf{Regularizacion de la Energía de Casimir}

Al inicio de la seccion se calculó la energía de Casimir como:

\begin{equation}
E _0 = \frac{A \hbar }{(2 \pi) ^2} \int dk _x dk _y 
\sum _{n=1} ^{\infty} 
c
\sqrt{
		\left( \frac{n \pi}{a } \right) ^2 + k _x ^2 + k _y ^2
		}
\end{equation}

Para regularizarla voy a calcular la funcion $\zeta _A (s)$ del problema:

\begin{equation}
\zeta _A (s) = 
\int dk _x dk _y 
\sum _{n=1} ^{\infty} 
\left(	\left( \frac{n \pi}{a } \right) ^2 + k _x ^2 + k _y ^2
		\right) ^{-2s} = 
\frac{\pi}{s-1} \left( \frac{\pi}{a} \right) ^{2-2s} \zeta (2s-2)
\end{equation}

La Energía de Casimir, se correspondera con $\zeta _A (-1/2)$, obteniendo 


\begin{equation}
\zeta _A (-1/2) = 
- \frac{\pi ^4}{180 a ^3}
\end{equation}

La Energía de Casimir queda definida por:

\begin{equation}
E _0 =  \frac{A c \hbar}{(2 \pi) ^2}
\zeta _A (-1/2) =
- \frac{A \hbar c \pi ^2}
		{720 L ^3}
\end{equation}

Obteniendo una fuerza atractiva dada por:

\begin{equation}
F(L) = - \partial _L E _0 (L) = 
- \frac{A c \pi ^2 \hbar}{240 L^4}
\end{equation}

Que coincide con lo expresado al principio de la seccion. \\ \\


En el ejemplo anterior la funcion $\zeta _A (s) $ era regular en $s= -1/2$, entonces no fue necesario introducir un regulador, para obtener la funcion $\zeta _A (s)$ adimensional para todos los valores de $s$, se la define de la siguiente forma (suponiendo que $A$ tiene auvalores $\lambda _n ^2 $):

\begin{equation}
\zeta _A (s) = \sum _{n \in N} \left( \frac{\lambda _n}{\mu }  \right) ^{-2s } = 
\mu ^{2s} \sum _{n \in N } \lambda _n ^{-2s}
\end{equation}

Donde $\mu $ tiene unidades de $longitud ^{-1}$ . \\

Para que coincida con $\underset{ {n \in N}}{  \sum } \lambda _n$ en $s= -1/2$, se procede a definir la energía de vacio como:

\begin{equation}
E _ 0 = 
\frac{\hbar c}{2 }
\left(
	\mu ^{2s+1} \sum _{n \in N} \lambda _n ^{-2s} 
	\right) _{s=-1/2}
\end{equation}