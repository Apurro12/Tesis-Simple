\chapter{Introducción}


En Teoría Cuántica de Campos, el cálculo de ciertas magnitudes físicas (acciones efectivas, energías de vacío, amplitudes de dispersión, etc.), conduce, en general, a valores formalmente divergentes, por lo cual se requiere un método para extraer resultados finitos. 

En el presente capítulo se presentarán la acción efectiva y la energía de Casimir de un campo cuántico y se mostrará la aparición de divergencias relacionadas con los modos de altas energías del espectro de oscilaciones del campo. Posteriormente, se presentarán las funciones espectrales denominadas {\it heat-kernel} y {\it función-$\zeta$}, utilizadas para regularizar estas divergencias y obtener resultados finitos.

A lo largo de este capítulo y en toda la tesis se utilizará: tiempo euclídeo, $c=1$ y $\hbar =1$, aunque en algunos resultados de este capítulo, para exponer manifiestamente el carácter cuántico se mantendrá $\hbar$.

\section{Acción Efectiva}\label{accion_efectiva}

En esta sección, escribiremos la acción efectiva en su aproximación a {\it 1-loop} para un campo escalar $\phi(x)$. En general, el procedimiento puede repetirse para campos fermiónicos y de gauge, si se considera que las expresiones que siguen llevan implícitas las sumas sobre índices espinoriales, de Lorentz o de gauge, según corresponda.

Consideremos entonces un campo escalar $\phi(x)$ definido sobre una región $\mathcal{M}\subset \mathbb{R}^m$ con borde $\partial \mathcal{M}$, donde $x \in  \mathcal{M}$. La dinámica del campo $\phi(x)$ está determinada a partir de una acción $S[\phi]$
\begin{align}
\label{eq.dinamica_campos}
		S[\phi]=\int_\mathcal{M} dx\ \mathscr L(\phi,\partial\phi)\,,
\end{align}
 donde $\mathscr L$ es el lagrangiano que define la teoría. La solución clásica $\phi _0(x)$ de las ecuaciones de movimiento es la configuración que minimiza la funcional $S[\phi]$,
\begin{equation}
\begin{array}{c}
\left. \frac{\delta S [ \phi ] }{\delta \phi (x)}  \right| _{\phi = \phi _0  } = 0 \, . \\[10pt]
\end{array}
\end{equation}
Como ejemplo, se escriben a continuación los lagrangianos correspondientes a un campo escalar libre masivo, al campo electromagnético y a un fermión masivo, respectivamente. Se detallan también las correspondientes ecuaciones de movimiento de Klein-Gordon, Maxwell y Dirac:
\begin{equation}
\begin{array}{lcl}
\mathscr{L} = \frac{1}{2} (\partial _t \phi ) ^2 - \frac{1}{2} (  \nabla \phi ) ^2 - 
	\frac{m^2}{2} \phi ^2 
&\rightarrow& 
\left(
	\partial _t ^2 - \nabla ^2 + m^2 
		\right) \phi = 0 \\[8pt]
		
\mathscr{L} = - \frac{1}{4} F _{\mu \nu} F ^{\mu \nu}
&\rightarrow&
 \partial _{\mu} F ^{\mu \nu} = 0 \\[8pt]

\mathscr{L} =  { \bar{\psi} } \left(
			i \gamma ^{\mu} \partial _{\mu} - m 
			\right) \psi 
&\rightarrow&
			\left( i  \gamma ^{\mu} \partial _{\mu}  - m \right)\psi = 0.\\[10pt]
\end{array}
\label{campos}
\end{equation}

En Teoría Cuántica de Campos, el campo se reinterpreta como un operador ($\phi(x) \rightarrow \hat{\phi}(x)$) cuya dinámica está dada por \eqref{eq.dinamica_campos}. Una alternativa para cuantizar el campo $\phi(x)$ está dada por las integrales de camino de Feynman. En esta formulación, las funciones de correlación de la teoría (que determinan las amplitudes de dispersión) en presencia de una fuente externa $J(x)$ están dadas por\footnote{ Como estamos interesados en un análisis semiclásico, reintroducimos $\hbar$ en las integrales funcionales.}
\begin{equation}
\langle 0 | \hat{ \phi  } (x _1) \ldots \hat{\phi  } (x _n) | 0 \rangle = \frac{1}{\langle 0|0\rangle} 
\int  \mathscr D
\phi \ e ^{- \frac{1}{\hbar} S[ \phi ] + \frac{1}{\hbar} (J, \phi )} \phi (x _1) ... \phi (x _n)\,.
\label{valor}
\end{equation}
En esta expresión $(\,,\,) $ representa el producto interno de funciones en $\mathcal{M}$,
\begin{align}
	(J,\phi) = \int_\mathcal{M} dx\ J(x) \phi (x)\,.
\end{align}
Para calcular \eqref{valor} se define la {\it funcional generatriz} $Z[J]$ dada por
\begin{equation}
Z [J] = \langle0|0\rangle=
\int \mathscr D \phi \ e ^{- \frac{1}{ \hbar} S[ \phi ] + \frac{1}{\hbar} (J, \phi )}\,.
\label{eq.generatriz}
\end{equation}
Una vez calculada la funcional generatriz, los valores medios (\ref{valor}) quedan determinados por sus derivadas funcionales,
\begin{equation}
\begin{array}{c}
\langle 0 | \hat{ \phi  } (x _1) \ldots \hat{\phi  } (x _n) | 0 \rangle = \frac{\hbar ^n}{Z[J]}
\left. \frac{\delta ^n  Z[J] }{ \delta J(x_n) \ldots \delta J(x _1) } 		\right| _{J=0}\,,
\end{array}
\end{equation}
de modo que el valor medio del campo en presencia de la fuente $J$ está dado por
\begin{equation}
\begin{array}{c}
\phi _J (x) \equiv \langle 0| \hat{\phi } (x)| 0 \rangle = \hbar \left. \frac{\delta \log Z[J] }{\delta J(x)} \right| _{J=0} \,.
\end{array}
\end{equation}

Así, como la ecuación de movimiento para el campo clásico está dada por el mínimo de la acción $S[\phi]$, el campo $ \phi _J (x) $ está dado por el mínimo de la funcional $\Gamma[\phi]-(J,\phi)$, esto es,
\begin{equation}
\begin{array}{c}
\left.\frac{\delta \Gamma [ \phi ]  }{\delta \phi (x)  }\right|_{\phi=\phi_J} = 
J (x)\,,
\end{array}
\label{eq.accion1}
\end{equation}
donde $\Gamma[\phi]$, denominada {\it acción efectiva}, coincide con la acción clásica en el límite $\hbar\to 0$ y, en general, está definida por
\begin{equation}
\Gamma [\phi _J] = (J, \phi _J) -  \hbar \, \log Z [J]\,.
\label{efectiva}
\end{equation}

La acción efectiva $\Gamma[\phi]$ contiene la información sobre el comportamiento del campo cuántico: en particular, sus derivadas funcionales determinan las amplitudes de dispersión de las partículas descriptas por el campo $\phi$.
\begin{comment}
Para demostrar esto se toma su derivada funcional evaluada en el campo medio $\phi _J $.
\begin{equation}
\frac{\delta \Gamma [ \phi _J ]}{\delta \phi _J (x) } = 
J(x) + \int dx ' \frac{\delta J [\phi _J ]}{\delta \phi _J (x) } \phi _J (x) - 
\frac{1}{Z[J]} \int dx' \frac{\delta Z[J] }{\delta J(x')} \frac{\delta J[\phi _J ]}{\delta \phi _J (x)} = J(x) \\[8pt]
\end{equation}
\end{comment}
Sin embargo, sólo en algunos casos es posible calcular en forma exacta la acción efectiva, de modo que el procedimiento usual es obtener una aproximación semiclásica a partir de un cálculo perturbativo de $\Gamma [ \phi _J]$ en potencias de $\hbar$.

Para obtener las primeras correcciones cuánticas hacemos el cambio de variables $\phi (x) \rightarrow \phi(x) + \phi _J (x) $ en (\ref{eq.generatriz}), para luego desarrollar la acción clásica alrededor de $\phi _J (x)$,
 \begin{equation}
\begin{array}{c}
Z[J] = e ^{- \frac{1}{\hbar} S[ \phi _J ] + \frac{1}{\hbar} (J, \phi _J )} 
\int \mathscr D \phi\ e ^{ - \frac{1}{\hbar} (\delta S  - J, \phi ) - \frac{1}{2 \hbar}  (\phi,\delta ^2 S\, \phi)+\ldots }\,,
\end{array}
\end{equation}
donde
\begin{equation}
\delta S(x) = \left. \frac{\delta S[\phi]}{ \delta \phi (x) } \right| _{\phi = \phi _J}
\end{equation}
y $\delta^2S$ es el operador definido por el núcleo $\delta ^2 S(x_1,x_2)$ dado por
\begin{equation}
		\delta ^2 S(x_1,x_2) = \left. \frac{\delta ^2 S[\phi]}{ \delta \phi (x_1) \delta \phi (x_2) } \right| _{\phi = \phi _J}\,.
\end{equation}
$\delta^2S$ se denomina {\it operador de fluctuaciones cuánticas} y su espectro determina los {\it modos de oscilación} del campo $\phi(x)$. Haciendo el cambio de variables $\phi (x) \rightarrow \sqrt{\hbar}\, \phi (x) $, se obtiene la primera correción cuántica a la  acción, o la acción efectiva a {\it 1-loop},
\begin{equation}
\Gamma [\phi] = S [ \phi] - 
\hbar\,\log\int \mathscr D \phi \ e ^{- \frac{1}{2}  (\phi, \delta ^2 S\, \phi) } + O(\hbar ^2)\,,
\end{equation}
que puede escribirse como
\begin{equation}\label{gama-det}
\Gamma [\phi] = S [\phi] + \frac{\hbar}{2}\, \log{\rm Det}\, ( \delta ^2 S ) +
O ( \hbar ^2 )\,.
\end{equation}

Si $ \{ \lambda ^2 _n \} _{n \in \mathbb N}$ es el espectro del operador de fluctuaciones $ \delta ^2 S $, su determinante puede escribirse formalmente como
\begin{equation}\label{det}
{\rm Det}\,(\delta ^2 S) = \prod_{ n \in \mathbb N }\ \lambda ^2 _n\,.
\end{equation}
Sin embargo, para teorías locales, $\delta ^2 S$ es típicamente un operador diferencial, de modo que $\lambda ^2 _n\to\infty$ a medida que $n\to \infty$. Como el espectro diverge en el ultravioleta (UV), la expresión \eqref{det} es divergente y requiere un mecanismo de regularización. Ésta es una manifestación de las divergencias UV debidas a las fluctuaciones de primer orden a {\it 1-loop} del campo $\phi (x)$.


\section{Energía de Casimir}\label{sec.casimir}

Como se ve, las fluctuaciones cuánticas del campo introducen correcciones divergentes en la acción efectiva. Esto implica que el cálculo de amplitudes de dispersión entre partículas requiere una reinterpretación de los parámetros de la teoría por medio de la cual estas divergencias no aparezcan en el resultado de cantidades físicas. Se analizará ahora otra manifestación de las divergencias en teoría cuántica de campos pero, esta vez, los infinitos no provienen de las interacciones entre partículas ``físicas'' sino que se originan en el mismo ``estado de vacío'' del campo.


En el año 1948 Hendrik Casimir \cite{Casimir:1948dh}, inspirado en las fuerzas de Van der Waals que sienten dos moléculas neutras entre sí, llegó a la conclusión de que dos placas metálicas paralelas neutras de área $A$ y separadas por una distancia $d$ en el vacío sufren una fuerza atractiva dada por
\begin{equation}
		F(d) = -  \frac{\hbar c \pi ^2 \,A\  }{240 \, d^4}\,.
	\label{casimir.1}
\end{equation}
La primera medida experimental de este efecto fue realizada en 1958 por Sparnaay \cite{SPARNAAY1958751}; aunque el resultado de este experimento no contradijo la predicción de Casimir, la incertidumbre en la medición tampoco permitió un resultado concluyente. Desde entonces, se han realizado varios experimentos para distintas configuraciones. La precisión de los experimentos varían desde el $15 \%$ en el caso de placas paralelas \cite{casimir.placas.paralelas,articulo.casimir} en el año $2002$, al rango del $0.1-5 \%$ para una esfera y un plano a partir del año $2001$\cite{casimir.cilindro1,PhysRevLett.81.4549,PhysRevD.60.111101,PhysRevA.62.052109}.

La fuerza de Casimir se origina en la energía de vacío de los campos contenidos entre las placas conductoras. En efecto, cada uno de los modos normales de oscilación del campo tiene asociada una frecuencia de oscilación $\omega_n$. La cuantización del campo conduce entonces a un conjunto de osciladores desacoplados, de modo que la energía en el estado fundamental (correspondiente al vacío de la teoría) resulta
\begin{equation}
E _0 = \frac{\hbar}{2} \sum _n \omega _n\,.
\label{eq.casimir.div}
\end{equation}
Como el espectro de oscilaciones de un campo (que tiene infinitos grados de libertad) no está acotado superiormente, $\omega_n\to\infty$ si $n\to\infty$, la energía de vacío es divergente. Este infinito se origina entonces, en los modos de alta frecuencia del campo o, en otras palabras, en las configuraciones que presentan grandes oscilaciones en pequeñas distancias. Sin embargo, existen procedimientos que permiten aislar esta divergencia UV, de las cantidades físicas de la teoría y permiten expresar la dependencia de la energía de vacío $E_0$, con la distancia en términos de cantidades finitas.

En general, la energía de vacío entre dos placas recibe contribuciones de las oscilaciones de vacío de todos los campos de la teoría. No obstante, típicamente, las contribuciones de los campos masivos decrecen exponencialmente con la separación de las placas, de modo que, la fuerza observada se debe mayormente a la contribución del campo electromagnético. Por simplicidad, se analiza a continuación un campo escalar sin masa entre dos placas infinitas paralelas separadas por una distancia $d$.



Se considera entonces la ecuación de Klein-Gordon \footnote{Como estamos interesados en obtener explícitamente la ecuación \ref{casimir.1}, aquí se reintroducirá $c$.},
\begin{equation}
\left( \frac{1}{c^2} \partial _t ^2 - \nabla  ^2  \right) \phi (\vec{x} ,t) = 0 \,.
\end{equation}
Descomponiendo al campo en modos normales de oscilación, se llega a la ecuación de autovalores,
\begin{align}
\phi ( \vec{x},t) &= e ^{-i \omega _n t} \phi ( \vec{x}) \,,\\
\nabla ^2 \phi ( \vec{x}) &= - \frac{\omega _n ^2}{c ^2} \phi ( \vec{x})\,.
\end{align}
Si se impone que la función de onda se anule sobre los bordes conductores, las frecuencias propias de oscilación resultan
\begin{equation}
\omega _n = c\, \sqrt{ \left( \frac{n \pi}{d} \right) ^2 + k _x ^2 + k _y ^2   }
\end{equation}
donde $k=(k_x,k_y)$ es el impulso del campo en la dirección transversal a las placas.

La energía de vacío, queda entonces expresada por
\begin{equation}
E _0 = \frac{A \hbar }{(2 \pi) ^2} \int dk _x dk _y 
\sum _{n=1} ^{\infty} 
c
\sqrt{
		\left( \frac{n \pi}{d} \right) ^2 + k _x ^2 + k _y ^2
		}\,.
\end{equation}
Se introdujo un factor 2 para representar las dos polarizaciones del campo electromagnético. Puede verse que esta expresión es divergente tanto por las integrales en $k$ como por la suma sobre los infinitos modos en $n$. Presentaremos ahora métodos funcionales que permiten aislar estas divergencias UV y separar la dependencia de $E_0$ con $d$.

\section{Funciones Espectrales}

En los trabajos \cite{Seeley:1967ea,10.2307/2373309,10.2307/2373312} R.T. Seeley estudió la existencia y propiedades de la resolvente $(A - \lambda) ^{-1}$ de un operador diferencial $A$, con coeficientes suaves, definido sobre secciones de un fibrado vectorial sobre una variedad de base compacta $\mathcal{M}$ con borde suave $\partial \mathcal{M}$ bajo determinadas condiciones de contorno. Estos resultados están relacionados con las propiedades del operador pseudo-diferencial $A ^{-s}$ con $s \in \mathbb{C}$.



Dado un operador diferencial $A$ con espectro $\{ \lambda ^2 _n \} _{n \in \mathbb N}$, existen dos funciónes espectrales que serán relevantes para la regularización de las divergencias en teoría cuántica de campos, la {\it  función-$\zeta$} y la  {\it traza del  heat-kernel} (en adelante llamado simplemente heat-kernel):
\begin{align}
\zeta  (s) &= {\rm Tr}\ A ^{-s} = \sum \limits_{n \in \mathbb N}   \lambda _n ^{-2s} \label{funcion.zeta}\\[2mm]
K (T) &=  {\rm Tr} \ e ^{-T A} = \sum \limits_{n \in \mathbb N} e ^{-T \lambda ^2 _{n} }
\end{align}
Como típicamente $\lambda ^2 _n\to \infty$ si $n\to \infty$, $\zeta  (s) $ es analítica, si ${\rm Re}\,(s)$ es suficientemente grande; $K(T)$, por su parte, está bien definida para $T>0$. Los polos $s _n \in\mathbb C$ de $\zeta(s)$ están relacionados con el desarrollo de $K(T)$ para $T\to 0$; esto se deduce de la relación entre ambas funciones, a través de la transformada de Mellin,
\begin{equation}
\zeta (s) = \frac{1}{\Gamma (s) } 
\int _0 ^{\infty} dT \
T^{s-1} K(T) \,.
\label{eq.mellin}
\end{equation}

Consideremos, por ejemplo, el caso en que $A$ es un operador de tipo Laplace, actuando en campos escalares $\phi(x)$ sobre una variedad $\mathcal{M}$  (de dimensión $m$), con condiciones de contorno locales,
\begin{align}\label{eq.cc}
&A = - \left(
			g ^{\mu \nu} \nabla _{\mu} \nabla _{\nu} + V(x)	\right) \,,\\[2mm]
\label{eq.condiciones.contorno}						
&\left (\partial _m + S \right) \phi | _{\partial \mathcal{M}} = 0\,,
\end{align}
donde $\nabla _{\mu}$ es la derivada covariante, $V(x)$ es un potencial, $\partial _m$ es la derivada normal, con respecto al borde, apuntando para adentro y $S$ es un parámetro que caracteriza la condición de contorno. Se puede probar \cite{10.2307/2373309,10.2307/2373078} que $K(T)$ admite un desarrollo asintótico para valores pequeños de $T$  de la forma
\begin{equation}
K(T) \sim 
\sum _{n=0} ^{\infty}
C _n (A) \ 
T^{\frac{(n-m)}{2}} 
\label{eq.heat.expansion}
\end{equation}
Los coeficientes $C_n(A)$ se denominan coeficientes de Seeley-DeWitt del problema y dependen del potencial $V(x)$, del parámetro $S$, de los invariantes geométricos de la variedad y sus derivadas. Una expresión para los primeros cinco coeficientes $C _n (A) $ puede encontrarse en \cite{Vassilevich:2003xt}. Para el caso de condiciones de contorno del tipo \ref{eq.condiciones.contorno}, los primeros tres coeficientes están dados por
\begin{align}
\label{C_0}
C _0 (A) &= \frac{1}{(4 \pi ) ^{m/2} }  \int  _{\mathcal{M}} d ^m x \sqrt{g}  \\[2mm]
C _1 (A) &= \frac{ 1 }{4 (4 \pi ) ^{(m-1)/2} } \int _{\partial \mathcal{M} } d ^{m-1} \sqrt{h} 
\, \chi
\label{C _1}
\\[2mm]
C _2 (A) &= \frac{ 1 }{6 (4 \pi) ^{m/2} } 
					\int _{\mathcal{M}} d ^m x\sqrt{g} (6 V + R)  
\nonumber
					\\
& + \frac{ 1 }{6 (4 \pi) ^{m/2} } 
					\int _{\partial \mathcal{M} } d ^{m-1} x 
				\sqrt{h} \ ( 2 L _{aa} + 12 S )
\label{coef}
\end{align} 
donde $C _0$ y $C _1$ representan el volumen de la variedad y del borde, respectivamente; así $g$ y $h$ representan sus tensores métricos, y $m$ la dimension de la variedad. $C _2$, así como el resto de los coeficientes, son funciones del potencial $V$, la curvatura escalar de Ricci $R$ de la variedad, el tensor de curvatura extrínseca $L _{ab } \ (a,b = 1,2,...,m-1)$ sobre el borde de la variedad, y la condición de contorno $S$, definida en la ecuación (\ref{eq.cc}). $\chi$ es una combinacion lineal de proyectores sobre el borde de la variedad, que depende de la condicion de contorno tomada, para su definición específica se puede consultar \cite{Vassilevich:2003xt}.


Utilizando (\ref{eq.heat.expansion}) en  (\ref{eq.mellin}) , se muestra que la función $\zeta (s)$ posee polos simples en
\begin{equation}
s _n = \frac{m-n}{2} 
\label{eq.ceros.zeta}
\end{equation}
con residuos dados por
\begin{equation}
\left. {\rm Res} \ \zeta  (s)  \right| _{s_n= \frac{m - n}{2}} =  
\frac{ C_n  (A) }{ {\Gamma ( \frac{m-n}{2}} ) }
	\, .
\label{losresi}
\end{equation}
En particular, puede verse de (\ref{C_0}) que el residuo en $s=\frac{1}{2}$ es proporcional al volumen de la variedad
\begin{equation}\label{eq.vol}
	{\rm Res} \ \zeta (s) | _{s=1/2} = \frac{ {\rm Vol (\mathcal{M})} }{2 \pi} \, .
\end{equation}


En las secciones siguientes se utilizarán las funciones espectrales $K(T)$ y $\zeta(s) $ para dar definiciones finitas de la acción efectiva y la energía de Casimir.

\section{Regularización de la Acción Efectiva}\label{cap.acc}

En las ecuaciones (\ref{gama-det}) y (\ref{det}) de la sección \ref{accion_efectiva}, se vio que la acción efectiva a {\it 1-loop} está dada por el siguiente determinante funcional,
\begin{equation}\label{eq.regu.gamma}
    \log {\rm Det} \, \delta ^2 S = 
    \sum_{n\in{\mathbb N}} \log \lambda ^2 _n
    \, ,
\end{equation}
en el que $\{\lambda ^2 _n\}_{n\in\mathbb N}$ representa el espectro del operador de fluctuaciones cuánticas $\delta^2S$. Uno de los métodos utilizados para regularizar esta cantidad divergente se basa en el desarrollo de la función-$\Gamma$ incompleta,
\begin{equation}
- \log\lambda ^2 _n=\int _ { \epsilon } ^{\infty}\frac{dT}{T}\ e ^{- T \lambda ^2 _n} +\gamma+\log\epsilon + O ( \epsilon  ) \,.
\end{equation}
Utilizando esta expresión en el determinante funcional  \eqref{eq.regu.gamma} la acción efectiva se puede escribir
\begin{equation}\label{gamma-hk}
\Gamma [ \phi ] = 
S[ \phi ] - 
\frac{\hbar }{2}
\int _ { \epsilon } ^{\infty} \frac{ dT}{T}\ K(T) \, .
\end{equation}
El comportamiento de $K(T)$ para pequeños valores de $T$ determina las divergencias UV de la acción efectiva. Las divergencias infrarrojas (IR) están asociadas con el comportamiento de $K(T)$ para grandes valores de $T$. Como puede verse de \eqref{gamma-hk}, los campos masivos no presentan, en general, divergencias IR.

Se considera, a modo de ejemplo, un campo escalar masivo $\phi(x,t)$ en 1+1 dimensiones con autointeracción $\lambda \phi ^4 $. La acción euclídea está dada por
\begin{equation}
S[ \phi ] = \int dx dt \ 
\frac{( \partial _t \phi ) ^2}{2} +  
\frac{( \partial _x \phi ) ^2}{2} +
\frac{m ^2 }{2} \phi ^2 +
\frac{\lambda}{4!} \phi ^4 \,.
\end{equation}
La segunda variación de la acción resulta
\begin{equation}
\delta ^2 S = 
- \partial _t ^2 
- \partial _x ^2 
+ m ^2 
+ \frac{\lambda}{2}\phi ^2 \,.
\end{equation}


De acuerdo con \eqref{gamma-hk}, la acción efectiva puede expresarse como
\begin{equation}\label{eq.accion-efectiva}
\Gamma [ \phi ] = 
S[ \phi ] - 
\frac{\hbar }{2}
\int _ { \epsilon } ^{\infty} \frac{ dT}{T} 
\ e ^{- T m ^2 }
\,{\rm Tr} \,  e ^{- T ( - \partial _t ^2 - \partial _x ^2 + \frac{\lambda}{2} \phi ^2 ) }
\end{equation}
Esta expresión muestra que un campo escalar masivo no tiene divergencias IR. Sin embargo, existen divergencias UV generadas por el comportamiento del integrando en $T=0$. Utilizando el desarrollo (\ref{eq.heat.expansion}) se puede ver que las contribuciones divergentes están dadas por:
\begin{equation}\label{seeley-div}
- \frac{\hbar }{2}\int _ { \epsilon } ^{1}  
\frac{ dT}{T} 
\left(
		1 - T m^2
		\right)
\left(
		\frac{C _0}{T} + C _2 
		\right),
\end{equation}
donde los términos superiores en el  desarrollo del Heat-Kernel, junto con la exponencial, contribuyen con potencias integrables en el límite $\epsilon \rightarrow 0 $.

Los 3 primeros coeficientes Seeley-DeWitt están dados en las ecuaciones \eqref{C_0}, \eqref{C _1} y \eqref{coef}, para este caso particular ($m=2, V = - \lambda /2,R=0$,variedad sin borde) se obtiene
\begin{align}
C _0 (A) &= \frac{1}{4 \pi   }  \int  _{M} d x dt   \\[2mm]
C _1 (A) &= 0 \\[2mm]
C _2 (A) &= - \frac{ \lambda }{8 \pi }  \int _M d x dt \  \phi (x) ^2 \, .
\label{coef2}
\end{align} 


Estos términos divergentes se introducen en el lagrangiano, redefiniendo la acción efectiva, de modo que resulte finita. $C_0$ renormaliza la constante cosmológica $\Lambda$, mientras que  $C_2$ renormaliza la masa $m$
\begin{equation}
m ^2 _{fis} = m ^2 ( \epsilon )  - \frac{\lambda \hbar \ln \epsilon}{8 \pi} .
\end{equation}
Donde $m _{fis}$, se reinterpreta como la corrección cuántica a la masa original del lagrangiano. Los siguientes coeficientes de  Seeley-DeWitt van a corregir a primer orden $\lambda$ y a orden superior $m$, de la ecuación (\ref{seeley-div}) resulta que estas correcciones son finitas en el límite $\epsilon \rightarrow 0$, por lo tanto es necesario calcular asintóticamente la integral (\ref{eq.accion-efectiva}) en el intervalo $[1, \infty)$, para obtener una corrección término a término.

\section{Regularización de la Energía de Casimir}\label{cap.casimir}

En la sección \ref{sec.casimir} se obtuvo la siguiente expresión para la energía de vacío de un campo escalar sin masa (con dos grados de polarización) entre dos placas paralelas de área infinita,
\begin{equation}
E _0 = \frac{A \hbar c}{(2 \pi) ^2} \int dk _x dk _y 
\sum _{n=1} ^{\infty} 
\sqrt{
		\left( \frac{n \pi}{d} \right) ^2 + k _x ^2 + k _y ^2
		} \, .
\end{equation}
La definición \eqref{funcion.zeta} sugiere reemplazar esta expresión divergente de la energía de vacío por la siguiente representación
\begin{equation}\label{e0-zeta}
E _0 = \frac{A \hbar c}{(2 \pi) ^2} 
\ \zeta \left( - \frac{1}{2} \right)
\end{equation}
donde $\zeta (s)$ está dada por
\begin{equation}
\zeta(s) = \int dk _x dk _y 
\sum _{n=1} ^{\infty} 
\left\{\left( \frac{n \pi}{d} \right) ^2 + k _x ^2 + k _y ^2\right\}^{-s}\,.
\end{equation}
Se calcula ahora la extensión analítica de $\zeta(s)$ en el punto $s=-\frac12$,
\begin{align}
\zeta (s) &= 
\int dk _x dk _y 
\ \sum _{n=1} ^{\infty} 
\left\{	\left( \frac{n \pi}{d} \right) ^2 + k _x ^2 + k _y ^2
		\right\}^{-s} \nonumber\\[2mm]
&=2\pi\ \sum _{n=1} ^{\infty}  \frac12\,\frac{1}{s-1} \left( \frac{n \pi}{d} \right) ^{2-2s} =
\frac{\pi}{s-1} \left( \frac{\pi}{d} \right) ^{2-2s} \zeta_R (2s-2)\,.
\end{align}
En esta expresión $\zeta_R(s)$ representa la función-$\zeta$ de Riemann,
\begin{align}\label{rieman-zeta-def}
	\zeta_R(s)=\sum_{n=1}^\infty n^{-s}\,,
\end{align}
cuya extensión analítica en $s=-3$ es $\zeta_R(-3)=\frac{1}{120}$, tal como está demostrado en el apéndice \ref{Apendice.2}. Utilizando finalmente la representación \eqref{e0-zeta}, se obtiene para la energía de vacío por unidad de área transversal
\begin{equation}
\frac{E _0}{A} = 
- \hbar c\ \frac{ \pi ^2}{720}\ \frac{1}{d^3}\,.
\end{equation}
A partir de esta expresión, se obtiene la fuerza atractiva entre las placas dada por \eqref{casimir.1}.

\bigskip

Como resultado general, se puede redefinir la energía de vacío \ref{eq.casimir.div} de la forma:
\begin{equation}
E _0 = \frac{\hbar}{2} \zeta  \left (- \frac{1}{2} \right) 
\label{eq.casimir.no.mu}
\end{equation}

\bigskip

\section{Adimensionalización}\label{seq.adim}

\medskip

En los capítulos \ref{cap.casimir} y \ref{cap.acc} se utilizó $K(T)$ y $\zeta (s)$ para dar definiciones finitas de la acción efectiva a {\it 1-loop} y la energía de Casimir respectivamente, lo que se va hacer de aquí en adelante, es adimensionalzar la función-$\zeta$, de manera que las potencias complejas estén bien definidas:

\begin{align}
\zeta  (s) &= \sum\limits_{n \in \mathbb N} \left( \frac{\lambda  _n}{\mu }  \right) ^{-2s } \label{def.adim} \, .
\end{align}
En este contexto, la Energía de Casimir queda definida como:
\begin{equation}
E _0 = \frac{\hbar \mu}{2} \zeta  \left( - \frac{1}{2} \right) \,.
\label{eq.casimir.mu}
\end{equation}

En el caso que $\zeta (s)$ tenga una extensión analítica regular en $s=-\frac{1}{2}$, no hay diferencia entre la energía de vacío definida en \eqref{eq.casimir.no.mu}, con la definida en \eqref{eq.casimir.mu}; en cambio, si posee un polo $\mu$ va a contribuir de manera no trivial. Por ejemplo, si $\zeta (s) $ posee un polo simple en $s= - \frac{1}{2}$ con residuo $C _{-1}$, la energía de vacío  \eqref{eq.casimir.mu} queda determinada como:
\begin{equation}
\begin{aligned}
E _0 &= 
\frac{\hbar \mu ^{2s+1}}{2} 
\sum\limits_{n \in \mathbb N}  \lambda _n   ^{-2s } \\ &= 
\frac{\hbar }{2} 
\left(
		\frac{C _{-1}}{  s+1/2 } + 2 C _{-1} \ln \mu + {\rm finito} 
		\right) 
\, ,
\end{aligned}
\end{equation}
donde en el caso particular que  $C _{-1} = 0 $ se recuperara la energía de vacío dada por \eqref{eq.casimir.no.mu}.


En \eqref{e0-zeta} no se introdujo $\mu$ explícitamente dado que $\zeta \left(- \frac{1}{2} \right)$ es regular y esto no afecta el resultado final. A partir del próximo capítulo se utilizará la definición de la energía de vacío dada por \eqref{eq.casimir.mu} en lugar de \eqref{eq.casimir.no.mu}.


\section{Realización numérica}

Como se vio en el presente capítulo, y se seguirá desarrollando en los siguientes, el calculo de la Energía de Casimir siempre requiere regularización, el ejemplo presentado poseía 2 particularidades importantes
\begin{enumerate}
\item Se conocía exactamente el espectro 
\item El espectro era sumable analíticamente.
\end{enumerate}
Existen pocos escenarios donde se puede encontrar una solución analítica al espectro, típicamente potenciales analíticos en regiones regulares, y aun así una vez obtenido el espectro no siempre es posible obtener una expresión completamente analítica de la Energía de Casimir, esto se desarrollará con mas detalle en el capítulo \ref{cap.singular}.

Para estos casos se puede desarrollar un formalismo que permite la evaluación numérica de la Energía de Casimir, el cual se basa en tomar la ecuación  \eqref{eq.accion-efectiva} y transformarla en una integral de caminos.

\begin{equation}
\Gamma \left[ V \right] = 
\left<
-  \frac{1}{8 \pi}
\int _{\frac{1}{\Lambda}} ^{\infty} \frac{dT}{T^ {2}} e ^{- m^2 T}
\int dx \left( W _v [ y(t); x] - 1\right)
\right> _{y} 
 ,
\end{equation}
donde $W _v [ y(t); x]$ es una cierta funcional que depende del potencial, y los valores medios son calculados tomando un ensamble de lineas de mundo.








