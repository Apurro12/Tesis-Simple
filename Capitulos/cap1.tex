\chapter{Introducción}


\section{ Aplicaciones Físicas de la Regularización }

En Teoría Cuántica de Campos el calculo de ciertas magnitudes físicas (Acción efectiva, energía de Casimir, Amplitudes de dispersión, Numero Fermionico, etc. ) conducen a valores formalmente divergentes, por lo cual se requiere un método para extraer resultados finitos. El presente capítulo se aplicaran los metodos regularización Heat-Kernel y Función $ \zeta _A (s) $ al calculo de la energía de Casimir y la acción Efectiva, junto con las herramientas para regularizarlas.\\





\textbf{Acción Efectiva:}\\

Lo siguiente vale tanto para Campos Escalares, como para Campos Fermionicos o Campos de Gauge, pero por simplicidad de la notación el desarrollo va a hacerse sobre Campos Escalares.

Consideremos una teoría de campo escalar $\phi(x)$ definidos sobre una variedad $M$ con borde $\partial M$, donde $x \in  M$. \\

La solución clásica de movimiento $ \phi _0 (x) $ va a estar dada por la que minimice la acción del campo:

\begin{equation}
\left. \frac{\delta S [ \phi ] }{\delta \phi (x)}  \right| _{\phi = \phi _0  } = 0
\end{equation}


Dependiendo de que tipo de teoría se tenga, se van a obtener distintos lagrangianos, lo cual conduce a distintas ecuaciones de movimiento, por ejemplo para campos Escalares, Bosonicos y Fermionicos en el Vacío, se obtienen las Ecuaciones de Klein Gordon, Maxwell y Dirac respectivamente.

\begin{equation}
\begin{array}{c}
\mathscr{L} = \frac{1}{2 c^2} (\partial _t \phi ) ^2 - \frac{1}{2} (  \nabla \phi ) ^2 - 
	\left( \frac{m c}{2 \hbar}  \right) ^2 \phi ^2 
\rightarrow 
\left(
	\partial _t ^2 - \nabla ^2 + m^2 
		\right) \phi = 0 \\[8pt]
		
\mathscr{L} = - \frac{1}{4 \mu _0} F _{\mu \nu} F ^{\mu \nu}
\rightarrow \partial _{\mu} F ^{\mu \nu} = 0 \\[8pt]

\mathscr{L} =  { \bar{\psi} } \left(
			i \hbar c \gamma ^{\mu} \partial _{\mu} - m c^2 
			\right) \psi 
\rightarrow
			\left( i \hbar  \gamma ^{\mu} \partial _{\mu}  - m c  \right)\psi = 0\\[10pt]
\end{array}
\label{campos}
\end{equation}




En Teorías Cuantícas de Campos, el campo se reinterpreta como un operador ($\phi \rightarrow \hat{\phi}$), cuya dinamica seguira estando dada por \ref{campos}. \\




Otra forma alternativa de obtener la dinamica de los campos $\hat{\phi } (x)$, es mediante la formulación mediante Integrales de Caminos dada por Richard Feynman. \\

En esta formulación, toda la información que contiene la teoría con todos los efectos cuánticos está dada por los valores medios de los campos ordenados temporalmente actuando sobre el vacio, dada una fuente externa $J(x)$ estos valores quedan expresados como:

\begin{equation}
\begin{array}{c}
< 0 | \hat{ \phi  } (x _1) .... \hat{\phi  } (x _n) | 0 > = \frac{1}{Z[J]} 
\int D \phi \ e ^{- S[ \phi ] + (J, \phi )} \phi (x _1) ... \phi (x _n) \\[10pt]
\end{array}
\label{valor}
\end{equation}

Donde N es una constante de normalización, $S[\phi]$ es la acción clasica del campo, $(.,.) $ es el producto interno de funciones sobre la variedad, en nuestro caso está dado por $(\phi,J) = \int J(x) \phi (x) d^n x$, y $Z[J]$ es la Funcional Generatriz que será definida a continuación.\\




Para poder calcular (\ref{valor}), se define la Funcional Generatriz, dada por:

\begin{equation}
Z [J] = 
\int D \phi \ e ^{- S[ \phi ] + (J, \phi )}
\label{eq.generatriz}
\end{equation}

Una vez calculada la funcional generatriz, los valores medios (\ref{valor}) quedan determinados por sus derivadas funcionales.

\begin{equation}
\begin{array}{c}
< 0 | \hat{ \phi  } (x _1) .... \hat{\phi  } (x _n) | 0 > = 
\frac{\delta ^n Log Z[J] }{ \delta J(x1) ... \delta J(x _n) } \\[10pt]
\end{array}
\end{equation}


Se define entonces el campo medio como:

\begin{equation}
\begin{array}{c}
\phi _J (x) \equiv < \hat{\phi } (x) > = \frac{\delta Log Z[J] }{\delta J(x)}  \\[10pt]
\end{array}
\end{equation}



Así como la ecuación de movimiento para el campo clásico estaba dada por la acción, el campo $ \phi _J (x) $ va a satisfacer una ecuación de movimiento dada por una acción efectiva $ \Gamma [\phi _J] $, la cual a orden $\hbar$ coincide con la acción clasica.


\begin{equation}
\begin{array}{c}
\frac{\delta \Gamma [ \phi _J ]  }{\delta \phi _J (x)  } = 
J (x) \\[8pt]
\underset{ \hbar \rightarrow 0 }{ Lim  } \Gamma [ \phi  ] = S [ \phi ]
\end{array}
\label{eq.accion1}
\end{equation}

Definiendo la acción efectiva como (\ref{efectiva}) se puede ver que cumple (\ref{eq.accion1}):

\begin{equation}
\Gamma [\phi _J] = (J, \phi _J) - Log Z [J]
\label{efectiva}
\end{equation}

Para demostrar esto se toma su derivada funcional evaluada en el campo medio $\phi _J $.


\begin{equation}
\frac{\delta \Gamma [ \phi _J ]}{\delta \phi _J (x) } = 
J(x) + \int dx ' \frac{\delta J [\phi _J ]}{\delta \phi _J (x) } \phi _J (x) - 
\frac{1}{Z[J]} \int dx' \frac{\delta Z[J] }{\delta J(x')} \frac{\delta J[\phi _J ]}{\delta \phi _J (x)} = J(x) \\[8pt]
\end{equation}


Se puede obtener un desarrollo en potencias de $\hbar$ de $\Gamma [ \phi _J]$ , sin necesidad de calcular explicitamente la Funcional Generatriz. \\


Para obtener dicha expresión se hace el cambi de variables $\phi (x) \rightarrow \phi(x) + \phi _J (x) $ en (\ref{eq.generatriz}), con el que se obtiene desarrollando la acción a orden 2:


\begin{equation}
\begin{array}{c}
Z[J] = e ^{-S[ \phi _J ] + (J, \phi _J )} 
\int D \phi e ^{ -(\delta S  - J, \phi ) - \frac{1}{2}  (\phi, (\delta ^2 S, \phi ) ) }
\end{array}
\end{equation}

Donde:

\begin{equation}
\begin{array}{ccc}
\delta S(x) = \left. \frac{\delta S[\phi]}{ \delta \phi (x) } \right| _{\phi = \phi _J} & &
\delta ^2 S(x,y) = \left. \frac{\delta ^2 S[\phi]}{ \delta \phi (x1) \delta \phi (x) } \right| _{\phi = \phi _J}
\end{array}
\end{equation}

Haciendo el cambio de variables $\phi (x) \rightarrow \sqrt{\hbar} \phi (x) $ se obtiene para la acción efectiva la expresión:


\begin{equation}
\Gamma [\phi _J ] = S [ \phi _J ] - Log \ \int D \phi \ e ^{- \frac{1}{2}  (\phi, (\delta ^2 S, \phi) } + O(\hbar ^2)
\end{equation}


Lo cual puede reescribirse como:

\begin{equation}
\Gamma [\psi] = S [\psi] + \frac{\hbar}{2} Log \ Det ( \delta ^2 S ) +
O ( \hbar ^2 )
\end{equation}


Si el operador $ \delta ^2 S $ tiene una base completa de autofunciones $ \{ \lambda _n \} _{n \in N}$ su determinante se puede escribir como:

\begin{equation}
Det \ \delta ^2 S = \underset{ n \in N }{ \Pi } \ \lambda _n
\end{equation}

Donde por lo general conduce a una cantidad divergente que debe ser regularizada.\\



\textbf{Energía de Casimir:} \\ 

En el año 1948 Hendrik Casimir tomando la idea de que dos moleculas neutras se atraen debido a las fuerzas de Van der Waals, llego a la conclucion que dos placas melaticas paralelas neutras en el vacio sufren una fuerza atractiva dada por (\ref{casimir.1}) aunque , la primer medicion del efecto fue en 1958 por Sparnaay que no pudo ser conclusivo debido al 100 \% de insertidumbre en la medicion, se han echo varias mediciones para distintas configuraciones (ya que la fuerza de Casimir depende la geometria), el experimento mas preciso fue echo por U. Mohideen y Anushree Roy, con una diferencia entre teoria y experimento del 1 \%, en la cual midieron la fuerza entre una esfera metalica y una placa plana, donde se utilizo un microscopio de fuerza atómica \cite{BORDAG20011} .


\begin{equation}
\begin{array}{c}
F(d) = - \frac{\pi ^2 \hbar c}{240} \frac{A}{d^4} \\
\end{array} 
\label{casimir.1}
\end{equation}




En el caso de dos placas paralelas el efecto casimir se puede ver, calculando el valor de expectación del vacío del campo entre las placas, el campo va a satisfacer la ecuación de Klein-Gordon.

\begin{equation}
( \partial _0 ^2 - \nabla  ^2  ) \phi (\vec{x} ,t) = 0 
\end{equation}

Descomponiendo al campo en modos normales de oscilación, llego a la ecuación de autovalores:

\begin{equation}
\begin{array}{c}
\phi ( \vec{x},t) = e ^{-i \omega t} \phi ( \vec{x}) \\
\nabla ^2 \phi ( \vec{x}) = - \frac{\omega ^2}{c ^2} \phi ( \vec{x})
\end{array}
\end{equation}

Imponiendo que la función de onda se anule sobre los bordes conductores, y condiciones de contorno periódicas sobre la dirección transversal (para luego tomar el limite yendo a infinito), los modos normales de oscilación están dados por:

\begin{equation}
\omega _n = c \sqrt{ k _x ^2 + k _y ^2 + \left( \frac{n \pi}{L} \right) ^2 }
\end{equation}

Teniendo en cuenta las dos polarizaciones posibles del campo electromagnético obtengo para la energía de vacío:

\begin{equation}
E _0 = \frac{A \hbar }{(2 \pi) ^2} \int dk _x dk _y 
\sum _{n=1} ^{\infty} 
c
\sqrt{
		\left( \frac{n \pi}{a } \right) ^2 + k _x ^2 + k _y ^2
		}
\end{equation}


Lo cual conduce a una energía que es divergente, la fuerza estaría dada por la derivada respecto de $L$, para obtener un resultado finito se requiere un proceso de regularización.

\section{Heat Kernel y Función Zeta}


En los trabajos \cite{ Seeley:1967ea,10.2307/2373309,10.2307/2373312} se han estudiado trazas de operadores diferenciales $A$ con coeficientes derivables, actuando sobre variedades compactas $M$ con borde suave $\partial M$, una de las funciones espectrales que se pueden definir es lo que se llama $\zeta _A (s)$ (Función zeta del operador A), si el espectro del operador $A$ está dado por $ \{ \lambda _n \} _{n \in N}$ la función $\zeta _A (s)$ queda expresada como:


\begin{equation}
\zeta _A (s) = Tr A ^{-s} = \sum _{n \in N}  \lambda _n ^{-s}
\label{funcion.zeta}
\end{equation}

La cual en principio converge para valores grandes de $s$, pero una vez calculada se puede hacer la prolongación analítica al plano complejo, en particular la función $\zeta _A (s)$ va a tener polos simples en $x _n$ dados por:   :

\begin{equation}
x _n = \frac{m-n}{d} 
\label{eq.ceros.zeta}
\end{equation}

Donde $m$ es la dimension de la variedad,$d$ el orden del operador A y $n= 0,1,2,3 ...$ \\

%En esta tesis se estudiaran los polos de la funcion $\zeta _A (s)$ de un operador singular unidimensional de segundo orden.
Tambien es posible definir otra funcion dependiente el espectro de A, La traza del Heat-Kernel \cite{VASSILEVICH2003279}, que está dada por:

\begin{equation}
K (t) =  Tr \ e ^{-t A} = 
\sum _{n  \in N} e ^{-t \lambda _{n} }
\end{equation}

En el caso que el operador diferencial $A$ sea del tipo Laplace actuando con condiciones de contorno local, sobre campos escalares $\phi $ :

\begin{equation}
\begin{array}{c}

A = - \left(
			g ^{\mu \nu} \nabla _{\mu} \nabla _{\nu} + V
			\right) \\
\left (\partial _m + S \right) \phi | _{\partial M} = 0

			

\end{array}
\end{equation}

Donde $\nabla$ es la derivada covariante, $V$ es el potencial y $\partial _m$ es la derivada normal con respecto al borde, $K(t)$ admite un desarrollo asintótico para valores pequeños de t  de la forma:

\begin{equation}
K(t) \approx 
\sum _{n=0} ^{\infty}
C _n (A) \ 
t ^{\frac{(-m+n)}{2}}
\label{eq.heat.expansion}
\end{equation}



Los primeros 5 terminos $C _n (A) $ están calculados en \cite{VASSILEVICH2003279}, la forma explicita de los primeros 3 es: 

\begin{equation}
\begin{array}{c}
C _0 (A) = (4 \pi ) ^{-m/2}  \int _M d ^m x \sqrt{g}  \\
C _1 (A) = \frac{(4 \pi) ^{-(m-1)/2} }{4} \int _{\partial M } d ^{m-1} \sqrt{h} \\
C _2 (A) = \frac{(4 \pi) ^{-m/2} }{6} \left(
									\int _M d ^m x\sqrt{g} (6 V + R) +
									\int _{\partial M } d ^{n-1} x 
									\sqrt{h} (2 L _{aa}  + 12 S)
									\right)
\end{array}
\end{equation} 

Donde $C _0$ y $C _1$ representan el volumen de la variedad y del borde respectivamente, $C _2$ así como el resto de los coeficientes son funciones del potencial $V$, el Campo de Gauge $\omega $, la condición de contorno $S$, en tensor de curvatura de la variedad $R _{\mu \nu \rho \sigma }$ y el tensor de curvatura extrínseca $K _{\mu \nu }$ sobre el borde de la variedad. \\

A su vez función $\zeta _A (s) $ y $K(t)$ están relacionadas a través de la Transformada de Mellin.



\begin{equation}
\zeta (s) = \frac{1}{\Gamma (s) } 
\int _0 ^{\infty} dt \
t ^{s-1} K(t) 
\end{equation}

Se puede ver entonces que residuos de la función $\zeta _A (s)$ están dados por:

\begin{equation}
Res[ \zeta _A (s) ] | _{s= m/2 - n/2} =  n _ {(A) } {\Gamma (n/2 + m/2)}
\end{equation}

Lo cual coincide con (\ref{eq.ceros.zeta}) para un operador del tipo Laplace. \\

\textbf{Regularización de la Acción Efectiva:} \\

Si el operador $\delta ^2 S$ tiene autovalores $\lambda _n$ la primer corrección a la acción efectiva puede expresarse como:

\begin{equation}
Log \ Det \ \delta ^2 S = 
\sum _n Log( \lambda _n )
\end{equation}

Donde puedo usar el desarrollo de la función Gamma incompleta, para expresar $Log ( \lambda _n )$:

\begin{equation}
\int _ { \epsilon } ^{\infty} \frac{e ^{- T \lambda _n}}{T} dT =
- \left(
		\gamma + Log ( \lambda  ) + Log ( \epsilon  ) + O ( \epsilon  ) 
		\right)
\end{equation}

Como ni $ \epsilon $ ni $ \gamma $ entran en juego en la acción efectiva, puedo reemplazar:

\begin{equation}
\Gamma [ \phi ] = 
S[ \phi ] - 
\frac{\hbar }{2}
\int _ { \epsilon } ^{\infty} \frac{ dT}{T} Tr \  e ^{- T \delta ^2 S}
\end{equation}

Dependiendo del tipo de acción y si el campo presenta o no masa, voy a tener divergencias ultravioletas o infrarrojas, en caso de que tenga masa, no se van a presentar divergencias infrarrojas y las divergencias UV van a poder ser interpretadas como correcciones cuánticas al lagrangiano de partida. \\


Como ejemplo se va a tomar el problema $\lambda \phi ^4 $.

Su Acción viene dada por:

\begin{equation}
S[ \phi ] = \int dx dt \ 
\frac{( \partial _t \phi ) ^2}{2} +  
\frac{( \partial _x \phi ) ^2}{2} +
\frac{m ^2 }{2} \phi ^2 +
\frac{\lambda}{4!} \phi ^4 
\end{equation}

Calculando la segunda variación de la acción obtengo:

\begin{equation}
\delta ^2 S = 
- \partial _t ^2 
- \partial _x ^2 
+ m ^2 
+ \frac{\lambda}{2}\phi ^2 
\end{equation}

La primer corrección a la acción efectiva para le potencial $\lambda \phi ^4 $ viene entonces dada por:

\begin{equation}
\Gamma [ \phi ] = 
S[ \phi ] - 
\frac{\hbar }{2}
\int _ { \epsilon } ^{\infty} \frac{ dT}{T} 
e ^{- T m ^2 }
Tr \  e ^{- T ( - \partial ^2 + \frac{\lambda}{2} \phi ^2 ) }
\end{equation}

Dada la exponencial generada por la masa, puedo ver que la acción efectiva no posee divergencias infrarrojas, el comportamiento divergente va a estar dado en el limite $T \rightarrow 0$, voy entonces a desarrollar el Heat-Kernel en este limite:



\begin{equation}
\begin{array}{c}
Tr \  e ^{- T ( - \partial ^2 + \frac{\lambda}{2} \phi ^2 ) } \approx
\frac{1}{4 \pi}
\int
\frac{  dx dt }{T}
\left(
1  -
T  \frac{\lambda}{2} \phi ^2  +
T ^2 \frac{\lambda ^2 }{8} \phi ^4 + O ( \phi ^6 T ^3)
\right)

\end{array}
\end{equation}

Donde en general, el desarrollo del Heat-Kernel va a ser un desarrollo en potencias de $ \lambda \phi ^2 T $, insertando el desarrollo hasta este orden en la acción efectiva obtengo:

\begin{comment}

\begin{equation}
\begin{array}{c}
\int _ { \epsilon } ^{\infty} \frac{ dt}{t} 
e ^{- t m ^2 }
Tr \  e ^{- t ( - \partial ^2 + \frac{\lambda}{2} \phi ^2 ) } = \\
\int _ { \epsilon } ^{1} \frac{ dt}{t} 
e ^{- t m ^2 }
Tr \  e ^{- t ( - \partial ^2 + \frac{\lambda}{2} \phi ^2 ) } + 
\int _ { 1 } ^{\infty} \frac{ dt}{t} 
Tr \  e ^{- t ( - \partial ^2 + m^2 + \frac{\lambda}{2} \phi ^2 ) }

\end{array}
\end{equation}

\end{comment}



\begin{equation}
\begin{array}{c}
\Gamma [ \phi ] = 
\int dx dt  \\
\left(
\frac{( \partial _t \phi ) ^2}{2} +  
\frac{( \partial _x \phi ) ^2}{2} +
\frac{m ^2 }{2} \phi ^2 +
\frac{\lambda}{4!} \phi ^4 
						\right)  \\
- \frac{1}{8 \pi}
\left(
	\frac{\lambda \phi ^2 Log( \epsilon )}{2}  + \frac{ \lambda ^2 \phi ^4 }{m}
	\right) + O ( \phi ^6)

\end{array}
\end{equation}

Así los términos que acompañan a $\phi ^2 $ y $\phi ^4 $ se reinterpretan como correcciones cuánticas a la masa y a la constante de acoplamiento $\lambda $:

\begin{equation}
\begin{array}{c}

\Gamma [ \phi ] = 
\int dx dt 
\frac{ ( \partial \phi ) ^2 }{2 } +
\frac{\phi ^2}{2} m ^2 + O ( \phi ^6 ) + .... 
+ \frac{\phi ^4}{4!} \lambda \\ \\
m ^2 _{fis} = m ^2 - \frac{\lambda phi ^2 Log( \epsilon )}{8 \pi} \\
\lambda _{fis} = \lambda + \frac{3 \lambda ^2}{m} 


\end{array}
\end{equation}

Donde $m _{fis}$ y $ \lambda _{fis} $ se reinterpretan como los parámetros originales del lagrangiano, a medida que valla corrigiendo la acción efectiva, también se van a ir corrigiendo la masa y $\lambda $ orden a orden.

\textbf{Regularización de la Energía de Casimir}

Al inicio de la sección se calculó la energía de Casimir como:

\begin{equation}
E _0 = \frac{A \hbar }{(2 \pi) ^2} \int dk _x dk _y 
\sum _{n=1} ^{\infty} 
c
\sqrt{
		\left( \frac{n \pi}{a } \right) ^2 + k _x ^2 + k _y ^2
		}
\end{equation}

Para regularizarla voy a calcular la función $\zeta _A (s)$ del problema usando coordenadas polares:

\begin{equation}
\begin{array}{c}

\zeta _A (s) = 
\int dk _x dk _y 
\sum _{n=1} ^{\infty} 
\left(	\left( \frac{n \pi}{a } \right) ^2 + k _x ^2 + k _y ^2
		\right) ^{-s} = \\
\sum _{n=1} ^{\infty}  \frac{\pi}{s-1} \left( \frac{n \pi}{a} \right) ^{-2s+2} =
\frac{\pi}{s-1} \left( \frac{\pi}{a} \right) ^{2-2s} \zeta (2s-2) 

\end{array}
\end{equation}

La Energía de Casimir, se corresponderá con $\zeta _A (-1/2)$, obteniendo 


\begin{equation}
\zeta _A (-1/2) = 
- \frac{\pi ^4}{180 a ^3}
\end{equation}

La Energía de Casimir queda definida por:

\begin{equation}
E _0 =  \frac{A c \hbar}{(2 \pi) ^2}
\zeta _A (-1/2) =
- \frac{A \hbar c \pi ^2}
		{720 L ^3}
\end{equation}

Obteniendo una fuerza atractiva dada por:

\begin{equation}
F(L) = - \partial _L E _0 (L) = 
- \frac{A c \pi ^2 \hbar}{240 L^4}
\end{equation}

Que coincide con lo expresado al principio de la sección. \\ \\


En el ejemplo anterior la función $\zeta _A (s) $ era regular en $s= -1/2$, entonces no fue necesario introducir un regulador, para obtener la función $\zeta _A (s)$ adimensional para todos los valores de $s$, se la define de la siguiente forma (suponiendo que $A$ tiene autovalores $\lambda _n ^2 $):

\begin{equation}
\zeta _A (s) = \sum _{n \in N} \left( \frac{\lambda _n}{\mu }  \right) ^{-2s } = 
\mu ^{2s} \sum _{n \in N } \lambda _n ^{-2s}
\end{equation}

Donde $\mu $ tiene unidades de $longitud ^{-1}$ . \\

Para que coincida con $\underset{ {n \in N}}{  \sum } \lambda _n$ en $s= -1/2$, se procede a definir la energía de vacío como:

\begin{equation}
E _ 0 = 
\frac{\hbar c}{2 }
\left(
	\mu ^{2s+1} \sum _{n \in N} \lambda _n ^{-2s} 
	\right) _{s=-1/2}
\end{equation}