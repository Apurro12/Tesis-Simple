\chapter{Introduccion}


\section{Aplicaciones Físicas de la Regularizacion}

En Teoría Cuántica de Campos el calculo de ciertas magnitudes físicas (Acción efectiva, energía de Casimir, Espectacion del Vacio, etc) conducen a valores formalmente divergentes, en lo siguiente se hará un resumen de la energía de Casimir, y la Accion Efectiva, junto con los metodos regularizacion.\\





\textbf{Acción Efectiva}\\


En Teorías Cuantícas de Campos, la Funcional Generatriz es el elemento basico mediante el cual se calculan todas las magnitudes físicas, utilizando el tiempo euclideo está dada por:

\begin{equation}
Z[J] = 
\int D \phi
e^{-\frac{1}{\hbar} S[\phi] + \frac{1}{\hbar} (J,\phi) }  =
e ^{-\frac{1}{\hbar} \Gamma[J] + \frac{1}{\hbar} (J,\phi )}
\end{equation}

Voy a hacer una traslacion del campo $\phi (x) = \phi _0 (x) + \psi (x) $, donde $\phi _0 (x)$ es la solucion clasica a la ecuacion de movimiento (la que extremiza la accion clasica) y $\psi (x)$ es la fluctuacion cuantica respecto a esa solucion, obteniendo:

\begin{equation}
Z[j] = 
\int D \phi
e^{-\frac{1}{\hbar} S[\phi _0 + \psi] + \frac{1}{\hbar} (J,\phi _0 + \psi ) }  
\end{equation}

Desarrollando la accion $S[\phi _0 + \psi]$ alrededor de $\phi _0$, y utilizando de $\phi _0$ es la solucion claisca, se obtiene para la accion fectiva:

\begin{equation}
\Gamma [\psi] = 
- \hbar 
Log
\left[
	e ^{-\frac{1}{\hbar} S[\phi _0] }
	\int D \phi \
	e ^{-\frac{1}{2 \hbar} (\phi,A \psi) }
	\right]
\end{equation}



Obteniendo en la aproximacion a 1-loop para la accion efectiva:

\begin{equation}
\Gamma [\psi] = S [\psi] + \frac{\hbar}{2} log \ Det (A) +
O ( \hbar ^2 )
\end{equation}

Donde el termino divergente proviene del determinante funcional el cual debe ser correctamente regularizado. \\


\textbf{Energía de Casimir} \\

En el año 1948 Hendrik Casimir tomando la idea de que dos moleculas neutras se atraen debido a las fuerzas de Van der Waals, llego a la conclucion que dos placas melaticas paralelas neutras en el vacio sufren una fuerza atractiva dada por (\label{casimir.1}) aunque , la primer medicion del efecto fue en 1958 por Sparnaay que no pudo ser conclusivo debido al 100 \% de insertidumbre en la medicion, se han echo varias mediciones para distintas configuraciones (ya que la fuerza de Casimir depende la geometria), el experimento mas definitivo fue echo por U. Mohideen y Anushree Roy, con una diferencia entre teoria y experimento del 1 \%, en la cual midieron la fuerza entre una esfera metalica y una placa plana, donde se utilizo un microscopio de fuerza atomica 


\begin{equation}
\begin{array}{c}
F(d) = - \frac{\pi ^2 \hbar c}{240} \frac{A}{d^4} \\
\end{array} 
\label{casimir.1}
\end{equation}




En el caso de dos placas paralelas el efecto casimir se puede ver, calculando el valor de espetacion del vacio del campo entre las placas, el campo va a satisfacer la ecuacion de Klein-Gordon.

\begin{equation}
( \partial _0 ^2 - \nabla  ^2  ) \phi (\vec{x} ,t) = 0 
\end{equation}

Descomponiendo al campo en modos normales de oscilacion, llego a la ecuacion de autovalores:

\begin{equation}
\begin{array}{c}
\phi ( \vec{x},t) = e ^{-i \omega t} \phi ( \vec{x}) \\
\nabla ^2 \phi ( \vec{x}) = - \frac{\omega ^2}{c ^2} \phi ( \vec{x})
\end{array}
\end{equation}

Imponiendo que la funcion de anule sobre los conductores, y condiciones de contorno periodicas sobre la direccion transveral (para luego tomar el limite yendo a infinito), los modos normales de oscilacion estan dados por:

\begin{equation}
\omega _n = c \sqrt{ k _x ^2 + k _y ^2 + \left( \frac{n \pi}{L} \right) ^2 }
\end{equation}

Teniendo en cuenta las dos polarizaciones posibles del campo electromagnetico obtengo para la energía de vacio:

\begin{equation}
E _0 = \frac{A \hbar }{(2 \pi) ^2} \int dk _x dk _y 
\sum _{n=1} ^{\infty} 
c
\sqrt{
		\frac{n \pi}{a } + k _x ^2 + k _y ^2
		}
\end{equation}


Lo cual conduce a una energía que es divergente, la fuerza estaría dada por la derivada respecto de $L$, para obtener un resultado finito se requiere un proceso de regularizacion.

\section{Heat Kernel y Funcion Zeta}

Dado un operador diferencial $A$ con coeficientes derivables, actuando sobre una variedad compacta $M$ con borde suave $\partial M$, se puede definir la traza del operador $A ^{-s}$ lo que se llama, $\zeta _A (s)$ (Funcion zeta del operador A), si el espectro del operador $A$ está dado por $ \{ \lambda _n \} _{n \in N}$ la funcion $\zeta _A (s)$ queda expresada como:


\begin{equation}
\zeta _A (s) = Tr A ^{-s} = \sum _{n \in N}  \lambda _n ^{-s}
\end{equation}

La cual en principio converge para valores grandes de $s$, pero una vez calculada se puede hacer la prolongacion analítica al plano complejo, en particular la funcion va a tener polos simples en $x _n$ dados por:

\begin{equation}
x _n = \frac{m-n}{d} 
\end{equation}

Donde $m$ es la dimension de la variedad,$d$ el orden del operador A y $n= 0,1,2,3 ...$

%En esta tesis se estudiaran los polos de la funcion $\zeta _A (s)$ de un operador singular unidimensional de segundo orden.
Tambien es posible definir otra funcion dependiente el espectro de A, La traza del Heat-Kernel, que está dada por:

\begin{equation}
Tr e ^{-t A} = 
\sum _{n  \in N} e ^{-t \lambda _{n} }
\end{equation}

El cual tiene la propiedad de admitir un desarrollo asintótico para valores pequeños de t de la forma:

\begin{equation}
Tr e ^{-t A} \approx 
\sum _{n=0} ^{\infty}
c _n (A) \ 
t ^{\frac{(-m+n)}{2}}
\end{equation}

Donde n es la dimension de la variedad.

La funcion $\zeta _A (s) $ y $Tr e ^{-A t}$ estan relacionadas a travez de la Transformada de Mellin.

\begin{equation}
\zeta (s) = \frac{1}{\Gamma [s] } 
\int _0 ^{\infty} dt
t ^{s-1} Tr e ^{-A t} 
\end{equation}

Se puede ver entonces que residuos de la funcion $\zeta$ estan dados por

\begin{equation}
Res[\zeta _A (s)] | _{s= s_0} = \frac{C _n (A)}{\Gamma (s _0)}
\end{equation}

\textbf{Regularizacion de la Accion Efectiva:} \\

Inspirados en las propiedades de las matrices se puede dar una definiciones al determinante funcional, de tal forma que sea convergene, para matrices de $ NxN$ se tiene la proiedad

\begin{equation}
Log [ Det \left[ \ e ^B \right] ] = Tr [B]
\end{equation}

Escribiendo entonces $A= e ^{B} \rightarrow A ^{-s} = e ^{-s B}$, se ve que:

\begin{equation}
\partial _s A ^{-s} | _{s=0} = (- B e ^{-s B}) | _{s=0} = - B
\end{equation}

Juntando todo obtengo:

\begin{equation}
Log [Det [A ]] = Tr[B] = - Tr [\partial _s A ^{-S} | _{s=0} ]= - \partial _s Tr[A ^{-s}] | _{s=0}
\end{equation}

Puedo definir entonces el determinante funcional de un operador como:

\begin{equation}
Det[A] = e ^{- \zeta _A ' (0)}
\end{equation}

Lo cual conduce a una contribucion finita a 1-loop de la accion efectiva (siempre que $\zeta _A (0)$ sea regular). \\

\textbf{Regularizacion de la Energía de Casimir}

Al inicio de la seccion se calculó la energía de Casimir como

\begin{equation}
\frac{E _0}{A} = \frac{\hbar }{(2 \pi) ^2} \int dk _x dk _y 
\sum _{n=1} ^{\infty} 
c
\sqrt{
		\frac{n \pi}{a } + k _x ^2 + k _y ^2
		}
\end{equation}

Para regularizarla voy a calcular la funcion $\zeta _A (s)$ del problema:

\begin{equation}
\zeta _A (s) = 
\int dk _x dk _y 
\sum _{n=1} ^{\infty} 
c ^{-2s}
\left(	\frac{n \pi}{a } + k _x ^2 + k _y ^2
		\right) ^{-2s}
\end{equation}

La Energía de Casimir, se correspondera con $\zeta _A (-1/2)$, obteniendo 


\begin{equation}
\zeta _A (-1/2) = 
- \frac{\pi ^4}{180 L ^3}
\end{equation}

La Energía de Casimir queda definida por:

\begin{equation}
E _0 =  \frac{A \hbar}{(2 \pi) ^2}
\zeta _A (-1/2) =
\frac{\hbar }{(2 \pi )^2} 
\left(
	- \frac{c \pi ^2 \hbar}{720 L ^3}
	\right)
\end{equation}

Obteniendo una fuerza atractiva dada por:

\begin{equation}
F[L] = - \partial _L E _0 (L) = 
- \frac{A c \pi ^2 \hbar}{240 L^4}
\end{equation}

Que coincide con lo expresado al principio de la seccion.