\chapter{Introduccion Física}

El Lagrangiano de un campo escalar en unidimensional, $\phi (x,t)$ en un potencial externo está dado por:

\begin{equation}
    L = \frac{1}{2} (\partial _t \phi  ) ^2
    - \frac{1}{2} (\partial _x \phi ) ^2 - 
    \frac{1}{2} m ^2 \phi ^2  - \frac{V}{2} \phi ^2 
\end{equation}

Del cual se pueden obtener las ecuaciones de movimiento dadas por:


\begin{equation}
    ( \partial _t ^2 - \nabla ^2 + m ^2 +V  ) 
    \phi = 0
\end{equation}

Descomponiendo el campo en modos normales de vibracion  $\phi (x,t) = \phi(x) e ^{i \omega t}$ y considerando a mi particula sin masa, llego a la ecuacion

\begin{equation}
   A \phi(x) =  ( - \nabla ^2 + V(x) \ ) \phi (x) = \omega ^2 \phi (x)  
\end{equation}


Donde tengo un problema de Schrodinger con autovalor $\omega  ^2$

Una vez obtenidos todos los autovalores $\omega ^2 _n $, para obtener la energía de vació debo sumar sobre todos los modos normales de oscilación, obteniendo la expresión: 

\begin{equation}
    E _0 = 
    \frac{\hbar}{2}  \sum _n \omega _n 
\label{eq.vacio}
\end{equation}

Donde la expresión anterior generalmente es divergente, para poder calcular la suma voy a tener que calcular la función $\zeta _A (s) $ definida en (\ref{eq.zeta.1}) para luego hacer la prolongación analítica, quedando definida mi energía de vació expresada por (\ref{eq.vacio}).  


\begin{equation}
\begin{array}{c}
    \zeta _A (s) = \sum _n ( \omega _n ^{2} ) ^{-s} = 
    L ^{2 s} \ \sum _n  \mu _n ^{-2 s}  \\
    Donde \ \mu \ es \ adimensional
\end{array}
\label{eq.zeta.1}
\end{equation}

\begin{equation}
    E _0 = \frac{\hbar}{2}  \ \zeta _A (-\frac{1}{2})
\label{eq.vacio}
\end{equation}

Lo cual conduce al resultado con las unidades correctas




















