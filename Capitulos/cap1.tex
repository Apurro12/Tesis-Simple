\chapter{Introduccion}


\section{Aplicaciones Físicas de la Regularizacion}

En Teoría Cuántica de Campos el calculo de ciertas magnitudes físicas (Acción efectiva, energía de Casimir, Espectacion del Vacio, etc) conducen a valores formalmente divergentes, en lo siguiente se hará un resumen de la energía de Casimir, y la Accion Efectiva, junto con los metodos regularizacion\\





\textbf{Accion Efectiva}\\

En Teorias Cuanticas de Campos la probabilidad de transicion de una configuracion de un campo está dada por:

\begin{equation}
< \phi _f (x) | e ^{- \frac{i H t}{\hbar}} | \phi _i (x) > =
\int D \phi \ e ^{- \frac{i S[\phi]}{\hbar}}
\end{equation}

La funcional Generatriz es el elemento basico mediante el cual se calculan todas las magnitudes físicas:

\begin{equation}
Z[j] = 
\int D \phi
e^{-\frac{1}{\hbar} S[\phi] + \frac{1}{\hbar} (J,\phi) }  
\end{equation}

Voy a hacer una traslacion del campo $\phi (x) = \phi _0 (x) + \psi (x) $, donde $\phi _0 (x)$ es la solucion clasica a la ecuacion de movimiento (la que extremiza la accion clasica) y $\psi (x)$ es la fluctuacion cuantica respecto a esa solucion, obteniendo:

\begin{equation}
Z[j] = 
\int D \phi
e^{-\frac{1}{\hbar} S[\phi _0 + \psi] + \frac{1}{\hbar} (J,\phi _0 + \psi ) }  
\end{equation}

Desarrollando la accion $S[\phi _0 + \psi]$ alrededor de $\phi _0$, y utilizando de $\phi _0$ es la solucion claisca, se obtiene para la accion fectiva:

\begin{equation}
\Gamma [\psi] = 
- \hbar 
Log
\left[
	e ^{-\frac{1}{\hbar} S[\phi _0] }
	\int D \phi \
	e ^{-\frac{1}{2 \hbar} (\phi,A \psi) }
	\right]
\end{equation}



Obteniendo en la aproximacion a 1-loop para la accion efectiva:

\begin{equation}
\Gamma [\psi] = S [\psi] + \frac{\hbar}{2} log \ Det (A) +
O ( \hbar ^2 )
\end{equation}

Donde el termino divergente proviene del determinante funcional el cual debe ser correctamente regularizado


\textbf{Energía de Casimir} \\

En el año 1948 Hendrik Casimir tomando la idea de que dos moleculas neutras se atraen debido a las fuerzas de Van der Waals, llego a la conclucion que dos placas melaticas paralelas neutras en el vacio sufren una fuerza atractiva dada por (\label{casimir.1}) aunque , la primer medicion del efecto fue en 1958 por Sparnaay que no pudo ser conclusivo debido al 100 \% de insertidumbre en la medicion, se han echo varias mediciones para distintas configuraciones (ya que la fuerza de Casimir depende la geometria), el experimento mas definitivo fue echo por U. Mohideen y Anushree Roy, con una diferencia entre teoria y experimento del 1 \%, en la cual midieron la fuerza entre una esfera metalica y una placa plana, donde se utilizo un microscopio de fuerza atomica 


\begin{equation}
F(d) = - \frac{\pi ^2 \hbar c}{240} \frac{A}{d^4}
\label{casimir.1}
\end{equation}




En el caso de dos placas paralelas el efecto casimir se puede ver, calculando el valor de espetacion del vacio del campo entre las placas, el campo va a satisfacer la ecuacion de Klein-Gordon

\begin{equation}
( \partial _t ^2 - \partial _x ^2  ) \phi (x,t) = 0 
\end{equation}

Descomponiendo al campo en modos normales de oscilacion, puedo llegar a la ecuacion de autovalores:

\begin{equation}
\begin{array}{c}
\phi (x,t) = e ^{-i \omega t} \phi (x) \\
\partial _x ^2 \phi (x) = - \omega ^2 \phi (x)
\end{array}
\end{equation}

Donde imponiendo las condiciones de contorno que el campo se anule en las placas $\phi (0) = \phi (L) = 0$ obtengo para los modos normales

\begin{equation}
\omega _n = \frac{2 \pi n}{L}
\end{equation}

El valor de espectacion del vacio va a estar dado entonces, por la suma de todos los modos normales de oscilacion:

\begin{equation}
E _0 = \frac{\hbar}{2} \sum \frac{2 n \pi}{L} 
\end{equation}

Lo cual conduce a una energía que es claramente divergente, por lo cual tambien se necesita un proceso de regularizacion.

\section{Heat Kernel y Funcion Zeta}

Dado un operador diferencial $A$ con coeficientes derivables, actuando sobre una variedad compacta $M$ con borde suave $\partial M$, se puede definir la traza del operador $A ^{-s}$ lo que se llama, $\zeta _A (s)$ (Funcion zeta del operador A), si el espectro del operador $A$ está dado por ${\lambda _n} _{n \in N}$ la funcion $\zeta _A (s)$ queda expresada como:


\begin{equation}
\zeta _A (s) = Tr A ^{-s} = \sum _{n}  \lambda _n ^{-s}
\end{equation}

La cual en principio converge para valores grandes de $s$, pero una vez calculada se puede hacer la prolongacion analítica al plano complejo, en particular la funcion va a tener polos simples en $x _n$ dados por:

\begin{equation}
x _n = \frac{m-n}{d} 
\end{equation}

Donde $m$ es la dimension de la variedad y $d$ el orden del operador A.

%En esta tesis se estudiaran los polos de la funcion $\zeta _A (s)$ de un operador singular unidimensional de segundo orden.
Tambien es posible definir otra funcion dependiente el espectro de A, La traza del Heat-Kernel, que está dada por:

\begin{equation}
Tr e ^{-t A} = 
\sum _{n} e ^{-t \lambda _n}
\end{equation}

El cual tiene la propiedad de admitir un desarrollo asintótico para valores pequeños de t de la forma:

\begin{equation}
Tr e ^{-t A} \approx 
\sum _{n=0} ^{\infty}
c _n (A) 
t ^{\frac{(-n+k)}{2}}
\end{equation}

Donde n es la dimension de la variedad.

La funcion $\zeta _A (s) $ y $Tr e ^{-A t}$ estan relacionadas a travez de la Transformada de Mellin.

\begin{equation}
\zeta (s) = \frac{1}{\Gamma [s] } 
\int _0 ^{\infty}
t ^{s-1} Tr e ^{-A t} dt
\end{equation}

Se puede ver entonces que residuos de la funcion $\zeta$ estan dados por

\begin{equation}
Res[\zeta _A (s)] | _{s= s_0} = \frac{C _n (A)}{\Gamma (s _0)}
\end{equation}

\textbf{Regularizacion de la Accion Efectiva:}

Inspirados en las propiedades de las matrices se puede dar una definiciones al determinante funcional, de tal forma que sea convergene, para matrices de $A $ de $ nxn$ se tiene la proiedad

\begin{equation}
Log [ Det e ^A] = Tr [A]
\end{equation}

Escribiendo entonces $B= e ^{A} \rightarrow B ^{-s} = e ^{-s A}$, se que ve y poniendolo en la ecuacion anterior:

\begin{equation}
\partial _s B ^{-s} | _{s=0} = (- A e ^{-s A}) | _{s=0} = - A
\end{equation}

Juntando todo obtengo:

\begin{equation}
Log [Det [e ^A ]] = Tr[A] = - Tr \partial _s B ^{-S} | _{s=0} = - \partial _s Tr[B ^{-s}] | _{s=0}
\end{equation}

Puedo definir entonces el determinante funcional de un operador como:

\begin{equation}
Det[A] = exp ^{- \zeta ' (0)}
\end{equation}

La primer correccion a la accion efectiva va a estar dada por 

\begin{equation}
\frac{\hbar}{2} Log Det A = - \frac{\hbar}{2} \partial _s \zeta' (0) =
- \frac{\hbar}{2} Log[ \mu ] \zeta (0) - \frac{\hbar}{2} \zeta '(0) 
\end{equation}

\begin{equation}
Log Det (A) = 
- \mu ^{2s}  \Gamma (s) \zeta (s) | _{s=0}
\end{equation} 

Donde $\zeta (0) = c _m (A) $ donde $m$ es la dimension de la variedad 

Tambien la puedo definir de la forma

\begin{equation}
Log Det (A/ \mu ^2) = 
- \zeta ' (0) - 2 Log ( \mu ) \zeta (0)
\end{equation}




La Accion Efecitva viene dada por

\begin{equation}
w = \frac{1}{2} Log[Det D]
\end{equation}

El cual conduce a sumas que son formalmente divergentes, en este trabajo se utilizaran funciones espectrales las cuales se definen a travez de las trasas de funciones del operador D \\






