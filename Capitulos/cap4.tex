\chapter{Aproximación númerica utilizando Monte Carlo}
{\label{cap.singular}}

En el presente capitulo se utilizarán simulaciones numéricas para obtener los mismos resultados  (poner figuras), 



\section{Algoritmo de las lineas-v}


Dados dos puntos inicial y final $ y_0$, $ y_N$ respectivamente, el algoritmo se basa en crear un ensamble de lineas que unan estos dos puntos siguiendo una distribución de velocidad gaussiana.
 


\begin{equation}
\mathcal{N} \int _{y_0} ^{y_N} \mathcal{D} y 
e^{-\frac{N}{4} S _{W} [y]} :=
\mathcal{N} \int \prod _{j=1} ^{N-1} d ^{D} y _{j} e^{- \frac{N}{4} \sum _{i=1} ^{N} ( y _i - y _{i-1} )^2 } ,
\end{equation}

el objetivo es realizar un sistema de cambios lineales de manera que la distribución de probabilidad se vuelve puramente guassiana, con este fin se completan cuadrados para la variable $ y_1$


\begin{equation}
S _{W} = 	2 \left( y _1 - \frac{y_0 + y_2}{2} \right) ^2 + 
			\frac{1}{2} \left( y ^2 _2 + y _0 ^2 \right)   -
			y _0 y_2 +
			\sum _{i = 3} ^{N} (y _i - y _{i-1}) ^2 
\end{equation}
Donde existe un solo término que contiene $y_0$, se define entonces la variable
\begin{equation}
z _1 := y_1 - \frac{y _0  + y_2 }{2} ,
\end{equation}
realizando el mismo procedimiento para la variable $y_1$ se obtiene


\begin{equation}
S _W = 2 z_1 ^2 +
		\frac{3}{2} \left( y _2 - \frac{y _0 + 2 y_3}{3} \right) ^2 +
		\frac{1}{3} \left( y _3 ^2 + y _0 ^2 \right) -
		\frac{2}{3} y_0 y_3 +
		\sum _{i = 4} ^{N} (y _i - y _{i-1}) ^2 
		,
\end{equation}
se obtiene entonces una forma cuadratica utilizando el cambio de variables 
\begin{equation}
z _2 := y_2 - \frac{y _0  + 2 y_3 }{3} ,
\end{equation}
la expresión general luego de completar cuadrados en la variable $y_i$ queda determinada por 
\begin{equation}
a _i y _i ^2 - 2 y_i ( y _{i+1} + b _i y_0 ) =
a _i \left( y _i - \frac{y _{i+1} + b_i y_0}{a_i} \right) ^2 -
\frac{\left( y _{i+1} + b_i y_0 \right) ^2}{a _i}
\end{equation}


Donde los coeficientes $a_i$ y $b_i$ están dados por el sistema de ecuaciones recurrentes

\[ 
f   ( i t ,\beta )=
\begin{cases} 
	  a_{i+1} = 2 - \frac{1}{a_i},  \, a _1 = 2
\\
	  	  b_{i+1} = \frac{b _i}{a _i},  \, b _1 = 1
   \end{cases}   
\] 

, donde la solución está dada por
\begin{equation}
a _1 = \frac{i+1}{i} , \, b_i = \frac{1}{i} ,
\end{equation}

por lo tanto la forma general de las variables $z _i$ queda definida por
\begin{equation}
z _i = y _i - \frac{y_0}{i+1} - \frac{i}{i+1} y _{i+1} ,
\end{equation}

reescribiendo la acción con estas nuevas variables, el exponente queda diagonalizado de la forma

\begin{equation}
S _{W}  = \sum _{i = 1} ^{N-1} \frac{i +1}{i} z _i ^2 + c y _0 ^2 + d y _N ^2
, 
\end{equation}
los valores $c$ y $d$ no son importantes ya que son constantes que se cancelarán con la correspondiente normalización.
En resumen el algoritmo de las lineas v dados $N-1$ puntos intermedios queda determinado por 
\begin{enumerate}
\item Generar $N-1$ numeros $w_i$ que estén distribuidos de acuerdo a la distribución de velocidades $e^{- w _{i} ^2}$
\item normalizar las variables $w _i$ para obtener las variables auxiliares $z _i$
	\begin{equation}
	z _i = \sqrt{\frac{4}{N}} \sqrt{\frac{i}{i+1}} w _i
	\end{equation}

\item Una vez obtenidos los puntos $ z_i$ se calculan los puntos $ y_i$ por medio de la formula recursiva
	\begin{equation}
	y _i = z _i + \frac{1}{i+1} y_0 + \frac{i}{i+1} y _{i+1} ,
	\end{equation}
\end{enumerate}
en el caso donde $y_0 = y _N$ se obtienen loops cerrados de $N$ puntos.

\section{Cálculo Asintótico de los autovalores}\label{seq.2.asin}


