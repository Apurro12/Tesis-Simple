\chapter{Extensiones Autoadjuntas del Operador}

    En Este capitulo se van a estudiar otras condiciones de contorno que hacen que el operador sea Autoadjunto en $x=0$ ademas de CC Dirichlet.

\section{Artesanal}

Dado el Hamiltoniano $\mathscr{H} = - \partial ^2 _x + \frac{\alpha}{x}  $ voy a buscar condiciones de contorno que lo hagan autoadjunto en $x=0$ que es donde está la singularidad. 

La primer condicion necesaria es que su dominio e imagen pertenezcan a $L _2$.

\begin{equation}
\begin{array}{c}
si, \ \phi \in L _2 \rightarrow \mathscr{H} \phi \in L _2
\end{array}
\end{equation}


Como voy a estudiar el comportamiento en $x=0$, suponiendo que en $x=L$ todo es bien comportado, voy a empezar buscando soluciones analiticas en $x=0$, 

Basandome en el desarrollo de las autofunciones ( \ref{eq.scat} ) a  $x \rightarrow 0 $ 
Suponiendo que la primer funcion decae igual que la primer autofuncion $ y_1 (x) $.

\begin{equation}
\begin{array}{c}
\phi _1 (x \rightarrow 0) \approx x \ \rightarrow \phi _1 \ \in \ L _2 \\
\mathscr{H} [ \phi _1 ] (x \rightarrow 0) \approx \alpha \in L _2
\end{array}
\label{eq.chico1}
\end{equation}

Esto tambien vale para cualquier potencia positiva mayor 1, no vale para funciones que van a una constante debido a que el termino $1/x$ lo vuelve no integrable en el origen, luego se estudiara mas formalmente esta afirmacion.

Visto el comportamiento de la segunda autofuncion, propongo su comportamiento en $x \rightarrow 0 $ sea :

\begin{equation}
\begin{array}{c}
\phi _2 (x \rightarrow 0) \approx 1 + \alpha x Log[x] \\
\mathscr{H} [\phi _2] (x \rightarrow 0 ) \approx \alpha ^2 Log[x] \ \in L _2
\end{array}
\label{eq.chico2}
\end{equation}

Con estas propuestas dos funciones cualquiera en el dominio del operador tienen el desarrollo a $x \rightarrow 0$ :

\begin{equation}
\phi , \chi = \beta , \beta _1 \ x + \gamma , \gamma _1 \ (1+ \alpha x Log[x] )
\label{eq.phi.chi}
\end{equation}

La siguiente condicion es pedir que operador sea simetrico.

\begin{equation}
(\phi, \mathscr{H} \chi ) = ( \mathscr{H} \phi , \chi )
\end{equation}

Para ello voy a pedir que se anulen los terminos de borde

\begin{equation}
\begin{array}{c}
\int _0 ^{L} 
\left(
- \partial ^2 _x + \frac{\alpha}{x}
\right)
\phi ^{*} \chi dx = \\
\phi ^{*'} \chi - \phi \chi ^{* '} + 
\int _0 ^{L} 
\phi ^{*} 
| _{x=0} ^{x=L}
\left(
- \partial _x ^2 + \frac{\alpha}{x}
\right)
\chi dx 
\end{array}
\end{equation}

Suponiendo que $\phi $ y $ \chi $ ya satisfacen la condicion de contorno en $x=L$ e insertando el desarrollo de $\phi$ y $\chi$ dado por ( \ref{eq.phi.chi} ) , obtengo que para que el termino de borde se satisfaga debe cumplirse:

\begin{equation}
\left(
-1 + \alpha x 
\right)
\left(
\beta _1 ^{* '} \gamma - \beta \gamma _1 ^{* '}
\right)
\end{equation}

La cual se satiface si se cumple la doncidion 

\begin{equation}
\gamma = c \beta 
\end{equation}

Entonces toda funcion que esté en el dominio del operador va a tener un desarollo dado por:

\begin{equation}
\phi ( x \rightarrow 0 ) \approx x + c \left(  1 + \alpha x Log[x]   \right)
\end{equation}

Para cada $c$ que yo elija voy a obtener un Hamiltoniano con unas condiciones de contorno que lo vuelvan autoadjunto

\section{Teoria de Von Neumann}


Dado un operador diferencial de segundo orden $\mathscr{H}$ hay que definir cieras condiciones de contorno para que sea Hermitico.

Una forma de encontrar las condiciones de contorno que hacen que un operador sea hermitico (en el caso que estamos estudiando) es que las funciones del dominio se comporten en $x \rightarrow 0$ como:

\begin{equation}
\phi (x) \approx  A \phi (x) _{+} (x) + U(n) A \phi (x) _{-}
\end{equation}
 

Donde $|| \phi _{+/-} (x) || = 1$ y ambas son soluciones de 

\begin{equation}
\mathscr{H} \phi _{+/-} (x) = \pm i \phi _{+/-} (x)
\end{equation}

En general $U(n)$ que es el grupo unitario de orden n, es una isometria entre el las funciones $\phi _{+}$ y $\phi _{-} (x) $ pero como el problema a tratar tiene $n=1$ el unico elemento de $U(1) = e ^{i \gamma}$.


Como mi operador es real, basta hallar $\phi _{+} (x) $ y conjugarlo para hallar $ \phi _{-} (x) $, obtengo entonces 

\begin{equation}
\begin{array}{c}
\phi _{+} (x) = e ^{- \sqrt[4]{-1} x } e ^{i \sqrt{2} x } x . \\
\left(
F _1 ^1 (1 - \frac{ 4]{-1} }{2} \alpha ,2 , -2 (-1) ^{\frac{3}{4} } x )  +
U(1 - \frac{ \sqrt[4]{-1} }{2} \alpha ,2 , -2 (-1) ^{\frac{3}{4} } x )
\right)
\end{array}
\end{equation}

Para que satisfagan la condicion de contorno en $ x=L $ reescribo $ \phi _{+}$ como.

\begin{equation}
\phi _{+} (x) = \phi * _{-} (x) = 
= e ^{- \sqrt[4]{-1} x } e ^{i \sqrt{2} x } x.
\left(
F _1 ^1 (x=L) U(x) - U(x=L) F _1 ^1 (x)
\right)
\end{equation}


Utilizando (REFERENCIAR CAPITULO ANTEIOR), Los desarrollos quedan:

\begin{equation}
\begin{array}{c}
F _1 ^1 (x \rightarrow 0 ) \approx 1 \\
U( \rightarrow 0  ) \approx - \frac{1}{\alpha x \Gamma} + \frac{C}{\Gamma} + \frac{Log[x]}{\Gamma}
\end{array}
\end{equation}


Donnde $C$ es un numero complejo, insertando este desarrollo en (ARRIBA) y multiplicando por $ e ^{- i \gamma /2}$ obtengo:

\begin{equation}
\begin{array}{c} \\
\psi (x \rightarrow 0 )
A N ^{+}
\left(
-2 Re[ \frac{e ^{-i \gamma /2} F}{\alpha \Gamma} ] + 
2 x Re [ e ^{-i \gamma /2} (\frac{C F}{\Gamma} - U ) ] + 
2 x Log[x] Re [ e ^{- i \gamma /2 }]
\right)
\end{array}
\end{equation}

Lo cual luego de redefinir $C$ y $A$ queda 

\begin{equation}
\psi (x \rightarrow 0 ) = A N ^{+} 
\left(
x + C (1- \alpha Log[x] )
\right)
\end{equation}

Lo cual coincide con lo calculado anteriormente.

\section{Autofunciones:}

Las solcuciones de $H \phi (x) = \lambda ^2 \phi (x)$ ya fueron calculadas anteriormente 