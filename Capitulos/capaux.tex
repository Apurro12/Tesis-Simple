\chapter{ahorita}

\section{Parte finita de la energía de vacío la venganza }

La energía de vacío puede expresarse como según REF, al igual que en la sección
\begin{equation}
	\zeta (s) = 
	\frac{1}{2 \pi i} \int _{\mathcal{C}} 
						\lambda ^{-2s}
						\partial _ \lambda 
						\log F _1 ^{1} ( \lambda )
						d \lambda
	\, .
\end{equation}
Utilizando las variables adimensionales $\beta = \alpha L, \mu = \lambda L$ la función $\zeta (s)$ queda expresada como
\begin{align}
	\zeta (s) =& 
	\frac{L ^{2s}}{2 \pi i} \int _{\mathcal{C}} 
	f (\mu , \beta) \mu ^{-2s} d \mu 
\, ,
\end{align}
donde $f( \mu, \beta)$ está dada por
\begin{align}
f(\mu, \beta) =& 	
i
\frac{
		\left(1 + \frac{ \beta}{2 i \mu} \right) 
		F _1 ^1 
			\left( 2 + \frac{ \beta}{2 i \mu} ,3 ,2 i \mu \right)
		+ \left( \frac{\beta				
				}
				{2 \mu ^2 } 
				\right)
				( F _{1} ^1 ) ^{(1,0,0)}
				\left( 1 + \frac{\beta}{2 i \mu} ,2 ,2 i \mu
						\right)
		}
		{F _1 ^1 \left( 1 + \frac{\beta}{2 i \mu},2,2 i \mu \right)} 
\, .		
\nonumber
\end{align}
Utilizando el camino de integración \ref{fig.derecha.derecha} la función $\zeta (s)$ puede expresarse como.
\begin{align*}
\zeta (s) = 
- \frac{L ^{2s}}{2 \pi } 
\Bigg(&	  e ^{- i \pi s} \int _0 ^{C _0}
			f (i t,\beta )
			t ^{-2s}  dt 
		+ e ^{- i \pi s} \int _{C _0} ^{\infty}
			f (i t,\beta )
			t ^{-2s}  dt \\
		+&e ^{-i \pi s} \int _{0} ^{C _0} 
			f (-i t,\beta )
			t ^{-2s}  dt 
		+ e ^{i \pi s} \int _{C _0} ^{\infty}
			f (-i t,\beta )
			t ^{-2s}  dt 
	\Bigg)
\, ,
\end{align*}
Donde la primer y tercer integral pueden calcularse de manera numérica. La segunda y la curta contienen la parte divergente de $\zeta$, para calcularlas se utiliza el desarrollo de  $f(i t,\beta)$ en los límites $t \rightarrow \pm \infty$ dado por la ecuación REFERENCIAR
\[ 
f  _a ( i t ,\beta )=
\begin{cases} 
	  i  \left(
			\frac{1}{t} - \frac{\beta}{2 t ^2 } + \frac{\beta}{2 t^2}
			\log (2 t) - \frac{\beta \gamma}{2 t^2} + 2
			\right) + O (t ^{-3})
	  & t > 0 \\
      i  \left(
			- \frac{1}{t} + \frac{\beta}{2 t ^2 } - \frac{\beta}{2 t^2}
			\log (2 t) - \frac{\beta \gamma}{2 t^2} 
			\right) + O (t ^{-3})
      & t < 0
   \end{cases}   
\]
Utilizando este desarrollo las ultimas dos integrales pueden expresarse como
\begin{align}
\nonumber
	\int _{C _0} ^{\infty}
			f (i t,\beta )
			t ^{-2s}  dt &= 
	\int _{C _0} ^{\infty}
		\left(
			f (it, \beta) - f _a (it, \beta )			
				\right) t ^{-2s} dt 
	\\ &+ 
\nonumber
	\int _{C _0} ^{\infty}
			f _a (i t, \beta)
			 t ^{-2s} dt  \\
\nonumber
	\int _{C _0} ^{\infty}
			f (-i t,\beta )
			t ^{-2s}  dt &= 
	\int _{C _0} ^{\infty}
		\left(
			f (-it, \beta) - f _a (-it, \beta )			
				\right) t ^{-2s} dt 
	\\ &+ 
\nonumber
	\int _{C _0} ^{\infty}
			f _a (-i t, \beta)
			 t ^{-2s} dt
\end{align}
Donde las términos de arriba pueden integrarse numéricamente en $s=-1/2$ dado que son convergentes, las contribuciones divergentes están dadas por
\begin{align}
&	\int _{C _0} ^{\infty}
			f _a (it, \beta )			
			 t ^{-2s} dt 
\\
&	\int _{C _0} ^{\infty}
			f _a (-i t, \beta)
			 t ^{-2s} dt
\end{align}
