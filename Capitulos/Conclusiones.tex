\chapter{Conclusiones}

A lo largo de este trabajo de tesis se estudió la energía de vacío asociada al operador singular $A = - \partial ^2 _x + \frac{\alpha}{x} $ para lo cual se analizaron los polos de su correspondiente función-$\zeta$.

En los capítulos \ref{cap.singular} y \ref{cap.continuo} se tomaron como dominio del operador los intervalos compacto $[0,L]$ y no compacto $[0, \infty)$ respectivamente, en ambos casos se obtuvo un polo simple en $s = \frac{1}{2}$ cuyo residuo está dado por
\begin{equation}
{\rm Res } \ \zeta (s) \big|_{s = \frac{1}{2}}
= \frac{L \mu}{2 \pi} 
= \frac{ {\rm Vol (\mathcal{M})} }{2 \pi} 
\, ,
\end{equation}
donde $\mathcal{M}$ es el volumen de la variedad, este resultado coincide con el presentado en \eqref{eq.vol}, lo cual es importante ya que el operador bajo estudio no cumple con las condiciones requeridas para que su Heat Kernel posea un desarrollo de la forma \eqref{eq.heat.expansion}, lo que abre el interrogante sobre la posible extensión del resultado \eqref{eq.vol} en operadores singulares.



En cuanto al estudio de $\zeta (s= -\frac{1}{2} )$, término presente en la definición de la Energía de Casimir \eqref{e0-zeta}, en ambos casos (intervalo compacto y no compacto) se demostró que tal como ocurre en $s= \frac{1}{2}$ el polo es independiente del intervalo pero a diferencia del caso regular se encontró que posee un polo doble dado por
\begin{align*}
&
	\zeta \left( - \frac{1}{2} + \epsilon \right) = 
	\frac{\alpha}{8 \pi \mu  \epsilon  ^2} +
	\frac{\alpha \left( \log (2 L \mu ) + \gamma -1  \right)}{4 \pi \mu  \epsilon } +
	{\rm PF } \zeta \left( - \frac{1}{2} \right)
\, ,
\end{align*}
lo que demuestra que el desarrollo asintótico del Heat-Kernel presenta términos logaritmicos. 
% tal como se conjeturó en 1980 \cite{callias1980}.


En cuanto al estudio de los demas polos se concluyó que al igual que en el caso regular son todos simples y están ubicados en los semienteros negativos. Utilizando este resultado y el anterior se confirma la conjetura dada en \cite{callias1980} la cual afirma que el Heat-Kernel para operadores singulares posee un desarrollo de la forma  
\begin{equation}
	K(T) \sim 
	\frac{ \log (T)}{T ^{\frac{1}{2} }} C _{2} +
	\sum _{\substack{n=0 \\ n \neq 2}} ^{\infty}
	C _n  \ 
	T^{\frac{(-1+n)}{2}} 
\, ,
\end{equation}
este resultado es muy importante ya que exceptuando el polo en $s = - \frac{1}{2}$ el desarrollo del Heat-Kernel coincide en estructura con el caso regular lo cual no se cumple para otros operadores singulares, un ejemplo de esto es el operador $- \partial ^2 _x + \frac{\alpha}{x ^2}$ en donde se demostró que sus polos pueden caer incluso valores irracionales \cite{doi:10.1063/1.1809257}.


En cuanto al estudio de la parte finita de $\zeta \left( - \frac{1}{2} \right)$ se obtuvieron dos representaciónes, una integral y un desarrollo en serie, dependientes del parámetro adimensional $\beta = \alpha L$, a partir de la cual se generaron curvas de la energía de vacío para distintos valores de $\alpha$ y $L$, se observó que la energía de vacío posee en máximo local en el cual para  $L < L _{critico} $ la fuerza es atractiva y luego se recupera el clásico comportamiento repulsivo, comportamiento también presente en potenciales regulares \cite{Beauregard_2013}.\\

Dado que la energía de casimir depende profundamente de la geometría se a utilizado ampliamente en distintos modelos, en \cite{Blau_1988} se mencionan algunas aplicaciones como por ejemplo, las placas paralelas neutras en el vacio \cite{PLUNIEN198687}, dinamica de campos sobre membranas en teoría $M$ \cite{DEWIT1988545}, modelos de quarks en el nucleo atómico \cite{PhysRevD.14.2622}, etc. En partícular un modelo que se cita es una aproximación semi-clasica de la estructura del electrón \cite{MILTON198049} en el cual se propone un modelo de electrón formado por un cascaron esferico que se mantiene estable debido a un equilibrio entre la  energía electrostática repulsiva y la energía de casimir atractiva del campo en el interior del cascaron.

Inspirado en este último ejemplo se pueden proponer dos interpretaciones, la primera está relacionada con el trabajo original de Casimir \cite{Casimir:1948dh}, en el cual el sistema está formado por dos placas paralelas en el vacío donde una está cargada de tal forma que genera un pontencial electrostático $V (x) = \frac{\alpha}{x}$, en esta interpretación la energía de casimir calculada corresponde a la energía por unidad de area tal como se explica en \cite{Blau_1988}.

La segunda interpretación posible está inspirada en \cite{MILTON198049} y \cite{Beauregard_2013}, donde en vez de pensar que dos placas paralelas interactuan en el vacío, el sistema son dos partículas de igual carga sometidas a su campo electromanético, lo cual para $L _{critico} < L$ genera una fuerza repulsiva como es de esperar, pero a partir de un valor límite $L < L _{critico}$ las partíclas sufren una fuerza atractiva.









