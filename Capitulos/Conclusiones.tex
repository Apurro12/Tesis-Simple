\chapter{Conclusiones}

Se estudió la estructura de polos del operador singular $A = - \partial ^2 + \frac{\alpha}{x} $ en el compacto $[0,L]$ y en el continuo $[0, \infty)$, en ambos casos no solo que se presentó un polo doble en $s= -\frac{1}{2}$, sino que ademas coinciden.
\begin{align*}
&
	\zeta \left( - \frac{1}{2} + \epsilon \right) = 
	\frac{\alpha}{8 \pi \mu  \epsilon  ^2} +
	\frac{\alpha \left( \log (2 L \mu ) + \gamma -1  \right)}{4 \pi \mu  \epsilon } +
	{\rm PF } \zeta \left( - \frac{1}{2} \right)
\, .
\end{align*}
La existencia de un polo doble en $\zeta (s)$ demuestra que el desarrollo asintótico del {\it Heat-Kernel} presenta términos logaritmicos.

En cuanto a los demas polos, se estudio la estructura completa de polos de $\zeta \left(s \right)$ en el compacto $[0,L]$ obteniendo que al igual que en el caso regular, los polos son todos simples (excepto en $s= - \frac{1}{2}$) y están ubicados en los semienteros negativos, lo cual implica tal como se concluyó en \cite{Callias1980} que el {\it Heat-Kernel} solo posee un desarrollo de la forma  
\begin{equation}
	K(T) \sim 
	\frac{ \log (T)}{T ^{\frac{1}{2} }} C ^{(2)} +
	\sum _{n=0} ^{\infty}
	C _n  \ 
	T^{\frac{(-1+n)}{2}} 
\, .
\end{equation}


Luego se estudio a través de dos métodos diferentes la parte finita de $\zeta \left( - \frac{1}{2} \right)$ obteniendo una representación fundamental de la energía de vacío en función de un parámetro adimensional $\beta$ a partir del cual se pudieron generar energías de vacío para distintos valores de $\alpha$ y $L$ observando que la energía de vacío posee en máximo local indicando que para valores pequeños de $L$ la fuerza es atractiva, y luego a partir de un $L$ crítico la fuerza se vuelve repulsiva, comportamiento que también se presenta para potenciales regulares \cite{Beauregard_2013}.

Dado la energía de casimir depende profundamente de la geometría se a utilizado ampliamente en distintos modelos, en \cite{Blau:1988kv} se mencionan algunas aplicaciones como por ejemplo, las placas paralelas neutras en el vacio \CITE{PLUNIEN198687}, dinamica de campos sobre membranas en teoría $M$ \cite{DEWIT1988545}, modelos de quarks en el nucleo atómico \cite{PhysRevD.14.2622}, etc. En partícular un modelo que se cita es un intento semi-clasico de la estructura del electrón \cite{MILTON198049} donde se propone un moelo de electrón donde el electrón es un cascaron esferico que existe un equilibrio entre la  energía electrostática repulsiva y la energía de casimir atractiva del campo en el interior del cascaron.

Inspirado en este último ejemplo se pueden proponer dos interpretaciones, la primera está relacionada con el trabajo original de Casimir \cite{Casimir:1948dh} donde el sistema son dos placas paralelas en el vacío donde una está cargada de tal forma que genera un pontencial electrostático $V (x) = \frac{\alpha}{x}$, en este sentido puede verse que las placas sufren una fuera repulsiva, pero a partir de  $L < L _{critico}$ las placas sufren una fuerza atractiva. El segundo modelo está inspirado en \cite{MILTON198049}, donde en vez de pensar que el sistema son dos placas interactuando, el sistema son dos electrónes sometidos a su campo electromanético, para $L > L _{critico}$ las partículas sufren una fuerza repulsiva, pero cuando pasan el valor límite $L < L _{critico}$ las partíclas sufren una fuerza atractiva.

