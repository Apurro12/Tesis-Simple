\chapter{Conclusiones}

Se estudió la estructura de polos del operador singular $A = - \partial ^2 + \frac{\alpha}{x} $, 
en los capítulos \ref{cap.singular} y \ref{cap.continuo} se tomó un intervalo compacto $[0,L]$ y no compacto $[0, \infty)$ respectivamente.
En ambos casos, el polo en $s = \frac{1}{2}$ estuvo dado por
\begin{equation}
{\rm Res } \ \zeta (s) \big|_{s = \frac{1}{2}}
= \frac{L \mu}{2 \pi}
\, ,
\end{equation}
lo cual coincide con el resultado que se presenta en la introducción \eqref{eq.vol} siendo ambas proporcional al volumen.


En ambos casos no solo que se presentó un polo doble en $s= -\frac{1}{2}$, sino que en ambos el polo coincide.
\begin{align*}
&
	\zeta \left( - \frac{1}{2} + \epsilon \right) = 
	\frac{\alpha}{8 \pi \mu  \epsilon  ^2} +
	\frac{\alpha \left( \log (2 L \mu ) + \gamma -1  \right)}{4 \pi \mu  \epsilon } +
	{\rm PF } \zeta \left( - \frac{1}{2} \right)
\, .
\end{align*}
La existencia de un polo doble en $\zeta \left(s = - \frac{1}{2} \right)$ demuestra que el desarrollo asintótico del {\it Heat-Kernel} presenta términos logaritmicos, tal como se conjeturó en 1980 \cite{Callias1980}


En cuanto a los demas polos, se estudio la estructura completa de polos de $\zeta \left(s \right)$, obteniedno que en ambos intervalos al igual que en el caso regular los polos son todos simples (excepto en $s= - \frac{1}{2}$) y están ubicados en los semienteros negativos, lo cual junto con el resultado anterior confirma la conjetura dada en \cite{Callias1980}, la cual afirma que el {\it Heat-Kernel} para operadores singulares posee un desarrollo de la forma  
\begin{equation}
	K(T) \sim 
	\frac{ \log (T)}{T ^{\frac{1}{2} }} C _{2} +
	\sum _{\substack{n=0 \\ n \neq 2}} ^{\infty}
	C _n  \ 
	T^{\frac{(-1+n)}{2}} 
\, ,
\end{equation}
lo cual es muy interesante ya que excepto por el polo en $s = - \frac{1}{2}$ el {\it Heat-Kernel} coincide con el caso regular, lo cual no se cumple para otros operadores singulares, por ejemplo de la forma $- \partial ^2 _x + \frac{\alpha}{x ^2}$ en donde se demostró que los polos de $\zeta (s)$ pueden caer incluso valores irracionales \cite{doi:10.1063/1.1809257}.



Luego a través de dos métodos diferentes se calculó la parte finita de $\zeta \left( - \frac{1}{2} \right)$ obteniendo una representación fundamental de la energía de vacío en función de un parámetro adimensional $\beta$, a partir de la cual se generaron curvas de la energía de vacío para distintos valores de $\alpha$ y $L$, se observó que la energía de vacío posee en máximo local indicando que para valores pequeños de $L$ la fuerza es atractiva, y luego a partir de un $L _{critico} < L $ la fuerza se vuelve repulsiva, comportamiento que también se presenta para potenciales regulares \cite{Beauregard_2013}.

Dado la energía de casimir depende profundamente de la geometría se a utilizado ampliamente en distintos modelos, en \cite{Blau:1988kv} se mencionan algunas aplicaciones como por ejemplo, las placas paralelas neutras en el vacio \cite{PLUNIEN198687}, dinamica de campos sobre membranas en teoría $M$ \cite{DEWIT1988545}, modelos de quarks en el nucleo atómico \cite{PhysRevD.14.2622}, etc. En partícular un modelo que se cita es una aproximación semi-clasica de la estructura del electrón \cite{MILTON198049} en el cual se propone un modelo de electrón formado por un cascaron esferico que se mantiene estable debido a un equilibrio entre la  energía electrostática repulsiva y la energía de casimir atractiva del campo en el interior del cascaron.

Inspirado en este último ejemplo se pueden proponer dos interpretaciones, la primera está relacionada con el trabajo original de Casimir \cite{Casimir:1948dh}, en el cual el sistema está formado por dos placas paralelas en el vacío donde una está cargada de tal forma que genera un pontencial electrostático $V (x) = \frac{\alpha}{x}$, en esta interpretación la energía de casimir calculada corresponde a la energía por unidad de area \cite{Blau:1988kv}, aquí puede verse que las placas sufren una fuera repulsiva , pero a partir de  $L < L _{critico}$ las placas se atraen. 

La segunda interpretación posible está inspirada en \cite{MILTON198049} y \cite{Beauregard_2013}, donde en vez de pensar que dos placas paralelas interactuan en el vacío, el sistema son dos partículas de igual carga sometidas a su campo electromanético, lo cual para $L _{critico} < L$ genera una fuerza repulsiva como es de esperar, pero a partir de un valor límite $L < L _{critico}$ las partíclas sufren una fuerza atractiva.









