\chapter{Conclusiones}

Se estudío la estructura de polos del operador singular $A = - \partial ^2 + \frac{\alpha}{x} $ en el compacto $[0,L]$ y en el continuo $[0, \infty)$, en ambos casos no solo que se presentó un polo doble en $s= -\frac{1}{2}$, sino que ademas coinciden.
\begin{align*}
&
	\zeta \left( - \frac{1}{2} + \epsilon \right) = 
	\frac{\alpha}{8 \pi \mu  \epsilon  ^2} +
	\frac{\alpha \left( \log (2 L \mu ) + \gamma -1  \right)}{4 \pi \mu  \epsilon } +
	f \left( - \frac{1}{2} \right)
\\
&
	\left| f \left( - \frac{1}{2} \right) \right| < \infty
\end{align*}
Luego se estudio a través de dos métodos diferentes la parte finita de $\zeta \left( - \frac{1}{2} \right)$ obteniendo una representación fundamental de la energía de vacío en función de un parámetro adimensional $\beta$ a partir del cual se pudieron generar energías de vacío para distintos valores de $\alpha,L$.

Se estudio la estructura completa de polos de $\zeta \left( - \frac{1}{2} \right)$ obteniendo que al igual que en el caso regular, los polos son todos simples (excepto en $s= - \frac{1}{2}$) y están ubicados en los semienteros negativos.

